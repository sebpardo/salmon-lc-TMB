\documentclass[12pt]{article}
\usepackage[top=0.85in,left=1.0in,right=1.0in,footskip=0.75in]{geometry}
%\usepackage[parfill]{parskip}
\usepackage{setspace}
\usepackage{lineno}
\usepackage[hidelinks]{hyperref}
%\onehalfspacing
\doublespacing

\usepackage[round,sectionbib]{natbib}
\setcitestyle{authoryear}
\bibpunct{(}{)}{;}{a}{}{;}
\bibliographystyle{icesjms}

%\usepackage{textcomp}
%\usepackage{libertine}
%\usepackage{inconsolata} % sans serif typewriter

\usepackage{mathtools} % for dcases
\usepackage{xcolor} % for textcolor
\usepackage{makecell} % for \makecell and within cell line breaks (\thead)

\usepackage[T1]{fontenc}

%\makeatletter\let\expandableinput\@@input\makeatother % expandable input for \input inside tables

% Linux Libertine:
\usepackage{textcomp}
\usepackage[sb]{libertine}
\usepackage[varqu,varl]{inconsolata}% sans serif typewriter
\usepackage[libertine,bigdelims,vvarbb]{newtxmath} % bb from STIX
\usepackage[vvarbb]{newtxmath} % bb from STIX, removed bigdelims for ScholarOne rendering
\usepackage[cal=boondoxo]{mathalfa} % mathcal
%\useosf % osf for text, not math
%\usepackage[supstfm=libertinesups,%
%  supscaled=1.2,%
%  raised=-.13em]{superiors}

\usepackage{xspace}
\usepackage{xfrac} % for diagonal inline fractions in text
%\usepackage{array} % for making whole row bold in table
\usepackage{colortbl} % for background colours in table rows
\usepackage{longtable}
\usepackage{amssymb} % for \checkmark 
\usepackage{amsmath} % for \checkmark 
\usepackage{rotating}
\usepackage[nolists,tablesfirst]{endfloat} % for putting figs and tables at end of document
\DeclareDelayedFloatFlavor{sidewaystable}{table}
\usepackage{makecell} % for \makecell in tables
\usepackage{doi}

%\usepackage{xr} % to obtain label references from supp materials file
%\externaldocument[S-]{suppmat}

\usepackage[utf8]{inputenc} % to properly print special characters directly

\usepackage{tabularx}
\usepackage{booktabs}
\usepackage{array} % for table wrapping of columns
\newcolumntype{L}[1]{>{\raggedright\let\newline\\\arraybackslash\hspace{0pt}}m{#1}}
\newcolumntype{C}[1]{>{\centering\let\newline\\\arraybackslash\hspace{0pt}}m{#1}}
\newcolumntype{R}[1]{>{\raggedleft\let\newline\\\arraybackslash\hspace{0pt}}m{#1}}
%\usepackage[detect-all]{siunitx} % for SI units

% custom hyphenation:
\hyphenation{inverse}
\hyphenation{At-lantic}
\hyphenation{elasmo-branchs}


% Macros
\newcommand{\So}{$S_{1}$\xspace}
\newcommand{\St}{$S_{2}$\xspace}
\newcommand{\Pg}{$P_g$\xspace}
\newcommand{\prmu}{$\mu_{Pg,r}$\xspace}
\newcommand{\prsig}{$\sigma_{Pg,r}$\xspace}
\newcommand{\Linf}{$L_{\infty}$}
\newcommand{\DWinf}{$DW_{\infty}$}
\newcommand{\alphat}{$\tilde{\alpha}$}
\newcommand{\lamat}{$l_{\alpha_{mat}}$}
\newcommand{\lamatb}{$l_{\alpha_{mat}}b$}
\newcommand{\rmax}{$r_{max}$\xspace}
\newcommand{\ageratio}{$\alpha_{mat}/\alpha_{max}$}
\newcommand{\yr}{year\textsuperscript{-1}}
\newcommand{\rsq}{$R^2$\xspace}
\newcommand{\mytilde}{\raise.17ex\hbox{$\scriptstyle\mathtt{\sim}$}}
% Select what to do with command \comment:  
%\newcommand{\comment}[1]{}  % comment not shown
\newcommand{\comment}[1]{\par {\bfseries \color{blue} #1 \par}} % comment shown
%% END MACROS SECTION

\begin{document}
\linenumbers


\section*{Trends in marine survival of Atlantic salmon populations \\ in eastern Canada}

\textbf{Sebasti\'{a}n A. Pardo\textsuperscript{1*}, 
        Geir H. Bolstad\textsuperscript{2}, 
        J. Brian Dempson\textsuperscript{3}, 
        Julien April\textsuperscript{4}, 
        Ross A. Jones\textsuperscript{5}, 
        %Martha J. Robertson\textsuperscript{3}, 
        Dustin Raab\textsuperscript{6}, 
Jeffrey A. Hutchings\textsuperscript{1}} 

\noindent\small{\textsuperscript{1} Department of Biology, Dalhousie University, Halifax, NS, Canada\\}
\small{\textsuperscript{2} Norwegian Institute for Nature Research (NINA), Trondheim, Norway\\}
\small{\textsuperscript{3} Fisheries and Oceans Canada, St. John's, NL, Canada\\}
\small{\textsuperscript{4} Minist\`{e}re des For\^{e}ts, de la Faune et des Parcs, Qu\'{e}bec, QC, Canada\\}
\small{\textsuperscript{5} Fisheries and Oceans Canada, Moncton, NB, Canada\\}
\small{\textsuperscript{6} Fisheries and Oceans Canada, Dartmouth, NS, Canada\\}
\small{\textsuperscript{*} Corresponding author: spardo@dal.ca}

\section*{Abstract}

Declines in wild Atlantic salmon (\emph{Salmo salar}) abundance throughout the north
Atlantic are primarily attributed to decreases in survival at sea. 
However, comparing trends in marine survival among populations
is challenging as data on both migrating smolts and returning adults is sparse and 
models are difficult to parameterize due to their varied life histories. We fit a hierarchical Bayesian
maturity schedule model to data from seven populations in eastern 
Canada to estimate numbers of out-migrating smolts, 
survival in the first and second year at sea, and the proportion returning after one year. 
Trends in survival at sea were not consistent among populations; we observe 
positive, negative and no correlations in these, suggesting that large
scale patterns of changes in marine survival are not necessarily representative for
individual populations. 
Variation in return abundances was mostly explained by marine survival in the
first winter at sea in all but one population.
However, variation in the other components were not negligible and
their relative importance differed among populations. 
If salmon populations do not respond in a uniform manner to
changing environmental conditions throughout their range,
future research initiatives should explore why.

% Declines in wild Atlantic salmon (\emph{Salmo salar}) abundance throughout the north
% Atlantic are primarily attributed to decreases in survival at sea. 
% Whether these changes in marine survival are highly variable among populations
% has not been straightforward to estimate because of
% the need to model data on both migrating smolts and returning adults and to
% simultaneously estimate multiple parameters. We fit a hierarchical Bayesian
% maturity schedule model to data spanning between 19 and 42 years for seven populations in eastern 
% Canada. We estimate numbers of out-migrating smolts, 
% survival in the first and second year at sea, and the proportion returning after one year. 
% Trends in survival at sea were not consistent among populations, and we observe both
% positive, negative and no correlations in the trends. This suggest that large
% scale patterns of changes in marine survival are not necessarily representative for
% individual populations. 
% Variation in return abundances was mostly explained by marine survival in the
% first winter at sea in all but one population.
% However, variation in the other components were not negligible and
% their relative importance differed among populations. Variation in number of
% out-migrating smolts had a moderate contribution on returns and probability of returning after one year was
% important in some populations. 
% % Trends were not consistent among rivers. 
% % Since 1990, \So increased at Western Arm Brook (NL)
% % and Rivi\`{e}re Saint-Jean (QC), but declined at Conne River (NL) and Rivi\'{e}re
% % de la Trinit\'{e} (QC). Since the mid-1990s, \So increased at Campbellton River (NL),
% % declined at LaHave River (NS), and fluctuated at Nashwaak
% % River (NB). Estimates of \St were highly uncertain, particularly for
% % 1SW-dominated populations; \Pg was generally stable. 
% % These results challenge the narrative that marine survival has changed in a
% % temporally consistent manner among spatially disparate populations. 
% If salmon populations do not respond in a consistent, uniform manner to
% changing environmental (including oceanic) conditions throughout their range,
% future research initiatives should explore why.

% The marine phase of anadromous Atlantic salmon (\emph{Salmo salar}) is the
% least known yet one of the most crucial with regards to population
% persistence. Declines in many Atlantic salmon populations in eastern
% Canada have often been attributed to changes in conditions at sea, negatively
% affecting their survival. However, marine survival estimates are difficult to
% obtain given that many individuals spend multiple winters in the ocean before
% returning to freshwater to spawn, necessitating the estimation of multiple
% parameters. To do so, we fit a hierarchical Bayesian maturity schedule model
% to smolt and adult abundance time series for seven populations located in
% Newfoundland and Labrador (NL), Qu\'{e}bec (QC), New Brunswick (NB), and Nova
% Scotia (NS). The datasets ranged between 19 and 42 years in length. We
% estimated three components of marine survival: survival in the first year at
% sea (\So), survival in the second year at sea (\St), and proportion returning
% as one sea-winter adults (\Pg). Controlling for time frame, trends in
% estimates of \So were not consistent among rivers.
% In the four populations for which data extended to 1990, marine survival
% during the first year at sea predominantly increased over time in Western Arm
% Brook (NL) and Rivi\`{e}re Saint-Jean (QC) but largely declined in Conne River
% (NL) and Rivi\`{e}re Trinit\'{e} (QC). In the three other populations, \So
% exhibited an increase in Campbellton River (NL), a decline in LaHave River
% (NS), and fluctuating stability in Nashwaak River (NB) since approximately the
% mid-1990s. 
% Estimates of \St were highly uncertain, particularly for 1SW-dominated rivers
% (Conne, Campbellton and WAB, where the abundance of 2SW returns is very low)
% and thus the posterior distributions matched our choice of prior and trends
% could not be assessed. 
% \Pg was temporally stable within rivers, except for LaHave and Nashwaak Rivers. 
% Our findings challenge the assumption that marine survival has changed in a consistent
% manner among populations across a broad geographic scale, and suggest that
% trends can be river-specific. 
% The well established correlations between climate variables and abundance are
% not being mediated solely by marine survival, and there can potentially be
% some important indirect effects on fecundity.
% Further work exploring potential correlates of marine survival and the
% potential non-lethal effects of climate effects is warranted.


Keywords: salmonid, survival at sea, natural mortality, marine mortality

%Running Head: 

\section*{Introduction} % (4-5 paragraphs)

%1. Declining salmon numbers

Reductions in fishing mortality, albeit necessary, are not always sufficient
to facilitate population recovery. Experience with numerous commercially
exploited marine fisheries since the early 1990s has shown that not all
populations respond as favourably as anticipated to major reductions in
exploitation \citep{Hutchings2017}. Gradual efforts to close commercial
Atlantic salmon (\emph{Salmo salar}) fisheries in eastern Canada culminated in
full moratoria in all regions, beginning in the Maritime provinces (1984) and
following in Newfoundland (1992), Labrador (1998), and Qu\'{e}bec (2000). Since
these closures, many populations have not increased as 
expected \citep{Dempson2004, ICES2019}; some 
have been assessed as threatened or endangered by the 
Committee on the Status of Endangered Wildlife in Canada \citep[][]{Cosewic2010}, 
Canada's science advisory body (to the national government) on
species risk of extinction.
While the mechanisms behind population declines are not fully understood, the potential
drivers of these are many \citep[see ][for a detailed discussion of possible
causes]{Cairns2001}, including but not limited to: fishing mortality \citep{Dempson2004}, 
damming of waterways and changes in the freshwater habitat \citep{Dunfield1985,Clarke2014a}, 
acidification \citep[particularly in Nova Scotia's Southern Uplands, see][]{Gibson2010}, 
predation by seals and birds \citep{Cairns2000}, negative
effects of interbreeding or interactions with escaped farmed salmon
\citep{Keyser2018}, and climate-driven changes in survival and productivity \citep{Mills2013}.

% 2.
% It is widely perceived that,
Several studies suggest that, over broad spatial scales,
marine survival of Atlantic salmon has declined throughout the North Atlantic,
particularly since the 1980's \citep{Massiot-Granier2014,ICES2019,Olmos2019}.
%\citep{Hansen1998,OMaoileidigh2003,Chaput2012a}.
These declines are thought to be driven by oceanic climate effects 
based on multiple lines of evidence suggesting that climate conditions can directly and
indirectly influence the abundance and productivity of Atlantic salmon
populations \citep{Mills2013,Almodovar2019,Olmos2020}.
Thus, an implicit assumption is that populations sharing a particular oceanic
route during seaward migration are likely to show similar trends in marine
survival. 
Put another way, given that salmon from different rivers 
are hypothesized to share marine habitat during some of their time at sea, it
has been presumed that populations would be more likely to experience similar temporal trends in
at-sea mortality \citep{Friedland1993, Friedland1998, Russell2012}.  
% A recent study by \citet{Olmos2019} suggested that trends in post-smolt
% survival, when estimated at the stock level, are synchronously declining
% for all Atlantic salmon in eastern North America.

% Recently \citet{Olmos2019} suggested that trends in post-smolt
% survival, when estimated at the stock unit level, are synchronously declining
% for all Atlantic salmon in Eastern North America and Canada, a conclusion ultimately 
% grounded on the veracity of highly variable stock-recruitment relationships.

% This perception is widespread both in the scientific
% literature as well as federal reports. The latest status report on Atlantic
% salmon by COSEWIC states that ``While the mechanism(s) of marine mortality is
% uncertain, what is clear is that the recent period of poor sea survival is
% occurring in parallel with many widespread changes in the North Atlantic
% ecosystem.'' \citep{Cosewic2010}. Consequently, there have several lines of
% inquiry as to what the causes behind these declines might be attributable to
% \citep{Friedland1993, Friedland1998}.

Despite the overall decreases in marine survival,
the conservation status of Canadian Atlantic salmon populations differs
considerably. In Canada, populations in the southern part of their range are more
likely to be assessed as being of conservation concern (i.e. Threatened or Endangered) 
than those in more
northerly regions \citep{Cosewic2010}. This latitudinal disparity 
suggests that if marine survival has been, or is, a key factor responsible for
most population declines, these changes are not uniformly distributed across
all populations. 

% Nonetheless, most of the populations declines assessed by COSEWIC have
% occurred in populations in the southern extent of its distribution:
% populations in the Bay of Fundy, Anticosti Island, and the Atlantic coast of
% Nova Scotia being assessed as Endangered, populations in the south coast of
% Newfoundland assessed as Threatened, And the populations in the New Brunswick
% and Qu\`{e}bec coasts of the Gulf of St. Lawrence assessed as Special Concern
% \citep{Cosewic2010}. On the other hand, northermost populations have shown
% stable, or even increasing population trends, suggesting that, if marine
% survival were to be a factor in many of these declines, these changes are not
% uniformly distributed across all populations.

% While the purported decline of marine survival is mentioned widely in the 
% scientific literature, only relatively few studies have quantified it in detail.
% The basis for this premise is a limited number of time series that exhibit
% temporal declines in a proxy of marine survival (i.e. return rates or
% post-smolt survival).

% 3. Nonetheless, this premise has never been examined in detail

Given the logistical challenges associated with estimating at-sea survival for
individual populations, it is not surprising that the number of studies that
have estimated temporal trends has been limited. 
The evidence for widespread declines in marine survival mostly comes from studies
over broad spatial scales that do not include empirical smolt abundance data, and
rely on highly variable stock-recruitment relationships.
An additional limitation has been the derivation of
proxies (e.g., return rates), rather than direct model-based estimates, of
marine survival.
\citet{Chaput2012a}, for example, examined the return rate of smolts to adult salmon 
as a metric of marine survival, finding that most Canadian populations 
had experienced declining return rates. 
However, with the exception of one-sea-winter (1SW) dominated populations 
(such as most populations in Newfoundland) where return rates 
closely approximate marine survival, return rates cannot directly be interpreted as marine survival rates and 
examination of trends in return rates alone can mask changes in
differential survival during different years at sea, as well as changes in the
proportion of adults returning after one or two years at sea \citep{Hubley2011}.

%4. 
In the present study, we compile data on the number of migrating smolts and number of returning adults 
for seven wild Canadian populations of Atlantic salmon to model population-level trends in marine survival
and assess their among-population variation.
While some studies have previously used maturity-schedule models to estimate marine
survival for a limited number of salmon populations \citep{Chaput2003b,Hubley2011}, none
have incorporated data extending over multiple decades, nor have they examined
trends among more than two or three populations. 
Accordingly, we develop a hierarchical Bayesian model that uses Murphy's maturity
schedule method \citep{Murphy1952}, in conjunction with informative priors, to estimate yearly
marine survival in salmon. In addition to accounting for observation error in
smolt and return estimates, we estimate the proportion of salmon returning
after one winter hierarchically.

\section*{Methods}

\subsection*{Data}

We obtained time series data of out-migrating smolt and returning adult
abundances for seven Atlantic salmon populations in eastern Canada, encompassing a
wide range of the species' westerly distribution. 
Populations included the LaHave River in Nova Scotia's (NS) Southern Uplands, 
Nashwaak River in New Brunswick (NB), Rivi\`{e}re de la Trinit\'{e} (Trinit\'{e}) and
Rivi\`{e}re Saint-Jean in Qu\'{e}bec (QC), and Western Arm Brook (also referred to as WAB), 
Campbellton, and Conne River, in Newfoundland and Labrador (NL) (Fig.~\ref{fig:map}). 
Data were collected in NS, NB, and Newfoundland
and Labrador (NL) by Fisheries and Oceans Canada (DFO) and in QC
by the Minist\`{e}re des For\^{e}ts, de la Faune et des Parcs, Qu\'{e}bec.

\begin{figure}[htbp] \centering
    \includegraphics[width=0.85\linewidth]{figures/rivers-map2.png}
    \caption{Locations of the seven rivers in eastern Canada with time series abundance data of out-migrating smolts and 
    returning adult Atlantic salmon.} \label{fig:map} 
\end{figure}

\subsubsection*{Smolt and adult return abundance data}

Smolt and adult return abundance estimates originate from a variety of
sources. Smolt estimates from the Trinit\'{e}, Saint-Jean, LaHave, Nashwaak,
and Conne populations were obtained using a mark-recapture approach, while
estimates from the WAB and Campbellton populations were obtained by direct
counts using fish counting fences.
For further details on the data collection methodologies refer to 
\citet{Dempson1991}, \citet{Schwarz1994}, and \citet{Venoitt2018} for NL populations, 
\citet{April2018}  for QC populations,
\citet{Jones2014} for NB population,
and \citet{Gibson2009} for NS population. 

Annual return data are often recorded in terms of two size groups: small ($< 63$ cm
FL) and large ($\geq 63$ cm FL) salmon, as these closely represent different
life-history strategies (i.e. one- and two-sea-winter), but can be confounded with repeat
spawners of different sizes. To correct for this in returns 
reported as small and large salmon, rather than 1SW and 2SW fish, we estimated the abundance
of 1SW and 2SW returns using yearly scale age data of a subsample of returns:

\begin{equation}
    p_{r,t,a} = \frac{\sum_{s}{(\frac{n_{r,t,s,a}}{n_{r,t,s}} * N_{r,y,s})}}{\sum_{s}{N_{r,t,s}}}
\end{equation}

where $p_{r,t,a}$ is the proportion of annual returns in river $r$, year $t$,
and of spawning history $a$ (either 1SW or 2SW returns); $n_{r,t,s,a}$ is the
number of samples in river $r$, year $t$, of returning age $a$, and of size
group $s$; $n_{r,t,s}$ is the total number of samples in river $r$, year $t$,
and of size group $s$; and $N_{r,t,s}$ is the returns of salmon
in river $r$, year $t$, and of size group $s$.
For years where scale data of a given size group is lacking, 
we averaged the proportions of annual returns
for the closest ten years for which there was data. 
We incorporated the uncertainty in the conversion between
size group and returning age in the model indirectly as an estimate
of the variance of annual log-1SW and 2SW return abundance (see Supplementary Materials).

\subsection*{Bayesian model}

We developed a hierarchical Bayesian model that uses Murphy's maturity
schedule method, in conjunction with informative priors, to estimate annual
marine survival in seven populations of Atlantic salmon. We account for
observation error in smolt and return estimates, as well as estimating the
proportion returning after one winter (i.e. \Pg) hierarchically.
There is an identifiability problem in the maturity schedule equations where
the parameter estimates cannot be optimally solved \citep{Chaput2003a}.
However, this issue can be mathematically overcome by
using informative priors for the two marine survival parameters and the maturation
parameter in a Bayesian framework.
This method requires abundance estimates of smolts as well as abundance estimates
of returning one-sea-winter (1SW) and two-sea-winter (2SW) adults. With these data,
it estimates three parameters: survival in the first year at sea (\So), survival
in the second year at sea (\St), and the proportion of fish returning after one
year at sea (\Pg). 

Our model does not include repeat spawners and assumes that no fish spend
three or more winters at sea before returning to spawn for the first time.
The model also assumes that mortality in the second winter at sea (\St) for 2SW returns
is additive to mortality in the first winter at sea in the previous year, 
and therefore does not account for differences in environmental conditions experienced
between 1SW and 2SW fish of the same smolt cohort during their overlapping first year at sea.
In other words, our model assumes that the decision of returning occurs just before
actually being counted as returns and that \St is additional mortality in
the subsequent year. 

%\comment{Other assumptions...}

Observed smolt estimates were modelled hierarchically and included
observation error:

\begin{equation}
log(smolts_{obs,t,r}) = log(smolts_{true,t,r}) + \epsilon_{t,r},
\end{equation}

where $smolts_{true,t,r}$ are the true smolt abundances for year $t$ and river
$r$, and $\epsilon_{t,r}$ is the error term, which is calculated from the yearly 
coefficient of variation in the empirically derived
smolt estimates (see Table S1 in the Supplementary material). 
Where available, we used population-specific measurement error estimates for smolt abundances; if not 
available, we set measurement error between 5 and 10\% (see Table S2 in the Supplementary material). 
The log-transformed true smolt abundances are
normally distributed around a population-level mean and standard deviation:

\begin{equation}
log(smolts_{true,t,r}) \sim Normal(\mu_{smolts,r}, \sigma_{smolts,r})
\end{equation}

where $\mu_{smolts,r}$ and $\sigma_{smolts,r}$ are the mean and standard
deviation of the of the hierarchical population-level log-smolt abundances
estimated by the model for each population $r$.

Once we have yearly estimates of smolt, 1SW, and 2SW abundances, we estimate
marine survival parameters using Murphy's maturity schedule method
\citep{Murphy1952, Ricker1975}:

\begin{align}
    R_{1,t} &= smolts_{true,t-1} * S_{1,t} * P_{g,t} \label{eq:1}, \\
    R_{2,t+1} &= smolts_{true,t-1} * S_{1,t} * (1 - P_{g,t}) * S_{2,t+1}, \label{eq:2}
\end{align}

where $R_{1,t}$ and $R_{2,t+1}$ are the estimated abundances of 1SW and 2SW
salmon returning in years $t$ and $t+1$, respectively, $smolts_{true,t-1}$ is the
estimated number of out-migrating smolts in year $t-1$, $S_{1,t}$ is the proportion of
salmon surviving in their first year ($t$) at sea, $P_{g,t}$ is the proportion of
salmon that return to spawn at year $t$, $S_{2,t+1}$ is the survival in their
second year at sea of the same cohort of salmon who did not return to spawn at
year $t$.

We log-transform equations~\ref{eq:1} and~\ref{eq:2} so the model 
is linear on the log-scale:

\begin{align}
    log(R_{1,r,t}) &= log(smolts_{r,t-1}) + log(Pr_{r,t}) - Z_{1,r,t} \label{eq:3}, \\
    log(R_{2,r,t+1}) &= log(smolts_{r,t-1}) - Z_{1,r,t} + log(1 - Pr_{r,t})  - Z_{2,r,t+1} \label{eq:4} 
\end{align}

where $Z_{1,r,t}$ and $Z_{2,r,t+1}$ are the instantaneous mortality rates and
$R_{1,r,t}$ and $R_{2,r,t+1}$ are the estimated 1SW and 2SW returns in
consecutive years, respectively. Observation error was included as the
standard deviation of the log-transformed return estimates from
equation~\ref{eq:4}:

\begin{align}
log(R_{obs,1,t,r}) &\sim Normal(log(R_{1,t,r}), \epsilon_{1,t,r}), \\
log(R_{obs,2,t,r}) &\sim Normal(log(R_{2,t,r}), \epsilon_{2,t,r}) \label{eq:5} 
\end{align}

where $R_{obs,1,t,r}$ and $R_{obs,2,t,r}$ are the observed return estimates
for year $t$ and river $r$ of 1SW and 2SW fish, respectively, $\epsilon_{1,t,r}$
and $\epsilon_{2,t,r}$ are the process error terms. 
These error terms are estimated by approximating the abundance of 1SW and 2SW returns
from hypergeometric distributions based on the scale sample data.
We assume that the error in the total number of returns is zero; 
the error in the total number of returns is likely minor
compared to the error in the number of outmigrating smolts and the error due to
a small scale sample size. Hence, ignoring the error in the returns is
unlikely to change our results significantly.
We were able to obtain scale data for all but two of the populations (LaHave and Nashwaak), and we 
bootstrapped a distribution of abundances of annual 1SW and 2SW returns for each populations from hypergeometric distributions,
from which we estimated the variance in log-space.
See the Supplementary Materials for a description of the methodology and the resulting estimates.

Furthermore, we use instantaneous mortality rates in the model instead of
survival probabilities as the model is more efficient in its parameter search
in log-space. Instantaneous rates are easily converted to survival
probabilities by 

\begin{align}
 S_{1} &= e^{-Z_1}, \\
 S_{2} &= e^{-Z_2}. 
\end{align}

The priors for $Z_1$ and $Z_2$ are specified as log-normal distributions:

\begin{align}
Z_1 &\sim logNormal(1, 0.22),   \\ 
Z_2 &\sim logNormal(0.2, 0.3).
\end{align}

These priors, when converted to yearly survival, are roughly constrained between
0 and 0.2 for \So and between 0.2 and 0.5 for \St (Fig.~\ref{fig:priors}a). They are
informative only to the extent that they limit the estimates of marine survival
to what we considered to be biologically realistic.

We estimate population-level mean logit \Pg values around which the yearly \Pg
values are normally distributed. We specify different informative hyperpriors
for \prmu and \prsig based on whether the population is 1SW-dominated (i.e.
with a proportion of 1SW fish in the total returns greater than 0.9) or not:

\begin{align}
    logit(P_{g,t}) &\sim Normal(\mu_{Pg,r}, \sigma_{Pg,r}) \\
    \mu_{Pg,r} &\sim 
    \begin{cases}
       Normal(2.3, 0.4),  &\text{for 1SW-dominated populations} \\
       Normal(0, 2.8), &\text{for non-1SW-dominated populations} \\
   \end{cases} \\
    \sigma_{Pg,r} &\sim halfNormal(0, 1).
\end{align}

The three Newfoundland populations (WAB, Campbellton and Conne) are 
1SW-dominated, while the other four have the more generic priors for $logit(P_{g,t})$.
These priors, when converted back to a proportion, are narrowly constrained
for 1SW-dominated populations, but relatively wide for all other populations
(Fig.~\ref{fig:priors}b).

\begin{figure}[htbp] \centering
    \includegraphics[width=0.55\linewidth]{figures/priors.png} \caption{Priors
        for a) marine survival of one sea-winter (\So) and two sea-winter
        (\St) returns, and b) probability of returning as 1SW for
        1SW-dominated populations (proportion 1SW > 0.9) and other populations.}
    \label{fig:priors} 
\end{figure}


\subsection*{Correlations among trends in survival and parameters}

We looked at the correlation among trends in \So between populations by
directly estimating the error-corrected correlation for each posterior
iteration and then calculating the distribution of correlation values:

\begin{equation}
cor_{corrected}(Z_{1,r=1},Z_{1,r=2}) = \frac{cov(Z_{1,r=1}, Z_{1,r=2}) - cov_{error}(Z_{1,r=1}, Z_{1,r=2})}
    {\sqrt{var(Z_{1,r=1}) - var_{error}(Z_{1,r=1})}*\sqrt{var(Z_{1,r=2}) - var_{error}(Z_{1,r=2})}}, \label{eq:corz1}
\end{equation}

% \begin{equation}
% cor(Z_{1,r=1},Z_{1,r=2}) = \frac{cov(Z_{1,r=1}, Z_{1,r=2})}{\sqrt{var(Z_{1,r=1})}*\sqrt{var(Z_{1,r=2})}}\label{eq:corz1-uncorrected}
% \end{equation}

where $Z_{1,r=1}$ and $Z_{1,r=2}$ are the $Z_1$ estimates, across years, for a
single posterior iteration for populations 1 and 2,
$cov(Z_{1,r=1}, Z_{1,r=2})$ is the covariance between these overlapping $Z_1$ estimates, 
$cov_{error}(Z_{1,r=1}, Z_{1,r=2})$ is the error covariance term,   
$var(Z_{1,r=1})$ and $var(Z_{1,r=2})$ are the variances, while
$var_{error}(Z_{1,r=1})$ and $var_{error}(Z_{1,r=2})$ and are error variance estimates.
Estimates of the error covariance ($cov_{error}$) and error
variances ($var_{error}$) are obtained by estimating the covariance of
two parameters across all values in the posterior for a given year (rather
than across years for each posterior iteration), which results in one
covariance estimate for each year in each river, which then are averaged
across years for each river. Likewise, the error variance of a parameter is estimated  
across all values in the posterior for a given year and then are averaged
across years for each river.
Given that the time series do not cover the same years, and that some populations
have missing years in the middle of the time series, only years with overlapping
$Z_1$ estimates were used for the pairwise correlation estimates, and thus
each pair-wise correlation is specific for those years and does not include sample uncertainty due to sample size.

We also estimated the correlation between the model parameters for each population using
the same approach as in equation~\ref{eq:corz1},
by including the error in the variance and covariance estimates:

\begin{equation}
cor_{corrected}(P_{g,r},-Z_{1,r}) = \frac{cov(P_{g,r}, -Z_{1,r}) - cov_{error}(P_{g,r}, -Z_{1,r})}
{\sqrt{var(P_{g,r})- var_{error}(P_{g,r})}*\sqrt{var(-Z_{1,r}) - var_{error}(-Z_{1,r})}}\label{eq:corparam}.
\end{equation}

Lastly, to determine what parameters best explains the number of returns in
the model (including the model uncertainty), we estimated the variance in returns explained by each parameter by
calculating the squared correlation between the estimated parameters and the
estimated 1SW and 2SW returns ($R_{1,r,t}$ and $R_{2,r,t+1}$ in
equation~\ref{eq:4}).

The model was written in Stan \citep{Carpenter2017} and all analyses were run in R version 3.6.1
\citep{RCoreTeam2019} using the \texttt{rstan} package version 2.19.2
\citep{StanDevelopmentTeam2019}.
The model was run with three chains and 3,000 iterations, with the first 1,500
discarded as a burn-in. The models were considered to have converged when the
$\hat R$ of all parameters were lower than 1.03 and the effective sample size 
were higher than 500. The data and code are available at \url{https://github.com/sebpardo/salmon-marine-survival}.

% \begin{equation}
% cor_{error}(P_{g,r},-Z_{1,r}) = \frac{cov_{error}(P_{g,r},-Z_{1,r})}{\sqrt{var_{error}(P_{g,r})}*\sqrt{var_{error}(-Z_{1,r})}}
% \end{equation}


\section*{Results}

%\subsubsection*{Model convergence}

\subsubsection*{Trends in marine survival parameters}

Estimates of marine survival probabilities in the first winter at sea (\So) were 
highly variable within and among populations
(Fig.~\ref{fig:s1-dual}). The highest median posterior estimates of \So
were for the Nashwaak River in 2006 and 2008, with values of 0.18 and 0.21,
respectively. The lowest median \So estimate was in the Trinit\'{e} in 2001,
with an estimate of \So of 0.007, while the Conne, LaHave, and Trinit\'{e} had
years where the lowest estimates of \So varied between 0.01 and 0.02 (Fig.~\ref{fig:s1-faceted}).

Long-term trends among populations also varied: Campbellton,
Saint-Jean, and Western Arm Brook populations showed increases in \So
over time, Trinit\'{e}, Conne, and LaHave populations showed decreases,
while at Nashwaak there was an increase in median \So during the early
2000s but a decrease since 2010. Annual estimates of \So had little
uncertainty for one sea-winter dominated populations (Conne, Campbellton, and
WAB), with the exception of the 2004-2011 periord in the Campbellton where
there were very few scale samples, and hence high process error; but in general, 
uncertainty was higher (i.e. wider credible intervals) in the other
populations.

% % Dual plot of trends in S1
% \begin{figure}[htbp] \centering
%     \includegraphics[width=0.95\linewidth]{figures/s1-trends-dual.png}
%     \caption{Trends in survival in the first year at sea (\So). a) Posterior
%         estimates of \So for the seven populations examined, error bars indicate
%         the 90\% credible intervals, and dashed lines denote years of commercial
%         fishing moratoria for each province. b) Z-scores of median posterior
%         estimates.} \label{fig:s1-dual} 
% \end{figure}

% Single plot of trends in S1
\begin{figure}[htbp] \centering
    \includegraphics[width=0.95\linewidth]{figures/s1-trends-all.png}
    \caption{Median posterior estimates of survival in the first year at sea (\So) 
        for the seven populations examined.} \label{fig:s1-dual} 
\end{figure}


\begin{figure}[htbp] \centering
    \includegraphics[width=0.95\linewidth]{figures/s1-trends-faceted.png}
    \caption{Posterior estimates of survival probabilities in the first year at sea (\So). Dots indicate median estimates and error
        bars indicate the 90\% credible intervals. Dashes vertical lines reflect the year the commercial fishing moratoria were enacted
    in each province.} \label{fig:s1-faceted}
\end{figure}

Estimates of survival in the second year at sea (\St) were highly uncertain in all populations, and trends 
were not apparent in most populations given the large range of the credible
intervals in the yearly estimates (Fig.~\ref{fig:s2-faceted}). The estimates
of \St for the Saint-Jean and Trinit\'{e} were considerably less uncertain
than for the other populations, with the latter showing an increase with a peak in the mid-1990s followed 
by a decrease.

\begin{figure}[htbp] \centering
    \includegraphics[width=0.95\linewidth]{figures/s2-trends-faceted.png}
    \caption{Posterior estimates of survival in the second yyear at sea (\St).
 Dots indicate median estimates and error bars indicate the 90\% credible
 intervals. Dashes vertical lines reflect the year the commercial fishing
 moratoria were enacted in each province.} \label{fig:s2-faceted}
\end{figure}

Estimates of \Pg were mostly stable across time, except for the LaHave and
Nashwaak populations; the estimates of \Pg were slightly lower in the last
four years than in the previous ones, while in the Nashwaak the posterior
estimates of \Pg in 2012 were much lower than in all other years
(Fig.~\ref{fig:s2-faceted}). Variation in yearly estimates was highest in the
LaHave, Nashwaak, and Trinit\'{e}, and lowest in the 1SW-dominated
populations.

\begin{figure}[htbp] \centering
    \includegraphics[width=0.95\linewidth]{figures/pr-trends-faceted.png}
    \caption{Posterior estimates of proportion returning after one winter at sea (\Pg).
 Dots indicate median estimates and error
        bars indicate the 90\% credible intervals. Dashes vertical lines reflect the year the commercial fishing moratoria were enacted
    in each province.} \label{fig:pr-faceted} 
\end{figure}

Estimates of the population-level mean (\prmu) and standard deviation (\prsig) of proportion returning after one year at sea varied considerably among populations. 
For all
three 1SW-dominated populations (Campbellton, Conne, and WAB), estimates of \prmu
were very close to 1.0 and had little variation in \prsig (Fig~\ref{fig:prmu-post}; Table S7 in the Supplementary Materials).
Estimates of \prmu were the lowest for the two QC populations, particularly the Saint-Jean (median \prmu = 0.11).
Estimates of \prmu for the Nashwaak and the LaHave populations were close to 0.5, with these two populations having 
the highest estimated values of \prsig, particularly the Nashwaak (Fig~\ref{fig:prmu-post};  Table S8 in the Supplementary Materials).

% % latex table generated in R 3.6.3 by xtable 1.8-4 package
% Tue May 12 21:42:58 2020
\begin{table}[ht]
\centering
\caption{Posterior estimates of mean population-level
                     proportion returning after one winter at sea 
                    ($\mu_{Pg,r}$, logit-transformed) for the seven populations examined.} 
\label{tab:prmu}
\begin{tabular}{lrrrrr}
  \hline
Population & Median & 2.5\% & 25\% & 75\% & 97.5\% \\ 
  \hline
Campbellton River & 2.556 & 2.403 & 2.504 & 2.607 & 2.709 \\ 
  Western Arm Brook & 2.749 & 2.567 & 2.688 & 2.811 & 2.922 \\ 
  Conne River & 1.935 & 1.780 & 1.883 & 1.987 & 2.089 \\ 
  LaHave River & 0.163 & -0.055 & 0.092 & 0.237 & 0.379 \\ 
  Nashwaak River & 0.032 & -0.229 & -0.046 & 0.116 & 0.286 \\ 
  Trinité River & -0.377 & -0.486 & -0.415 & -0.340 & -0.261 \\ 
  Saint-Jean River & -1.212 & -1.291 & -1.239 & -1.183 & -1.124 \\ 
   \hline
\end{tabular}
\end{table}


% % latex table generated in R 3.6.3 by xtable 1.8-4 package
% Tue May 12 21:42:57 2020
\begin{table}[ht]
\centering
\caption{Posterior estimates of population-level
                    standard deviations of the proportion returning 
                    after one winter at sea 
                    ($\sigma_{Pg,r}$, logit-transformed) for the seven populations examined.} 
\label{tab:prsigma}
\begin{tabular}{lrrrrr}
  \hline
Population & Median & 2.5\% & 25\% & 75\% & 97.5\% \\ 
  \hline
Campbellton River & 0.329 & 0.234 & 0.292 & 0.374 & 0.487 \\ 
  Western Arm Brook & 0.573 & 0.461 & 0.529 & 0.621 & 0.732 \\ 
  Conne River & 0.371 & 0.265 & 0.332 & 0.416 & 0.525 \\ 
  LaHave River & 0.376 & 0.234 & 0.322 & 0.442 & 0.601 \\ 
  Nashwaak River & 0.491 & 0.339 & 0.433 & 0.563 & 0.750 \\ 
  Trinité River & 0.170 & 0.031 & 0.125 & 0.214 & 0.308 \\ 
  Saint-Jean River & 0.062 & 0.004 & 0.033 & 0.094 & 0.164 \\ 
   \hline
\end{tabular}
\end{table}



% \begin{figure}[htbp] \centering
%     \includegraphics[width=1.0\linewidth]{figures/pr-mu-posteriors.png}
%     \caption{Posterior estimates of the population-level parameters $Pr_{\mu}$
%        $logit(Pr_{\mu})$, and $logit(Pr_{\sigma})$. Dots denote median estimates, while the thick and thin error bars indicate
%        the 50\% and 90\% credible intervals, respectively.} 
%    \label{fig:prmu-post} 
% \end{figure}

\begin{figure}[htbp] \centering
    \includegraphics[width=1.0\linewidth]{figures/logitpr-mu-sigma-post.png}
    \caption{Posterior probability distributions resulting from the population-level estimates of the proportion returning after one winter at sea based on
        the parameters \prmu and \prsig. Gray lines denote population-specific median estimates of \prmu and \prsig, while
       the colored lines represent a sample of 20 posterior iteration draws for each population.} 
   \label{fig:prmu-post} 
\end{figure}


\subsubsection*{Correlations}

Roughly half (11) out of the 21 pair-wise correlations were significantt (p-value below 0.05)
(Fig.~\ref{fig:s1-corr}). When looking at the direction of the correlations,
these spanned both positive and negative coefficients. All seven populations had both
positive and negative correlations among them.
The corrected pair-wise correlations in $Z_1$ were almost identical to the
uncorrected ones as the error variance and covariance terms were very small.
The correlations between populations in the region were positive between Nashwaak and LaHave (Scotia-Fundy region), non-significant between the Trinit\'{e} and Saint-Jean (QC), 
non-significant between WAB and Campbellton but negative between Conne and WAB (NL).
The Campbellton had mostly non-significant pairwise correlations with other populations, even though there are both positive and negative correlation coefficients, except with the Saint-Jean
where the correlation between $Z_1$ trends is positive and significant.
Note that the uncertainty in these correlations reflect the uncertainty in the
measurements, but not in the process (in this respect they are equivalent to
population variance and not sample variance). 

\begin{figure}[htbp] \centering
    \includegraphics[width=0.95\linewidth]{figures/corrplot-post-z1-corrected.png} \caption{
        Correlations of estimated trends in $Z_1$ among populations. 
        Correlation coefficients are shown in each square, while colouring 
        denotes significance of the correlation ($p \leq 0.05$) as well as the direction and magnitude.}
\label{fig:s1-corr} 
\end{figure}

As is expected from Murphy's method, there are strong negative, yet highly uncertain, correlations
between survival in the second year at sea (i.e. $-Z_2$) and proportion returning
as 1SW ($log(P_g)$, Fig.~\ref{fig:cor-params}). Some populations (Campbellton,
LaHave, Nashwaak, and Saint-Jean) show a negative correlation between the
estimated number of smolts and survival in the first year at sea, while the
others fluctuate around zero. There is overall a positive correlation between
survival in the first year (i.e. $-Z_1$) and proportion returning as 1SW,
except for the Campbellton population, where this correlation is negative.


\begin{figure}[htbp] \centering
    \includegraphics[width=0.95\linewidth]{figures/corrected-cor-ALL.png}
    \caption{Correlations among estimated model parameters. \emph{log(smolts)} = estimated smolt abundances in log-space, $log(P_g)$ = logit-transformed probability of
        returning after one year at sea, $-Z_1$ = marine survival in the first year at sea, and $-Z_2$ = marine survival in the second year at sea.
        Note the correlations derived from negative instantaneous total mortalities $-Z_1$ and $-Z_2$ are representative to those of marine survival $S_1$ and $S_2$.}
    \label{fig:cor-params} 
\end{figure}

Overall, variation in return abundances was mostly explained by marine survival in the
first winter at sea, with $R^2$ estimates between 0.5 and 0.75 for all populations
with the exception of the Saint-Jean (Fig.~\ref{fig:mu-rsq}).
The low $R^2$ for the Saint-Jean is likely due to the negative correlation
between survival (i.e. $-Z_1$) and number of out-migrating smolts (see Fig.~\ref{fig:cor-params}). 
There are also moderately negative correlations between survival and smolts in
the Campbellton, LaHave, and Nashwaak populations.
These negative correlations could be indicative of density-dependent
mortality at sea, however this is generally thought to be unlikely.
Populations with a lower proportion of 1SW in their returns (e.g. Trinit\'{e}
and Saint-Jean) have a higher $R^2$ with estimated number of smolts than those
that are 1SW-dominated, but overall, smolts estimates had low $R^2$ values in 
all populations except the Saint-Jean (Fig.~\ref{fig:mu-rsq}).
The proportion returning as 1SW explained a moderate amount of the variance in
estimated 1SW returns in the Conne, LaHave, and Nashwaak, and much less in the
other four populations.
Values of $R^2$ were overall low for all parameters with regards to explaining
the variance of estimated 2SW returns. Estimates of mortality in the second winter at sea ($Z_2$)
explained little variance in estimates for 2SW returns, with the Saint-Jean again being the
exception.

\begin{figure}[htbp] \centering
    \includegraphics[width=0.95\linewidth]{figures/rsquared-mu-params.png}
    \caption{Estimated $R^2$ values for the correlation between estimated
        returns ($R_{1,r,t}$ and $R_{2,r,t+1}$) and the model parameters used
        for their estimation (i.e. estimated smolts, mortality in the first
        year at sea $Z_1$, mortality in the second year at sea $Z_2$, and
        proportion returning as 1SW, \Pg). Note we did not estimate $R^2$
        values between $R_{1,est}$ and $Z_2$ as the latter is not used to
        estimate the former, and that only the non-1SW dominated populations are
        shown in the $R^2$ values of the model parameters and $R_{2,est}$.}
    \label{fig:mu-rsq} 
\end{figure}


\section*{Discussion} 

% 1. Main findings
% 2. How they compare to other publications
% 3. Potential reasons
% 4. Caveats
% 5. Future directions

% 1. Main findings

Our results show that trends in marine survival among monitored Atlantic salmon
populations in eastern Canada are highly variable, both temporally and
spatially. Over the time periods for which data were available, some
populations show positive trends in survival in the first winter at sea (\So)
while other exhibits highly variable yet stationary trends, and some show declines.
Variation in survival at the population level can be large even within a
region. With one possible exception (LaHave River), consistent declines in
marine survival within populations are not evident since the onset of fishing
moratoria.

While correlations in marine survival trends among rivers are highly 
variable (Fig.~\ref{fig:s1-corr}),
there are some years in which the populations appear to behave in concert:
years 1997, 2007, and to some extent 2001 have consistently low survival across all populations 
(i.e. <5\%, Fig.~\ref{fig:s1-dual}) and little variation across populations.
Our results also show that variation of survival in the first year at sea is the most
important factor determining variation in number of returns, suggesting that changes in abundance are primarily driven by changes in marine survival. 
Nonetheless, variation in number of out-migrating smolts and, for some populations,
variation in probability of returning after one winter at sea, also explain
variation in returns, thus contributing to the observed variation in return abundances.
Furthermore, the negative correlation between marine survival and smolt
abundances in some populations is suggestive of density-dependent processes.
We could not assess trends for survival in the second winter at sea (\St) or
proportion returning as grilse (\Pg), as these parameter estimates were highly
uncertain and were strongly influenced by the priors. 

% Perhaps there is a need to rethink
% our understanding of Atlantic salmon population dynamics in light of the
% possibilities that (1) any real and persistent declines in marine survival
% have been experienced by some but not necessarily all populations, (2)
% reductions in survival might have occurred over a relatively brief period of
% time and have not persisted, (3) marine survival has been relatively stable,
% or increasing in some populations for one or more decades, (4) variation in
% survival at the population level can be large even within a region.

% 2. How they compare to other publications
% Population vs regional trends 
At first glance, the large variation in marine survival trends among the seven
populations examined seems to be at odds with the synchronous
trends of declining marine survival detected at broad regional scales
\citep{Olmos2019}. However, these two observations are not necessarily mutually
exclusive: our study represents a subsample of the populations in the region,
and, with one exception, encompasses a shorter period of time; highly variable 
trends in marine survival among populations can still add
up to overall negative trends at a regional scale.
Aside from the overarching difference in the spatial scale of data sources
(i.e. river vs province scales), the divergence in local versus regional
trends could be further confounded by differences in model structure (e.g.
methods used for estimating covariance), use of stock-recruitment
relationships \citep{Olmos2019} rather than empirical smolt count estimates 
(present study) to estimate marine survival, and separation between fishing mortality 
and natural mortality, which was not done in our study.
% As trends in marine survival during the first winter at sea are highly
% independent among rivers on relatively small spatial scale, 
Our results suggest that trends from broader geographical areas (i.e.
province, state, or country-wide estimates) might not be representative of
individual populations, and concomitantly, trends of individual populations
might also not be representative of region-wide trends.

% Trends in survival at the population level
Trends in marine survival
among populations were compared by \citet{Chaput2012a} using adult return rates.
He found that for 4 of 6 populations examined, return rates in the 1990s 
were lower than those during the 1970s.
\citet{Gibson2016} calculated higher return rates of 2SW for Nashwaak River salmon in
the 1970s than in the period since, with return rates of 1SW in the 1970s
being comparable to those in the late 2000s.
\citet{Friedland1993} compared return rates for a number of populations in eastern
North America between 1973 and 1988, and suggested there are similar trends among these. 
However, the similarity in these trends was driven primarily by two years, 1977 and 1978, which
show concurrent low and high relative return rates across populations,
respectively. Other years are much more variable relative to each other.
The time series in \citet{Friedland1993} end in  
1988; thus there are only a few years that overlap the
time series in our study.
While declines in return rates since the 1970s seem to be consistent among
populations, we were unable to assess if marine survival estimates were also
higher in the 1970s because smolt count data from this decade are not available.
\citet{Dempson2003} described a general declining trend in marine survival for
Newfoundland populations (except WAB). We drew the same conclusion for Conne
River, but not Campbellton River or WAB. It is not possible to draw broader
conclusions with data from only three Newfoundland populations, but it seems
that among index rivers, those in northern Newfoundland are among those with
the highest marine survival rates.

% Return rates vs marine survival estimates
In any event, we caution that the pooling of adult return rates
\citep{Chaput2012a, Friedland1993,Gibson2016} can mask inter-annual
variation in marine survival \citep{Hubley2011}, and hence might not produce
an accurate depiction of marine survival trends.
While river-specific return rate estimates are available for the populations
used in this study \citep{ICES2019}, these estimates are only an approximation
of marine survival with varying degrees of similarity depending on each
population's life history. 

The estimates of return rates of one sea-winter salmon
approximate marine survival in the first year at sea for one
sea-winter-dominated populations, particularly for years in which 
directed marine fisheries are largely non-existent,
and we see this by comparing $S_1$ in the three NL populations
with the return rates presented in \citet{ICES2019}. However, as the proportion of fish that
return as one sea-winter decreases, marine survival in the first year at sea
can increasingly diverge from return rates. Thus, trends in return rate
would be particularly misleading if the proportion returning as one sea-winter
varies in time;
proportion returning as one sea-winter was an important factor driving number of 
1SW returns in three of the populations examined (Fig.~\ref{fig:mu-rsq}).
Furthermore, return rates of two sea-winter fish also include survival in the
first winter at sea and the proportion returning as two sea-winter, thus
providing a very coarse estimate of marine survival in the second year at sea. 
Given that changes in return rates of two sea-winter fish are a result of a
combination of changes in survival in the first year at sea in the year prior,
survival in the second year at sea, and in the proportion returning as two
sea-winter, an increase in any of those parameters while the others remain
the same will result in an increase in return rates. Our model attempts to
improve some of the shortcomings of using return rates as a proxy of marine
survival by directly estimating marine survival in the first and second years
at sea, as well as the proportion returning as 1SW.

Interestingly, our estimates of survival in both first and second years at sea
for the Trinit\'{e} population are very similar to those produced by
\citet{Chaput2003b}, who applied a two-sex model, and our trends are almost identical
for the overlapping time period that marine survival was estimated for in
their study (1984-1998). While \citet{Chaput2003b} separated abundance data
for males and females based on sex ratio information and assumed their
survival rates were the same (to be able to reach an analytical solution), our
study reached almost the same results (albeit with slightly higher
uncertainty), using a Bayesian approach with informative priors. These
overlapping trends obtained with two different methods suggest that our method
is indeed effective at estimating marine survival.

% 3. Potential drivers of patterns 
There are a variety of potential explanations for the highly variable trends
in estimates marine survival in the first winter at sea.
Environmental drivers of survival (e.g., temperature, predation, prey abundance, 
interactions with farmed salmon) could be highly localized 
relative to broader-scale, even ocean-wide, drivers.
A recent study showed that correlations between oceanic temperature, primary
productivity, and post-smolt survival in the North Atlantic
are better explained by broad-scale environmental trends than regional ones
\citep{Olmos2020}.
However, these local drivers are potentially much more difficult to quantify than 
broader ones.
% The synchrony reported for marine survival trends at broader spatial 
% scales \citep{Massiot-Granier2014,Olmos2019} might be attributable to the 
% use of stock-recruitment relationships to estimate survival,
% relationships that may have been confounded by changes in recruitment dynamics.
There is some evidence of a correlation between return rate and growth (as
indicated by inter-circuli spacing on scales), where years of poor growth
tended to also be years of poor survival \citep{Friedland1993}, supporting the
idea that marine survival is mediated by environmentally-driven changes in growth.
Furthermore, among European salmon, there is evidence of a positive correlation
between spring temperature in the Norwegian and North Seas and population abundance, 
suggesting warmer
conditions favor post-smolts \citep{Friedland1998}, based on mapping the
extent of area of suitable temperature (7-13 \textdegree C).

Nonetheless, the causal mechanisms for why warming should affect post-smolt
survival almost certainly differs depending on the difference between
temperature experienced by the post-smolts and their respective
population-specific thermal optimum. 
This difference could explain why populations in eastern North America are
declining in the southern part of their range but potentially increasing, or
remaining stable,
further north, and also why some studies find positive correlations between
temperature and abundance \citep{Friedland1998, Friedland1998b, Jonsson2004}
while others find negative associations \citep{Friedland1993, Todd2008}.
\citet{Olmos2020} documented a positive relationship between temperature and marine survival 
in northern regions and a negative one in southern regions, providing 
evidence of differing mechanisms across latitudes.
Putative associations between temperature and direct estimates of marine
survival warrants further study at the population level.

Oceanic conditions have been correlated with abundance trends, growth, and
marine survival in Atlantic salmon, which are thought to be mediated by
bottom-up effects driving ocean productivity and food availability
\citep{Todd2008, Renkawitz2015,Olmos2020}. 
However, the mechanism by which such bottom-up effects, mediated by changes in
food availability, affect population dynamics beyond marine survival needs to
be considered further. 
There are potential carry-on effects of oceanic conditions that manifest with
regards to freshwater production for individuals that survive the marine phase
of their life cycle. 
For example, adults that return to their natal streams after spending
suboptimal conditions at sea might be less likely to make it to the spawning
grounds or secure a mate, and might also produce fewer eggs or eggs with a
lower energetic content than those produced by adults which grew in optimal
oceanic conditions.
As larger females tend to be more productive, in terms of fecundity and total
reproductive energy, than the same weight's worth of smaller females
\citep{Barneche2018}, a small decrease in body condition resulting from bottom-up
impacts on food availability and quality could potentially have
disproportionate effects on fecundity and fitness of the offspring.

% Furthermore, freshwater carryover effects acting on smolts, such as pH effects
% on prey-avoidance behaviour, could potentially mediate post-smolt survival
% \citep{Halfyard2012}.

Egg-to-smolt survival in Atlantic salmon is also highly variable, and perhaps more so
than marine survival \citep{Klemetsen2003,Chaput2015}.
While this variation is attributable to changes in freshwater
conditions (e.g. discharge, temperature, water quality) and uncertainty in
spawner and smolt estimates, 
changes in the oceanic conditions that spawners experience could also be
contributing to this variation through, for example, decreases in body condition, fecundity and hence fitness.
However, while variation in number of smolts was 
important for number of returns, it was considerably less important than
survival during the first year at sea, suggesting that return abundances 
are mostly influenced by the marine phase of their life cycle.
While there would be a generational lag in how such effects
might be manifest at the subsequent adult stage,
given that most correlations are between relatively monotonic declines
in abundance coupled also monotonic increases in climatic indices
over decadal time scales \citep[e.g.,][]{Friedland1998, Todd2008,
    Beaugrand2012}, it would be expected that these correlations would be
maintained even if salmon abundances were lagged by a generation length.

There is little evidence that marine survival is density-dependent in
Atlantic salmon \citep{Jonsson1998,Gibson2006}, but these 
density-dependent processes could occurr during parts of the marine phase, 
particularly for populations that are 
near historically low levels of abundance. One potential mechanism might be a
``predator pit'': when prey populations are very small, predator-induced
mortality can also be low because the prey are simply not abundant enough to
be generally consumed by an optimal forager (for example, search costs may be
too high relative to the fitness benefits of consuming the prey). But as prey
abundance increases from very low levels, predator-induced mortality might
also increase as preference for the prey increases. 
Correlations between return estimates and survival were negative for 
some populations (Fig.~\ref{fig:cor-params}),
indicating that there may be some compensatory density dependent effects during the 
marine phase.
Exploring relationships between survival and population size could potentially
shed light about the processes that have caused many of the population
declines that have been documented.


% 4. Caveats
As with all novel modelling approaches, there are caveats to acknowledge.
The seven populations examined in the present study represent a small subset of the total number of salmon rivers in eastern Canada and hence might not be representative
of the overall regional trends in marine survival. However, there are no other
long-term time series of smolts and adult returns from which to draw inferences from.
While there are analytical issues associated with the estimation of \So, \St, and \Pg,
the assumption that survival in the second year at sea is multiplicative to survival in the first 
year at sea could produce unrealistic results.
We know there is a period of a few months where 1SW 
returns are subject to a different environment than those salmon that will
return as 2SW the next year. 
While this is not ideal,
overcoming this assumption would require an additional parameter to be
estimated, or an additional assumption as to what proportion of \So is not
multiplicative to \St (as the returning 1SW adults do not experience the same
environment when they return to their natal streams as those fish who stayed
at sea for an additional winter before returning to spawn).

Secondly, our error estimates of 1SW and 2SW returns might not approximate the
underlying uncertainty. Nonetheless, our approach likely overestimates
uncertainty as we did not model error in annual 1SW and 2SW returns
hierarchically within each population (see Supplementary Materials for
details), which should produce conservative results. 

% Secondly, the assumed hierarchical structure might not be the most appropriate
% for modelling smolt abundances. This approach results in shrinkage to the
% mean, which means that the variation in annual smolt estimates is less than
% it would be if not modelled hierarchically. 
% That said, this assumption is likely to result in more conservative trends in
% marine survival, as the variation is smolt estimates is reduced.
% This shrinkage of annual smolt abundances is also reducing the $R^2$ values 
% between smolts and estimates returns, albeit only marginally.
% It is important to note that the hierarchical structure in the estimation of
% \Pg seems like a reasonable assumption given that the probability of returning
% as grilse has a genetic component associated to it \citep{Barson2015} and its inter-annual
% variation in a given population is likely partly constrained by this.

Thirdly, the accuracy and precision of the model's output is different
for different populations. Survival in the first year at sea is the variable
that explains most of the variance in estimates of 1SW returns, except for the
Saint-Jean; it seems 
the model does not provide reliable parameter estimates for this population,
which is the population with the highest proportion of 2SW returns.
Furthermore, estimates of \St and \Pg in all populations are negatively correlated and highly
uncertain, and this must be considered when making any inferences based on
these parameters for any of the populations examined.

Lastly, our model does not differentiate mortality resulting from natural
causes (e.g. predation) or from anthropogenic factors such as fishing or
interactions with aquaculture. Fishing mortality has decreased
significantly due to the sequential moratoria enacted across eastern Canada, and
aquaculture, which has grown over recent decades, can negatively impact salmon survival through sea lice
\citep{Shephard2020,Bohn2020} or genetic introgression \citep{Glover2017,Vollset2020}. 
Thus, estimated trends in marine survival are confounded because we know marine
survival will be reduced in years where commercial fishing occurred (i.e.
pre-2000s), while the trends in aquaculture-related mortality are unknown.
In other words, any increasing trends in marine survival are confounded by the additional 
fishing mortality before the turn of the century.

% 5. Future directions
Perhaps a reframing of the issue of marine survival is key to furthering our
understanding of Atlantic salmon population dynamics. Marine survival has not 
declined consistently, and over the same time periods, across all
populations, even if the overall trend has been one of decline. 
But the fact that, for most of the populations examined, 
present-day survival fluctuates around levels similar to those that have occurred in the past, 
\emph{despite} reduced commercial fishing mortalities, suggests
that there may well be an interaction between small population size (small
relative to unfished population size or carrying capacity), recovery
potential, and environmental stochasticity that has not been fully explored in
Atlantic salmon. 
All else being equal, relatively small populations are more vulnerable to
demographic, environmental, and genetic stochasticity than large populations
\citep{Lande1993, Hutchings2015}. Interactions between population size and the
demographic consequences of environmental stochasticity appear to have
affected recovery in many marine fishes that have exhibited impaired recovery
since mitigation of the threat posed by fishing mortality
\citep{Hutchings2017, Hutchings2020}. The possibility that similar
interactions may be impairing the recovery of wild Atlantic salmon merits
study.

\section*{Acknowledgements}

We would like to thank Martha Robertson, G\'{e}rald Chaput, and Carmen David
their useful comments on the manuscript, and Sean Anderson for his help with
implementing the non-centered parameterization of the model. Hydro-Québec
contributed to data acquisition from Rivière de la Trinité. This research was
supported by the Atlantic Salmon Conservation Foundation and the Atlantic
Salmon Research Joint Venture.

\section*{Conflicts of Interest}

The authors declare no conflicts of interest.
 
\bibliography{subset}

%\documentclass[12pt]{article}
\usepackage[top=0.85in,left=1.0in,right=1.0in,footskip=0.75in]{geometry}
%\usepackage[parfill]{parskip}
\usepackage{setspace}
\usepackage{lineno}
\usepackage[hidelinks]{hyperref}
%\onehalfspacing
\doublespacing

\usepackage[round,sectionbib]{natbib}
\setcitestyle{authoryear}
\bibpunct{(}{)}{;}{a}{}{;}
\bibliographystyle{fishfishnourl}

%\usepackage{textcomp}
%\usepackage{libertine}
%\usepackage{inconsolata} % sans serif typewriter

\usepackage{mathtools} % for dcases
\usepackage{xcolor} % for textcolor
\usepackage{makecell} % for \makecell and within cell line breaks (\thead)

\usepackage[T1]{fontenc}

%\makeatletter\let\expandableinput\@@input\makeatother % expandable input for \input inside tables

% Linux Libertine:
\usepackage{textcomp}
\usepackage[sb]{libertine}
\usepackage[varqu,varl]{inconsolata}% sans serif typewriter
\usepackage[libertine,bigdelims,vvarbb]{newtxmath} % bb from STIX
\usepackage[vvarbb]{newtxmath} % bb from STIX, removed bigdelims for ScholarOne rendering
\usepackage[cal=boondoxo]{mathalfa} % mathcal
%\useosf % osf for text, not math
%\usepackage[supstfm=libertinesups,%
%  supscaled=1.2,%
%  raised=-.13em]{superiors}

\usepackage{xspace}
\usepackage{xfrac} % for diagonal inline fractions in text
%\usepackage{array} % for making whole row bold in table
\usepackage{colortbl} % for background colours in table rows
\usepackage{longtable}
\usepackage{amssymb} % for \checkmark 
\usepackage{amsmath} % for \checkmark 
\usepackage{rotating}
\usepackage[nolists,tablesfirst]{endfloat} % for putting figs and tables at end of document
\DeclareDelayedFloatFlavor{sidewaystable}{table}
\usepackage{makecell} % for \makecell in tables
\usepackage{doi}

%\usepackage{xr} % to obtain label references from supp materials file
%\externaldocument[S-]{suppmat}

\usepackage{tabularx}
\usepackage{booktabs}
\usepackage{array} % for table wrapping of columns
\newcolumntype{L}[1]{>{\raggedright\let\newline\\\arraybackslash\hspace{0pt}}m{#1}}
\newcolumntype{C}[1]{>{\centering\let\newline\\\arraybackslash\hspace{0pt}}m{#1}}
\newcolumntype{R}[1]{>{\raggedleft\let\newline\\\arraybackslash\hspace{0pt}}m{#1}}
%\usepackage[detect-all]{siunitx} % for SI units

% custom hyphenation:
\hyphenation{inverse}
\hyphenation{At-lantic}
\hyphenation{elasmo-branchs}


% Macros
\newcommand{\So}{$S_{1}$\xspace}
\newcommand{\St}{$S_{2}$\xspace}
\newcommand{\Pg}{$P_r$\xspace}
\newcommand{\prmu}{$\mu_r$\xspace}
\newcommand{\prsig}{$\sigma_r$\xspace}
\newcommand{\Linf}{$L_{\infty}$}
\newcommand{\DWinf}{$DW_{\infty}$}
\newcommand{\alphat}{$\tilde{\alpha}$}
\newcommand{\lamat}{$l_{\alpha_{mat}}$}
\newcommand{\lamatb}{$l_{\alpha_{mat}}b$}
\newcommand{\rmax}{$r_{max}$\xspace}
\newcommand{\ageratio}{$\alpha_{mat}/\alpha_{max}$}
\newcommand{\yr}{year\textsuperscript{-1}}
\newcommand{\rsq}{$R^2$\xspace}
\newcommand{\mytilde}{\raise.17ex\hbox{$\scriptstyle\mathtt{\sim}$}}
% Select what to do with command \comment:  
%\newcommand{\comment}[1]{}  % comment not shown
\newcommand{\comment}[1]{\par {\bfseries \color{blue} #1 \par}} % comment shown
%% END MACROS SECTION

\begin{document}
\linenumbers


\section*{Trends in marine survival of Atlantic salmon in eastern Canada}

\textbf{Sebasti\'{a}n A. Pardo\textsuperscript{1*}, 
        Geir H. Bolstad\textsuperscript{2}, 
        J. Brian Dempson\textsuperscript{3}, 
        Julien April\textsuperscript{4}, 
        Ross A. Jones\textsuperscript{5}, 
        Martha J. Robertson\textsuperscript{3}, 
        Dustin Raab\textsuperscript{6}, 
Jeffrey A. Hutchings\textsuperscript{1}} 

\noindent\small{\textsuperscript{1} Department of Biology, Dalhousie University, Halifax, NS, Canada\\}
\small{\textsuperscript{2} Norwegian Institute for Nature Research (NINA), Trondheim, Norway\\}
\small{\textsuperscript{3} Fisheries and Oceans Canada, St. John's, NL, Canada\\}
\small{\textsuperscript{4} Minist\`{e}re des For\^{e}ts, de la Faune et des Parcs, Qu\'{e}bec, QC, Canada\\}
\small{\textsuperscript{5} Fisheries and Oceans Canada, Moncton, NB, Canada\\}
\small{\textsuperscript{6} Fisheries and Oceans Canada, Dartmouth, NS, Canada\\}
\small{\textsuperscript{*} Corresponding author: spardo@dal.ca}

\section*{Abstract}

Declines in wild Atlantic salmon (\emph{Salmo salar}) throughout the north
Atlantic are primarily attributed to declining survival at sea. This
hypothesis has proven challenging to test on a river-by-river basis because of
the need to model data on both migrating smolts and returning adults and to
simultaneously estimate multiple parameters, especially for salmon spending
more than one winter at sea (1SW) before spawning. We fit a hierarchical Bayesian
maturity schedule model to data (19-42 years) for seven populations in  
Newfoundland and Labrador (NL), Qu\'{e}bec (QC), New Brunswick (NB), and Nova
Scotia (NS), Canada. We estimate survival in the first (\So) and second year at sea (\St),
and the proportion returning as 1SW adults (\Pg). Trends in  were not consistent
among rivers. Since 1990, \So increased at Western Arm Brook (NL)
and Rivi\`{e}re Saint-Jean (QC), but declined at Conne River (NL) and Rivière
de la Trinité (QC). Since the mid-1990s, \So increased at Campbellton River (NL),
declined at LaHave River (NS), and fluctuated at Nashwaak
River (NB). Estimates of \St were highly uncertain, particularly for
1SW-dominated populations; \Pg was generally stable. These results challenge the
narrative that marine survival has changed in a temporally consistent manner
among spatially disparate populations. Our findings suggest that, at the
population level, changes in abundance are attributable to temporal shifts in
multiple components of individual fitness, including at-sea survival. If
salmon populations do not respond in a consistent, uniform manner to changing
environmental conditions throughout their range, future research initiatives
should explore why.

% The marine phase of anadromous Atlantic salmon (\emph{Salmo salar}) is the
% least known yet one of the most crucial with regards to population
% persistence. Declines in many Atlantic salmon populations in eastern
% Canada have often been attributed to changes in conditions at sea, negatively
% affecting their survival. However, marine survival estimates are difficult to
% obtain given that many individuals spend multiple winters in the ocean before
% returning to freshwater to spawn, necessitating the estimation of multiple
% parameters. To do so, we fit a hierarchical Bayesian maturity schedule model
% to smolt and adult abundance time series for seven populations located in
% Newfoundland and Labrador (NL), Qu\'{e}bec (QC), New Brunswick (NB), and Nova
% Scotia (NS). The datasets ranged between 19 and 42 years in length. We
% estimated three components of marine survival: survival in the first year at
% sea (\So), survival in the second year at sea (\St), and proportion returning
% as one sea-winter adults (\Pg). Controlling for time frame, trends in
% estimates of \So were not consistent among rivers.
% In the four populations for which data extended to 1990, marine survival
% during the first year at sea predominantly increased over time in Western Arm
% Brook (NL) and Rivi\`{e}re Saint-Jean (QC) but largely declined in Conne River
% (NL) and Rivi\`{e}re Trinit\'{e} (QC). In the three other populations, \So
% exhibited an increase in Campbellton River (NL), a decline in LaHave River
% (NS), and fluctuating stability in Nashwaak River (NB) since approximately the
% mid-1990s. 
% Estimates of \St were highly uncertain, particularly for 1SW-dominated rivers
% (Conne, Campbellton and WAB, where the abundance of 2SW returns is very low)
% and thus the posterior distributions matched our choice of prior and trends
% could not be assessed. 
% \Pg was temporally stable within rivers, except for LaHave and Nashwaak Rivers. 
% Our findings challenge the assumption that marine survival has changed in a consistent
% manner among populations across a broad geographic scale, and suggest that
% trends can be river-specific. 
% The well established correlations between climate variables and abundance are
% not being mediated solely by marine survival, and there can potentially be
% some important indirect effects on fecundity.
% Further work exploring potential correlates of marine survival and the
% potential non-lethal effects of climate effects is warranted.


Keywords: salmonid, survival at sea, natural mortality, marine mortality

%Running Head: 

\section*{Introduction} % (4-5 paragraphs)

%1. Declining salmon numbers

Reductions in fishing mortality, albeit necessary, are not always sufficient
to facilitate population recovery. Experience with numerous commercially
exploited marine fisheries since the early 1990s has shown that not all
populations respond as favourably as anticipated to major reductions in
exploitation \citep{Hutchings2017}. Gradual efforts to close commercial
Atlantic salmon (\emph{Salmo salar}) fisheries in eastern Canada culminated in
full moratoria in all regions, beginning in the Maritime provinces (1984) and
following in Newfoundland (1992), Labrador (1998), and Qu\'{e}bec (2000). Since
these closures, many populations have not increased as 
expected \citep{Dempson2004, ICES2019}; some 
have been assessed as considered threatened or endangered by the 
Committee on the Status of Endangered Wildlife in Canada \citep[][]{Cosewic2010}, 
Canada's national science advisory body (to the national government) on
species risk of extinction.
While it is not fully understood what is driving population declines, the potential
drivers of these are many \citep[see ][for a detailed discussion of possible
causes]{Cairns2001}, including but not limited to: fishing mortality \citep{Dempson2004}, 
damming of waterways and changes in the freshwater habitat \citep{Dunfield1985}, acidification
\citep[particularly in the Southern Uplands region of
NS, see][]{Gibson2010}, predation by seals and birds \citep{Cairns2000}, negative
effects of interbreeding or interactions with escaped farmed salmon
\citep{Keyser2018}, and climate-driven changes in survival and productivity \citep{Mills2013}.

% 2.
Over the past three decades, a narrative has emerged that marine survival of
Atlantic salmon has declined throughout the North Atlantic \citep{ICES2019}.
%\citep{Hansen1998,OMaoileidigh2003,Chaput2012a}.
Based on multiple lines of evidence that climate conditions can directly and
indirectly influence the abundance and productivity of Atlantic salmon
populations \citep{Mills2013,Almodovar2019}, it has been presumed that oceanic climate effects are
driving population dynamics primarily through changes in marine survival.
An implicit assumption is that any trend in
survival in ocean habitat that is shared by multiple populations during their
seaward migration period will be experienced similarly. 
Put another way, given that salmon from different rivers 
are hypothesized to share marine habitat during some of their time at sea, it
has been presumed that populations share similar temporal trends in
at-sea mortality \citep{Friedland1993, Friedland1998, Russell2012}.  
A recent study by \citet{Olmos2019} suggested that trends in post-smolt
survival, when estimated at the stock level, are synchronously declining
for all Atlantic salmon in eastern North America.

% This perception is widespread both in the scientific
% literature as well as federal reports. The latest status report on Atlantic
% salmon by COSEWIC states that ``While the mechanism(s) of marine mortality is
% uncertain, what is clear is that the recent period of poor sea survival is
% occurring in parallel with many widespread changes in the North Atlantic
% ecosystem.'' \citep{Cosewic2010}. Consequently, there have several lines of
% inquiry as to what the causes behind these declines might be attributable to
% \citep{Friedland1993, Friedland1998}.


In contrast to the narrative of widespread, demographically similar increases
in at-sea mortality, the conservation status of Canadian salmon populations differs
considerably. Populations in the southern part of their range are more
likely to be assessed as being of conservation concern than those in more
northerly regions \citep{Cosewic2010}. This geographical disparity in status
suggests that if marine survival has been, or is, a key factor responsible for
most population declines, these changes are not uniformly distributed across
all populations. 

% Nonetheless, most of the populations declines assessed by COSEWIC have
% occurred in populations in the southern extent of its distribution:
% populations in the Bay of Fundy, Anticosti Island, and the Atlantic coast of
% Nova Scotia being assessed as Endangered, populations in the south coast of
% Newfoundland assessed as Threatened, And the populations in the New Brunswick
% and Qu\`{e}bec coasts of the Gulf of St. Lawrence assessed as Special Concern
% \citep{Cosewic2010}. On the other hand, northermost populations have shown
% stable, or even increasing population trends, suggesting that, if marine
% survival were to be a factor in many of these declines, these changes are not
% uniformly distributed across all populations.

% While the purported decline of marine survival is mentioned widely in the 
% scientific literature, only relatively few studies have quantified it in detail.
% The basis for this premise is a limited number of time series that exhibit
% temporal declines in a proxy of marine survival (i.e. return rates or
% post-smolt survival).

% 3. Nonetheless, this premise has never been examined in detail
Given the logistical challenges associated with estimating at-sea survival, it
is not surprising that the number of studies that have estimated temporal
trends has been limited. An additional limitation has been the derivation of
proxies (e.g., return rates), rather than direct model-based estimates, of
marine survival.
\citet{Chaput2012a}, for example, examined the return rate of smolts to adult salmon 
as a metric of marine survival, finding that most Canadian populations 
had experienced declining return rates. 
However, examination of trends in return rates alone
can mask changes in differential survival during different years at sea, as well
as changes in the proportion of adults returning after one or two years at sea.
Recently \citet{Olmos2019} suggested that trends in post-smolt
survival, when estimated at the stock unit level, are synchronously declining
for all Atlantic salmon in Eastern North America and Canada, a conclusion ultimately 
grounded on the veracity of highly variable stock-recruitment relationships.

%4. 
In the present study, we compile data on the number of migrating smolts and number of returning adults 
for seven wild Canadian populations of Atlantic salmon to model trends in marine survival.
While some studies have previously used maturity-schedule models to estimate marine
survival for a limited number of salmon populations \citep{Chaput2003b}, none
have incorporated data extending over multiple decades, nor have they examined
trends among more than two or three populations. 
Here, we develop a hierarchical Bayesian model that uses Murphy's maturity
schedule method, in conjunction with informative priors, to estimate yearly
marine survival in salmon. In addition to accounting for observation error in
smolt and return estimates, we estimate the proportion of salmon returning
after one winter hierarchically.

\section*{Methods}

\subsection*{Data}

We obtained time series data of outmigrating smolt and returning adult
abundances for seven Atlantic salmon populations in eastern Canada, encompassing a
wide range of the species' distribution (Fig.~\ref{fig:map}). 
Populations included the LaHave River in the Southern Uplands region of Nova
Scotia (NS), Nashwaak River, New Brunswick (NB), Rivi\`{e}re de la Trinit\'{e} (Trinit\'{e}) and
Saint-Jean rivers in Qu\'{e}bec (QC), and  Western Arm Brook (also referred to as  WAB), Campbellton, and
Conne River, Newfoundland (NL). 
Data were collected in NS, NB, and Newfoundland
and Labrador (NL) by Fisheries and Oceans Canada (DFO) and in QC
by the Minist\`{e}re des For\^{e}ts, de la Faune et des Parcs, Qu\'{e}bec.

\begin{figure}[htbp] \centering
    \includegraphics[width=0.85\linewidth]{figures/rivers-map2.png}
    \caption{Locations of the seven rivers in eastern Canada with time series abundance data of outmigrating smolts and 
    returning adults.} \label{fig:map} 
\end{figure}

\subsubsection*{Smolt and adult return abundance data}

Smolt and adult return abundance estimates originate from a variety of
sources. Smolt estimates from the Trinit\'{e}, Saint-Jean, and Conne populations were obtained using a
mark-recapture approach, while estimates from the ..., WAB, and Campbellton populations
were obtained by direct counts using fish counting fences.
For further details on the data collection methodologies refer to 
\citet{Dempson1991, Venoitt2018} for NL populations, 
\citet{April2018}  for QC populations,
\citet{Jones2014} for NB populations,
and \citet{Gibson2009} for NS populations. 

Yearly return data are often recorded in terms of two size groups: small ($< 63$ cm
FL) and large ($\geq 63$ cm FL) salmon, as these closely represent different
life-history strategies (i.e. 1SW and 2SW), but can be confounded with repeat
spawners of different sizes. To correct for this in returns 
reported as small and large salmon, we estimated the abundance
of 1SW and 2SW returns using yearly scale age data of a subsample of returns:

\begin{equation}
    p_{r,t,a} = \frac{\sum_{s}{(\frac{n_{r,t,s,a}}{n_{r,t,s}} * N_{r,y,s})}}{\sum_{s}{N_{r,t,s}}}
\end{equation}

where $p_{r,t,a}$ is the proportion of annual returns in river $r$, year $t$,
and of spawning history $a$ (either 1SW or 2SW returns); $n_{r,t,s,a}$ is the
number of samples in river $r$, year $t$, of spawning history $a$, and of size
group $s$; $n_{r,t,s}$ is the total number of samples in river $r$, year $t$,
and of size group $s$; and $N_{r,t,s}$ is the returns of salmon
in river $r$, year $t$, and of size group $s$.

\subsection*{Bayesian model}

We developed a hierarchical Bayesian model that uses Murphy's maturity
schedule method, in conjunction with informative priors, to estimate yearly
marine survival in seven populations of Atlantic salmon. We account for
observation error in smolt and return estimates, as well as estimating the
proportion returning after one winter (i.e. \Pg) hierarchically.
There is an identifiability problem in the maturity schedule equations where
the parameter estimates cannot be optimally solved \citep{Chaput2003a}.
However, this issue can be mathematically overcome, at least partially, by
using informative priors for all three marine survival parameters in a
Bayesian framework.
This method requires abundance estimates of smolts as well as abundance estimates
of returning one-sea-winter (1SW) and two-sea-winter adults. With these data,
it estimates three parameters: survival in the first year at sea (\So), survival
in the second year at sea (\St), and the proportion of fish returning after one
year at sea (\Pg). 

Our model does not include repeat spawners and assumes that no fish spend
three or more winters at sea before returning to spawn for the first time.
The model also assumes that mortality in the second winter at sea (\St)
is additional to mortality in the first winter at sea in the previous year, 
and therefore does not account for differences in environmental conditions experienced
between 1SW and 2SW fish of the same smolt cohort during their overlapping first year at sea.
In other words, our model assumes that the decision of returning occurs just before
actually being counted as returns and that \St is any additional mortality in
the subsequent year. 

%\comment{Other assumptions...}

Observed smolt estimates were modelled hierarchically and included
observation error:

\begin{equation}
log(smolts_{obs,t,r}) = log(smolts_{true,t,r}) + \epsilon_{t,r}
\end{equation}

where $smolts_{true,t,r}$ are the true smolt abundances for year $t$ and river
$r$, and $\epsilon_{t,r}$ is the error term, which is empirically derived from
by calculating the yearly coefficient of variation in the empirically derived
smolt estimates (see Table S1 in the Supplementary material). 
Where available, we used population-specific measurement error estimates for smolt abundances; if not 
available, we set measurement error at 5\%. 
The log-transformed true smolt abundances are
normally distributed around a population-level mean and standard deviation:

\begin{equation}
log(smolts_{true,t,r}) \sim Normal(\mu_{smolts,r}, \sigma_{smolts,r})
\end{equation}

where $\mu_{smolts,r}$ and $\sigma_{smolts,r}$ are parameters estimated by the
model for each river.

Once we have yearly estimates of smolt, 1SW, and 2SW abundances, we estimate
marine survival parameters using Murphy's maturity schedule method
\citep{Murphy1952, Ricker1975}:

\begin{align}
    R_{1,t} &= smolts_{true,t-1} * S_{1,t} * Pr_t \label{eq:1}, \\
    R_{2,t+1} &= smolts_{true,t-1} * S_{1,t} * (1 - Pr_t) * S_{2,t+1} \label{eq:2}
\end{align}

where $R_{1,t}$ and $R_{2,t+1}$ are the estimated abundances of 1SW and 2SW
salmon returning in years $t$ and $t+1$, respectively, $smolts_{true,t-1}$ is the
estimated number of outmigrating smolts in year $t-1$, $S_{1,t}$ is the proportion of
salmon surviving in their first year ($t$) at sea, $Pr_t$ is the proportion of
salmon that return to spawn at year $t$, $S_{2,t+1}$ is the survival in their
second year at sea of the same cohort of salmon who did not return to spawn at
year $t$.

However, to allow for normally and log-normally-distributed parameters we
log-transform equations~\ref{eq:1} and~\ref{eq:2} to obtain

\begin{align}
    log(R_{1,r,t}) &= log(smolts_{r,t-1}) + log(Pr_{r,t}) - Z_{1,r,t} \label{eq:3}, \\
    log(R_{2,r,t+1}) &= log(smolts_{r,t-1}) - Z_{1,r,t} + log(1 - Pr_{r,t})  - Z_{2,r,t+1} \label{eq:4} 
\end{align}

where $Z_{1,r,t}$ and $Z_{2,r,t+1}$ are the instantaneous mortality rates.
Process error was included as the standard deviation of the log-transformed
return estimates from equation~\ref{eq:4}:

\begin{align}
log(R_{obs,1,t,r}) &\sim Normal(log(R_{1,t,r}), \epsilon_{1,r}), \\
log(R_{obs,2,t,r}) &\sim Normal(log(R_{2,t,r}), \epsilon_{2,r}) \label{eq:5} 
\end{align}

where $R_{obs,1,t,r}$ and $R_{obs,2,t,r}$ are the observed return estimates
for year $t$ and river $r$ of 1SW and 2SW fish, respectively, $\epsilon_{1,r}$
and $\epsilon_{2,r}$ are the process error terms. 
These error terms are normally distributed with a standard deviation
of 0.01, which are almost equivalent to using a coefficient of variation in
return estimates 1\%, given that return estimates in equations~\ref{eq:3}
and~\ref{eq:4} are log-transformed:

\begin{align}
\epsilon_{1,r} &\sim Normal(0, 0.01) \\
\epsilon_{2,r} &\sim Normal(0, 0.01).
\end{align}

Furthermore, we use instantaneous mortality rates in the model instead of survival probabilities
as the model is more efficient in its parameter search in log-space, and instantaneous rates
are easy to interpret. Instantaneous rates are easily converted to survival probabilities by 

\begin{align}
 S_{1} &= e^{-Z_1}, \\
 S_{2} &= e^{-Z_2}. 
\end{align}

We estimate population-level mean \Pg values around which the yearly \Pg
values are normally distributed. We specify different informative hyperpriors
for \prmu and \prsig based on whether the population is 1SW-dominated or not:

\begin{align}
    logit(P_{r,t}) &\sim Normal(\mu_r, \sigma_r) \\
    \mu_r &\sim 
    \begin{cases}
       Normal(2.3, 0.4),  &\text{for 1SW-dominated populations} \\
       Normal(0, 2.8), &\text{for non-1SW-dominated populations} \\
   \end{cases} \\
    \sigma_r &\sim halfNormal(0, 1).
\end{align}

The priors for $Z_1$ and $Z_2$ are specified as log-normal distributions:

\begin{align}
Z_1 &\sim logNormal(1, 0.22),   \\ 
Z_2 &\sim logNormal(0.2, 0.3).
\end{align}

The model was written in Stan \citep{Carpenter2017} and run in R version 3.6.1
\citep{RCoreTeam2019} using the \texttt{rstan} package version 2.19.2
\citep{StanDevelopmentTeam2019}.
The model was run with three chains and 3,000 iterations, with the first 1,500
discarded as a burn-in. The models were considered to have converged when the
$\hat R$ of all parameters were lower than 1.03 and the effective sample size 
were higher than 500.

\subsection*{Correlations among trends in survival}

%\comment{How do were quantitatively compare trends among rivers? DFA is an
%option, but so is a t-test of averages before and after a certain year}

We looked at the correlation of trends in \So by calculating the Pearson's
correlation coefficient for the Z-scores of these trends. Given that the time
series do not cover the same years, and that some rivers have missing years in
the middle of the time series, pair-wise Pearson's tests were done using only
the years where there is data for both rivers.

\section*{Results}

%\subsubsection*{Model convergence}

\subsubsection*{Trends in marine survival parameters}

Trends in estimates of \So were highly variable within and among rivers
(Fig.~\ref{fig:s1-dual}). The highest median posterior estimates of \So
were for the Nashwaak River in 2006 and 2008, with values of 0.18 and 0.21,
respectively. The lowest median \So estimate was in the Trinit\'{e} in 2001,
with an estimate of \So of 0.007, while the Conne, LaHave, and Trinit\'{e} had
years where estimates of \So varied between 0.01 and 0.02 (Fig.~\ref{fig:s1-faceted}).

Trends among populations also varied: Campbellton,
Saint-Jean, and Western Arm Brook populations showed increases in \So
over time, Trinit\'{e}, Conne, and La Have populations showed decreases,
while at Nashwaak there was an increase in median \So during the early
2000s but a decrease in the 2010s. Yearly estimates of \So had very little
variability for one sea-winter dominated populations (Conne, Campbellton, and
WAB), but were more variable (i.e. wider credible intervals) in the other
populations.

\begin{figure}[htbp] \centering
    \includegraphics[width=0.95\linewidth]{figures/s1-trends-dual.png}
    \caption{Trends in survival in the first year at sea (\So). a) Posterior
        estimates of \So for the seven rivers examined, error bars indicate
        the 90\% credible intervals, and dashed lines denote years of commercial
        fishing moratoria for each province. b) Z-scores of median posterior
        estimates.} \label{fig:s1-dual} \end{figure}

\begin{figure}[htbp] \centering
    \includegraphics[width=0.95\linewidth]{figures/s1-trends-faceted.png}
    \caption{Posterior estimates of \So for the seven rivers examined, error
        bars indicate the 90\% credible intervals.} \label{fig:s1-faceted}
\end{figure}

Estimates of \St were highly uncertain in all rivers, and trends 
were not apparent in most rivers given the large range of the credible
intervals in the yearly estimates (Fig.~\ref{fig:s2-faceted}). The estimates
of \St for the Saint-Jean and Trinit\'{e} were considerably less uncertain
than for the other populations, but showed no apparent trends through time.

\begin{figure}[htbp] \centering
    \includegraphics[width=0.95\linewidth]{figures/s2-trends-faceted.png}
    \caption{Posterior estimates of \St for the seven rivers examined, error
        bars indicate the 90\% credible intervals.} \label{fig:s2-faceted}
\end{figure}

Estimates of \Pg were mostly stable across time, except for the LaHave and
Nashwaak Rivers; The estimates of \Pg were slightly lower in the last four
years than in the previous ones, while in the Nashwaak the posterior estimates
of \Pg in 2012 were much lower than in all other years
(Fig.~\ref{fig:s2-faceted}). Uncertainty in yearly estimates was highest in
the LaHave, Nashwaak, and Trinit\'{e}, and lowest in the 1SW-dominated
populations.

\begin{figure}[htbp] \centering
    \includegraphics[width=0.95\linewidth]{figures/pr-trends-faceted.png}
    \caption{Posterior estimates of \Pg for the seven rivers examined, error
        bars indicate the 90\% credible intervals.} \label{fig:pr-faceted}
\end{figure}

Population-level estimates of \prmu and \prsig varied considerably among rivers. For all
three 1SW-dominated rivers (Campbellton, Conne, and WAB), estimates of \prmu
were very close to 1.0 and had little variability in \prsig (Fig~\ref{fig:prmu-post}).
Estimates of \prmu were the lowest for the two QC rivers, particularly the Saint-Jean (median \prmu = 0.11).
Estimates of \prmu for the Nashwaak and the LaHave Rivers were close to 0.5, with these two rivers having 
the highest estimated values of \prsig, particularly the Nashwaak (Fig~\ref{fig:prmu-post}).

\begin{figure}[htbp] \centering
    \includegraphics[width=1.0\linewidth]{figures/pr-mu-posteriors.png}
    \caption{Posterior estimates of the population-level parameters $Pr_{\mu}$
       $logit(Pr_{\mu})$, and $logit(Pr_{\sigma})$. Dots denote median estimates, while the thick and thing error bars indicate
       the 50\% and 90\% credible intervals, respectively.} 
   \label{fig:prmu-post} 
\end{figure}

\subsubsection*{Correlations}

Most correlations between z-scored trends were not statistically significant, with only 
three out of the 21 pair-wise comparisons having a p-value below 0.01 (Fig.~\ref{fig:s1-corr}a).
When looking at the direction of the correlation, regardless whether they were
significant or not, these spanned both positive and negative coefficients
(Fig.~\ref{fig:s1-corr}b).

\begin{figure}[htbp] \centering
    \includegraphics[width=0.95\linewidth]{figures/corr-s1.png} \caption{
        Correlation among Z-scores of median estimated trends in \So among
        rivers. a) Pearson's correlation coefficients are shown in each square,
        while colouring denotes significance of the correlation ($p \leq 0.01$), b)
        colours denote direction and magnitude of correlation, while asterisks denote significance.}
\label{fig:s1-corr} 
\end{figure}

\section*{Discussion} 

% 1. Main findings
% 2. How they compare to other publications
% 3. Potential reasons
% 4. Caveats
% 5. Future directions

% 1. Main findings
Our results challenge the narrative that marine survival, specifically survival in the first year at sea, is declining
uniformly throughout the range of Atlantic salmon in the northwest Atlantic.
Temporal trends are not consistent among populations. 
Over the time periods for which data were available, some rivers show positive trends in survival in the
first winter at sea (\So) while other exhibit highly variable yet stable trends, and some show
declines. We could not assess trends in the second winter at sea (\St) or
proportion returning as grilse (\Pg), as these parameter estimates were highly
uncertain and were strongly influenced by the priors.
Perhaps there is a need to rethink our understanding of Atlantic salmon
population dynamics in light of the possibilities that (1) any real and persistent decline in marine survival was
experienced by some but not necessarily all populations, (2) reductions in survival might
have occurred over a relatively brief period of time and have not persisted, and (3) marine survival has
been relatively stable, or increasing in some populations for one or more decades.

% 2. How they compare to other publications
Our results are contrary to those of \citet{Olmos2019}, who detected positive
correlations in post-smolt survival among spatially broad stock units. These differences
could be due to a number of reasons: different model specifications and
structure, different methods for estimating covariance, difference in the
spatial scales of data sources (i.e. river vs province scales), and perhaps most importantly the use stock-recruitment relationships
rather than empirical smolt count data to estimate marine survival.
As trends in marine survival during the first winter at sea are highly independent
among rivers on relatively small spatial scale, trends from broader
geographical areas (i.e. province, state, or country-wide estimates) may not
be representative.
Interestingly, our estimates of \So and \St are very similar to those produced
by \citet{Chaput2003b}, and our trends are almost identical for the
overlapping time period that marine survival was estimated for in their study
(1984-1998). 
While \citet{Chaput2003b} separated abundance data for males and females
and assumed their survival rates were the same (to be able to reach an
analytical solution), our study reached almost the same results (albeit with
slightly higher uncertainty), using a Bayesian approach with informative
priors. These overlapping trends obtained with two different methods 
suggest that our method is effective at estimating marine survival.

Trends in marine survival
among populations were compared by \citet{Chaput2012a} using adult return rates.
He found that for 4 of 6 populations examined, return rates in the 1990s 
were lower than those during the 1970s.
\citet{Friedland1993} compared return rates for a number of rivers in eastern
North America between 1973 and 1988, and suggested there are similar trends among these. 
However, the similarity in these trends was driven primarily by two years, 1977 and 1978, which
show concurrent low and high relative return rates across rivers,
respectively. Other years are much more variable relative to each other.
\citeauthor{Friedland1993}'s \citeyear{Friedland1993} time series ends in  
1988; thus there are only a few years for which to assess overlap with the
time series in our study.
In any event, we caution that the pooling of adult return rates \citep{Chaput2012a, Friedland1993} 
can mask inter-annual variability in marine survival,
and hence might not produce an accurate depiction of marine survival trends.
\citet{Dempson2003} described a general declining trend in marine survival for
Newfoundland rivers (except WAB); we drew the same conclusion for 
Conne River but not Campbellton River or WAB. It is not possible to draw broader conclusions
with data from only three Newfoundland rivers, but it seems that among index rivers,
those in Newfoundland are among those with the highest marine survival rates.

% 3. Potential reasons
There are a variety of potential explanations for the lack of synchronous
trends in estimates of \So. 
Marine survival in the first winter at sea could be highly variable between
populations because of the predominance of spatially local environmental drivers of survival (e.g., temperature, predation) 
relative to broader-scale, even ocean-wide, drivers.
The synchrony reported for marine survival trends at broader spatial scales \citep{Olmos2019}
might be attributable to the use of stock-recruitment relationships to estimate survival,
relationships that may have been confounded by changes in recruitment dynamics.
There is some evidence of a correlation between return rate and growth (as
indicated by inter-circuli spacing on scales), where years of poor growth
tended to also be years of poor survival \citep{Friedland1993}, supporting the
idea that environmental variability can affect marine survival.
Furthermore, among European salmon, there is evidence of a positive correlation
between spring temperature in the Norwegian and North Seas and population abundance, suggesting warmer
conditions favour post-smolts \citep{Friedland1998}, based on mapping the
extent of area of suitable temperature (7-13 \textdegree C).

Nonetheless, the causal mechanisms for why warming should affect post-smolt
survival almost certainly differs depending on the difference between
temperature experienced by the post-smolts and their respective
population-specific thermal optima. 
This difference could explain why populations in eastern North America are
declining in the southern part of their range but potentially increasing
further north, and also why some studies find positive correlations between
temperature and abundance \citep{Friedland1998, Friedland1998b, Jonsson2004}
while others find negative ones \citep{Friedland1993, Todd2008}.
Putative associations between temperature and direct estimates of marine
survival warrants further study at the population level.

While there is little evidence that marine survival is density-dependent in
Atlantic salmon \citep{Jonsson1998,Gibson2006}, there could potentially be
some density-dependent processes during parts of the post-smolt migration
period, particularly for populations that are likely to be subjected to
declining per capita population growth rates ($r$) generated by Allee effects.
Exploring relationships between survival and population size could potentially
shed light about the processes that have caused many of the population
declines that have been documented.

Oceanic conditions have been correlated with abundance trends and growth
\citep{Todd2008}, however, the mechanism by which such bottom-up effects, \
mediated by changes in food availability,
affect population dynamics beyond marine survival needs to
be thoroughly reassessed. If marine survival on its own cannot fully explain
trends in abundance, then there are potential carry-on effects of oceanic
conditions that manifest with regards to fresh production. 
For example, adults
that return to spawn after spending suboptimal conditions at sea might be less
likely to make it to their spawning grounds, successfully secure a mate,
produce fewer eggs, or produce eggs with lower per capita fitness than those
produced by adults which grew in optimal oceanic conditions.
As larger females tend to be more productive, in terms of fecundity and total
reproductive energy, than the same weight's worth of smaller females
\citep{Barneche2018}, a small decrease in body condition resulting from bottom-up
impacts on food availability could potentially have disproportionate effects on fecundity
and fitness of the offspring.

Egg-to-smolt survival in Atlantic salmon is highly variable \citep{Klemetsen2003,Chaput2015}
and changes in the oceanic conditions that spawners experience could be
contributing to this variability.
Obviously, there would be a time lag (perhaps as much as a generation) in how such effects
might be manifest at the adult stage.
However, given that most correlations are between relatively monotonic declines
in abundance coupled also monotonic increases in climatic indices
over decadal time scales \citep[e.g.,][]{Friedland1998, Todd2008,
    Beaugrand2012}, it would be expected that this correlations would be
maintained even if salmon abundances were lagged by a generation length.


% 4. Caveats
As with all novel modelling approaches, there are caveats to acknowledge.
The seven populations explored in the present study might not be representative
of regional trends in marine survival. However, there are no other
long-term time series of smolts and adult returns to draw inferences from.
While there are analytical issues associated with the estimation of \So, \St, and \Pg,
the assumption that \St is additive to \So could produce unrealistic results.
We know there is a period of a few months where 1SW
returns are subject to a different environment than those salmon that will
return as 2SW the next year. 
While this is not ideal,
overcoming this assumption would require an additional parameter to be
estimated, or an additional assumption as to what proportion of \So is not
additive to \St (as the returning 1SW adults do not experience the same
environment when they return to their natal streams as those fish who stayed
at sea for an additional winter before returning to spawn).

Secondly, the assumed hierarchical structure might not be the most appropriate
for modelling smolt abundances. This approach results in shrinkage to the
mean, which means that the variability of yearly smolt estimates is less than
it would be if not modelled hierarchically. 
That said, this assumption is likely to result in more conservative trends in marine survival, as the
variation is smolt estimates is reduced.
It is important to note that the hierarchical structure in the estimation of
\Pg seems like a reasonable assumption given that the probability of returning
as grilse has a genetic component associated to it \citep{Aykanat2019} and is
not expected to vary much, within a population, among years.

% 5. Future directions
Perhaps a reframing of the issue of marine survival is key to furthering our
understanding of Atlantic salmon population dynamics. Marine survival may not
have declined consistently, and over the same time periods, across all
populations. 
But the fact that it has remained at roughly similar levels as it was
previously \emph{despite} reduced commercial fishing mortalities, suggests
that there may well be an interaction between small population size (small
relative to unfished population size or carrying capacity), recovery
potential, and environmental stochasticity that has not been fully explored in
Atlantic salmon. 
All else being equal, relatively small populations are more vulnerable to
demographic, environmental, and genetic stochasticity than large populations
\citep{Lande1993, Hutchings2015}. Interactions between population size and the
demographic consequences of environmental stochasticity appear to have
affected recovery in many marine fishes that have exhibited impaired recovery
since mitigation of the threat posed by fishing mortality
\citep{Hutchings2017, Hutchings2020}. The possibility that similar
interactions may be impairing the recovery of wild Atlantic salmon merits
study.

\section*{Acknowledgements}

% We would like to thank Geir Bolstad, G\'{e}rald Chaput, Brian Dempson, and
% Martha Robertson for their useful discussions on estimating marine survival
% in Atlantic salmon, and Amanda Kissel for her helpful comments on the
% manuscript. Brian Dempson and Geoff Venoit for providing the data Conne
% River data.
We would like to thank G\'{e}rald Chaput for his useful discussions on
estimating marine survival in Atlantic salmon, Carmen David for her comments
on the manuscript, and Sean Anderson for his help with implementing the
non-centered parameterization of the model. This research was supported by the
Atlantic Salmon Conservation Foundation and the Atlantic Salmon Research Joint
Venture.

\section*{Conflicts of Interest}

The authors declare no conflicts of interest.
 
\bibliography{subset}

%\documentclass[12pt]{article}
\usepackage[top=0.85in,left=1.0in,right=1.0in,footskip=0.75in]{geometry}
%\usepackage[parfill]{parskip}
\usepackage{setspace}
\usepackage{lineno}
\usepackage[hidelinks]{hyperref}
%\onehalfspacing
\doublespacing

\usepackage[round,sectionbib]{natbib}
\setcitestyle{authoryear}
\bibpunct{(}{)}{;}{a}{}{;}
\bibliographystyle{fishfishnourl}

%\usepackage{textcomp}
%\usepackage{libertine}
%\usepackage{inconsolata} % sans serif typewriter

\usepackage{mathtools} % for dcases
\usepackage{xcolor} % for textcolor
\usepackage{makecell} % for \makecell and within cell line breaks (\thead)

\usepackage[T1]{fontenc}

%\makeatletter\let\expandableinput\@@input\makeatother % expandable input for \input inside tables

% Linux Libertine:
\usepackage{textcomp}
\usepackage[sb]{libertine}
\usepackage[varqu,varl]{inconsolata}% sans serif typewriter
\usepackage[libertine,bigdelims,vvarbb]{newtxmath} % bb from STIX
\usepackage[vvarbb]{newtxmath} % bb from STIX, removed bigdelims for ScholarOne rendering
\usepackage[cal=boondoxo]{mathalfa} % mathcal
%\useosf % osf for text, not math
%\usepackage[supstfm=libertinesups,%
%  supscaled=1.2,%
%  raised=-.13em]{superiors}

\usepackage{xspace}
\usepackage{xfrac} % for diagonal inline fractions in text
%\usepackage{array} % for making whole row bold in table
\usepackage{colortbl} % for background colours in table rows
\usepackage{longtable}
\usepackage{amssymb} % for \checkmark 
\usepackage{amsmath} % for \checkmark 
\usepackage{rotating}
\usepackage[nolists,tablesfirst]{endfloat} % for putting figs and tables at end of document
\DeclareDelayedFloatFlavor{sidewaystable}{table}
\usepackage{makecell} % for \makecell in tables
\usepackage{doi}

%\usepackage{xr} % to obtain label references from supp materials file
%\externaldocument[S-]{suppmat}

\usepackage{tabularx}
\usepackage{booktabs}
\usepackage{array} % for table wrapping of columns
\newcolumntype{L}[1]{>{\raggedright\let\newline\\\arraybackslash\hspace{0pt}}m{#1}}
\newcolumntype{C}[1]{>{\centering\let\newline\\\arraybackslash\hspace{0pt}}m{#1}}
\newcolumntype{R}[1]{>{\raggedleft\let\newline\\\arraybackslash\hspace{0pt}}m{#1}}
%\usepackage[detect-all]{siunitx} % for SI units

% custom hyphenation:
\hyphenation{inverse}
\hyphenation{At-lantic}
\hyphenation{elasmo-branchs}


% Macros
\newcommand{\So}{$S_{1}$\xspace}
\newcommand{\St}{$S_{2}$\xspace}
\newcommand{\Pg}{$P_r$\xspace}
\newcommand{\prmu}{$\mu_r$\xspace}
\newcommand{\prsig}{$\sigma_r$\xspace}
\newcommand{\Linf}{$L_{\infty}$}
\newcommand{\DWinf}{$DW_{\infty}$}
\newcommand{\alphat}{$\tilde{\alpha}$}
\newcommand{\lamat}{$l_{\alpha_{mat}}$}
\newcommand{\lamatb}{$l_{\alpha_{mat}}b$}
\newcommand{\rmax}{$r_{max}$\xspace}
\newcommand{\ageratio}{$\alpha_{mat}/\alpha_{max}$}
\newcommand{\yr}{year\textsuperscript{-1}}
\newcommand{\rsq}{$R^2$\xspace}
\newcommand{\mytilde}{\raise.17ex\hbox{$\scriptstyle\mathtt{\sim}$}}
% Select what to do with command \comment:  
%\newcommand{\comment}[1]{}  % comment not shown
\newcommand{\comment}[1]{\par {\bfseries \color{blue} #1 \par}} % comment shown
%% END MACROS SECTION

\begin{document}
\linenumbers


\section*{Trends in marine survival of Atlantic salmon in eastern Canada}

\textbf{Sebasti\'{a}n A. Pardo\textsuperscript{1*}, 
        Geir H. Bolstad\textsuperscript{2}, 
        J. Brian Dempson\textsuperscript{3}, 
        Julien April\textsuperscript{4}, 
        Ross A. Jones\textsuperscript{5}, 
        Martha J. Robertson\textsuperscript{3}, 
        Dustin Raab\textsuperscript{6}, 
Jeffrey A. Hutchings\textsuperscript{1}} 

\noindent\small{\textsuperscript{1} Department of Biology, Dalhousie University, Halifax, NS, Canada\\}
\small{\textsuperscript{2} Norwegian Institute for Nature Research (NINA), Trondheim, Norway\\}
\small{\textsuperscript{3} Fisheries and Oceans Canada, St. John's, NL, Canada\\}
\small{\textsuperscript{4} Minist\`{e}re des For\^{e}ts, de la Faune et des Parcs, Qu\'{e}bec, QC, Canada\\}
\small{\textsuperscript{5} Fisheries and Oceans Canada, Moncton, NB, Canada\\}
\small{\textsuperscript{6} Fisheries and Oceans Canada, Dartmouth, NS, Canada\\}
\small{\textsuperscript{*} Corresponding author: spardo@dal.ca}

\section*{Abstract}

Declines in wild Atlantic salmon (\emph{Salmo salar}) throughout the north
Atlantic are primarily attributed to declining survival at sea. This
hypothesis has proven challenging to test on a river-by-river basis because of
the need to model data on both migrating smolts and returning adults and to
simultaneously estimate multiple parameters, especially for salmon spending
more than one winter at sea (1SW) before spawning. We fit a hierarchical Bayesian
maturity schedule model to data (19-42 years) for seven populations in  
Newfoundland and Labrador (NL), Qu\'{e}bec (QC), New Brunswick (NB), and Nova
Scotia (NS), Canada. We estimate survival in the first (\So) and second year at sea (\St),
and the proportion returning as 1SW adults (\Pg). Trends in  were not consistent
among rivers. Since 1990, \So increased at Western Arm Brook (NL)
and Rivi\`{e}re Saint-Jean (QC), but declined at Conne River (NL) and Rivière
de la Trinité (QC). Since the mid-1990s, \So increased at Campbellton River (NL),
declined at LaHave River (NS), and fluctuated at Nashwaak
River (NB). Estimates of \St were highly uncertain, particularly for
1SW-dominated populations; \Pg was generally stable. These results challenge the
narrative that marine survival has changed in a temporally consistent manner
among spatially disparate populations. Our findings suggest that, at the
population level, changes in abundance are attributable to temporal shifts in
multiple components of individual fitness, including at-sea survival. If
salmon populations do not respond in a consistent, uniform manner to changing
environmental conditions throughout their range, future research initiatives
should explore why.

% The marine phase of anadromous Atlantic salmon (\emph{Salmo salar}) is the
% least known yet one of the most crucial with regards to population
% persistence. Declines in many Atlantic salmon populations in eastern
% Canada have often been attributed to changes in conditions at sea, negatively
% affecting their survival. However, marine survival estimates are difficult to
% obtain given that many individuals spend multiple winters in the ocean before
% returning to freshwater to spawn, necessitating the estimation of multiple
% parameters. To do so, we fit a hierarchical Bayesian maturity schedule model
% to smolt and adult abundance time series for seven populations located in
% Newfoundland and Labrador (NL), Qu\'{e}bec (QC), New Brunswick (NB), and Nova
% Scotia (NS). The datasets ranged between 19 and 42 years in length. We
% estimated three components of marine survival: survival in the first year at
% sea (\So), survival in the second year at sea (\St), and proportion returning
% as one sea-winter adults (\Pg). Controlling for time frame, trends in
% estimates of \So were not consistent among rivers.
% In the four populations for which data extended to 1990, marine survival
% during the first year at sea predominantly increased over time in Western Arm
% Brook (NL) and Rivi\`{e}re Saint-Jean (QC) but largely declined in Conne River
% (NL) and Rivi\`{e}re Trinit\'{e} (QC). In the three other populations, \So
% exhibited an increase in Campbellton River (NL), a decline in LaHave River
% (NS), and fluctuating stability in Nashwaak River (NB) since approximately the
% mid-1990s. 
% Estimates of \St were highly uncertain, particularly for 1SW-dominated rivers
% (Conne, Campbellton and WAB, where the abundance of 2SW returns is very low)
% and thus the posterior distributions matched our choice of prior and trends
% could not be assessed. 
% \Pg was temporally stable within rivers, except for LaHave and Nashwaak Rivers. 
% Our findings challenge the assumption that marine survival has changed in a consistent
% manner among populations across a broad geographic scale, and suggest that
% trends can be river-specific. 
% The well established correlations between climate variables and abundance are
% not being mediated solely by marine survival, and there can potentially be
% some important indirect effects on fecundity.
% Further work exploring potential correlates of marine survival and the
% potential non-lethal effects of climate effects is warranted.


Keywords: salmonid, survival at sea, natural mortality, marine mortality

%Running Head: 

\section*{Introduction} % (4-5 paragraphs)

%1. Declining salmon numbers

Reductions in fishing mortality, albeit necessary, are not always sufficient
to facilitate population recovery. Experience with numerous commercially
exploited marine fisheries since the early 1990s has shown that not all
populations respond as favourably as anticipated to major reductions in
exploitation \citep{Hutchings2017}. Gradual efforts to close commercial
Atlantic salmon (\emph{Salmo salar}) fisheries in eastern Canada culminated in
full moratoria in all regions, beginning in the Maritime provinces (1984) and
following in Newfoundland (1992), Labrador (1998), and Qu\'{e}bec (2000). Since
these closures, many populations have not increased as 
expected \citep{Dempson2004, ICES2019}; some 
have been assessed as considered threatened or endangered by the 
Committee on the Status of Endangered Wildlife in Canada \citep[][]{Cosewic2010}, 
Canada's national science advisory body (to the national government) on
species risk of extinction.
While it is not fully understood what is driving population declines, the potential
drivers of these are many \citep[see ][for a detailed discussion of possible
causes]{Cairns2001}, including but not limited to: fishing mortality \citep{Dempson2004}, 
damming of waterways and changes in the freshwater habitat \citep{Dunfield1985}, acidification
\citep[particularly in the Southern Uplands region of
NS, see][]{Gibson2010}, predation by seals and birds \citep{Cairns2000}, negative
effects of interbreeding or interactions with escaped farmed salmon
\citep{Keyser2018}, and climate-driven changes in survival and productivity \citep{Mills2013}.

% 2.
Over the past three decades, a narrative has emerged that marine survival of
Atlantic salmon has declined throughout the North Atlantic \citep{ICES2019}.
%\citep{Hansen1998,OMaoileidigh2003,Chaput2012a}.
Based on multiple lines of evidence that climate conditions can directly and
indirectly influence the abundance and productivity of Atlantic salmon
populations \citep{Mills2013,Almodovar2019}, it has been presumed that oceanic climate effects are
driving population dynamics primarily through changes in marine survival.
An implicit assumption is that any trend in
survival in ocean habitat that is shared by multiple populations during their
seaward migration period will be experienced similarly. 
Put another way, given that salmon from different rivers 
are hypothesized to share marine habitat during some of their time at sea, it
has been presumed that populations share similar temporal trends in
at-sea mortality \citep{Friedland1993, Friedland1998, Russell2012}.  
A recent study by \citet{Olmos2019} suggested that trends in post-smolt
survival, when estimated at the stock level, are synchronously declining
for all Atlantic salmon in eastern North America.

% This perception is widespread both in the scientific
% literature as well as federal reports. The latest status report on Atlantic
% salmon by COSEWIC states that ``While the mechanism(s) of marine mortality is
% uncertain, what is clear is that the recent period of poor sea survival is
% occurring in parallel with many widespread changes in the North Atlantic
% ecosystem.'' \citep{Cosewic2010}. Consequently, there have several lines of
% inquiry as to what the causes behind these declines might be attributable to
% \citep{Friedland1993, Friedland1998}.


In contrast to the narrative of widespread, demographically similar increases
in at-sea mortality, the conservation status of Canadian salmon populations differs
considerably. Populations in the southern part of their range are more
likely to be assessed as being of conservation concern than those in more
northerly regions \citep{Cosewic2010}. This geographical disparity in status
suggests that if marine survival has been, or is, a key factor responsible for
most population declines, these changes are not uniformly distributed across
all populations. 

% Nonetheless, most of the populations declines assessed by COSEWIC have
% occurred in populations in the southern extent of its distribution:
% populations in the Bay of Fundy, Anticosti Island, and the Atlantic coast of
% Nova Scotia being assessed as Endangered, populations in the south coast of
% Newfoundland assessed as Threatened, And the populations in the New Brunswick
% and Qu\`{e}bec coasts of the Gulf of St. Lawrence assessed as Special Concern
% \citep{Cosewic2010}. On the other hand, northermost populations have shown
% stable, or even increasing population trends, suggesting that, if marine
% survival were to be a factor in many of these declines, these changes are not
% uniformly distributed across all populations.

% While the purported decline of marine survival is mentioned widely in the 
% scientific literature, only relatively few studies have quantified it in detail.
% The basis for this premise is a limited number of time series that exhibit
% temporal declines in a proxy of marine survival (i.e. return rates or
% post-smolt survival).

% 3. Nonetheless, this premise has never been examined in detail
Given the logistical challenges associated with estimating at-sea survival, it
is not surprising that the number of studies that have estimated temporal
trends has been limited. An additional limitation has been the derivation of
proxies (e.g., return rates), rather than direct model-based estimates, of
marine survival.
\citet{Chaput2012a}, for example, examined the return rate of smolts to adult salmon 
as a metric of marine survival, finding that most Canadian populations 
had experienced declining return rates. 
However, examination of trends in return rates alone
can mask changes in differential survival during different years at sea, as well
as changes in the proportion of adults returning after one or two years at sea.
Recently \citet{Olmos2019} suggested that trends in post-smolt
survival, when estimated at the stock unit level, are synchronously declining
for all Atlantic salmon in Eastern North America and Canada, a conclusion ultimately 
grounded on the veracity of highly variable stock-recruitment relationships.

%4. 
In the present study, we compile data on the number of migrating smolts and number of returning adults 
for seven wild Canadian populations of Atlantic salmon to model trends in marine survival.
While some studies have previously used maturity-schedule models to estimate marine
survival for a limited number of salmon populations \citep{Chaput2003b}, none
have incorporated data extending over multiple decades, nor have they examined
trends among more than two or three populations. 
Here, we develop a hierarchical Bayesian model that uses Murphy's maturity
schedule method, in conjunction with informative priors, to estimate yearly
marine survival in salmon. In addition to accounting for observation error in
smolt and return estimates, we estimate the proportion of salmon returning
after one winter hierarchically.

\section*{Methods}

\subsection*{Data}

We obtained time series data of outmigrating smolt and returning adult
abundances for seven Atlantic salmon populations in eastern Canada, encompassing a
wide range of the species' distribution (Fig.~\ref{fig:map}). 
Populations included the LaHave River in the Southern Uplands region of Nova
Scotia (NS), Nashwaak River, New Brunswick (NB), Rivi\`{e}re de la Trinit\'{e} (Trinit\'{e}) and
Saint-Jean rivers in Qu\'{e}bec (QC), and  Western Arm Brook (also referred to as  WAB), Campbellton, and
Conne River, Newfoundland (NL). 
Data were collected in NS, NB, and Newfoundland
and Labrador (NL) by Fisheries and Oceans Canada (DFO) and in QC
by the Minist\`{e}re des For\^{e}ts, de la Faune et des Parcs, Qu\'{e}bec.

\begin{figure}[htbp] \centering
    \includegraphics[width=0.85\linewidth]{figures/rivers-map2.png}
    \caption{Locations of the seven rivers in eastern Canada with time series abundance data of outmigrating smolts and 
    returning adults.} \label{fig:map} 
\end{figure}

\subsubsection*{Smolt and adult return abundance data}

Smolt and adult return abundance estimates originate from a variety of
sources. Smolt estimates from the Trinit\'{e}, Saint-Jean, and Conne populations were obtained using a
mark-recapture approach, while estimates from the ..., WAB, and Campbellton populations
were obtained by direct counts using fish counting fences.
For further details on the data collection methodologies refer to 
\citet{Dempson1991, Venoitt2018} for NL populations, 
\citet{April2018}  for QC populations,
\citet{Jones2014} for NB populations,
and \citet{Gibson2009} for NS populations. 

Yearly return data are often recorded in terms of two size groups: small ($< 63$ cm
FL) and large ($\geq 63$ cm FL) salmon, as these closely represent different
life-history strategies (i.e. 1SW and 2SW), but can be confounded with repeat
spawners of different sizes. To correct for this in returns 
reported as small and large salmon, we estimated the abundance
of 1SW and 2SW returns using yearly scale age data of a subsample of returns:

\begin{equation}
    p_{r,t,a} = \frac{\sum_{s}{(\frac{n_{r,t,s,a}}{n_{r,t,s}} * N_{r,y,s})}}{\sum_{s}{N_{r,t,s}}}
\end{equation}

where $p_{r,t,a}$ is the proportion of annual returns in river $r$, year $t$,
and of spawning history $a$ (either 1SW or 2SW returns); $n_{r,t,s,a}$ is the
number of samples in river $r$, year $t$, of spawning history $a$, and of size
group $s$; $n_{r,t,s}$ is the total number of samples in river $r$, year $t$,
and of size group $s$; and $N_{r,t,s}$ is the returns of salmon
in river $r$, year $t$, and of size group $s$.

\subsection*{Bayesian model}

We developed a hierarchical Bayesian model that uses Murphy's maturity
schedule method, in conjunction with informative priors, to estimate yearly
marine survival in seven populations of Atlantic salmon. We account for
observation error in smolt and return estimates, as well as estimating the
proportion returning after one winter (i.e. \Pg) hierarchically.
There is an identifiability problem in the maturity schedule equations where
the parameter estimates cannot be optimally solved \citep{Chaput2003a}.
However, this issue can be mathematically overcome, at least partially, by
using informative priors for all three marine survival parameters in a
Bayesian framework.
This method requires abundance estimates of smolts as well as abundance estimates
of returning one-sea-winter (1SW) and two-sea-winter adults. With these data,
it estimates three parameters: survival in the first year at sea (\So), survival
in the second year at sea (\St), and the proportion of fish returning after one
year at sea (\Pg). 

Our model does not include repeat spawners and assumes that no fish spend
three or more winters at sea before returning to spawn for the first time.
The model also assumes that mortality in the second winter at sea (\St)
is additional to mortality in the first winter at sea in the previous year, 
and therefore does not account for differences in environmental conditions experienced
between 1SW and 2SW fish of the same smolt cohort during their overlapping first year at sea.
In other words, our model assumes that the decision of returning occurs just before
actually being counted as returns and that \St is any additional mortality in
the subsequent year. 

%\comment{Other assumptions...}

Observed smolt estimates were modelled hierarchically and included
observation error:

\begin{equation}
log(smolts_{obs,t,r}) = log(smolts_{true,t,r}) + \epsilon_{t,r}
\end{equation}

where $smolts_{true,t,r}$ are the true smolt abundances for year $t$ and river
$r$, and $\epsilon_{t,r}$ is the error term, which is empirically derived from
by calculating the yearly coefficient of variation in the empirically derived
smolt estimates (see Table S1 in the Supplementary material). 
Where available, we used population-specific measurement error estimates for smolt abundances; if not 
available, we set measurement error at 5\%. 
The log-transformed true smolt abundances are
normally distributed around a population-level mean and standard deviation:

\begin{equation}
log(smolts_{true,t,r}) \sim Normal(\mu_{smolts,r}, \sigma_{smolts,r})
\end{equation}

where $\mu_{smolts,r}$ and $\sigma_{smolts,r}$ are parameters estimated by the
model for each river.

Once we have yearly estimates of smolt, 1SW, and 2SW abundances, we estimate
marine survival parameters using Murphy's maturity schedule method
\citep{Murphy1952, Ricker1975}:

\begin{align}
    R_{1,t} &= smolts_{true,t-1} * S_{1,t} * Pr_t \label{eq:1}, \\
    R_{2,t+1} &= smolts_{true,t-1} * S_{1,t} * (1 - Pr_t) * S_{2,t+1} \label{eq:2}
\end{align}

where $R_{1,t}$ and $R_{2,t+1}$ are the estimated abundances of 1SW and 2SW
salmon returning in years $t$ and $t+1$, respectively, $smolts_{true,t-1}$ is the
estimated number of outmigrating smolts in year $t-1$, $S_{1,t}$ is the proportion of
salmon surviving in their first year ($t$) at sea, $Pr_t$ is the proportion of
salmon that return to spawn at year $t$, $S_{2,t+1}$ is the survival in their
second year at sea of the same cohort of salmon who did not return to spawn at
year $t$.

However, to allow for normally and log-normally-distributed parameters we
log-transform equations~\ref{eq:1} and~\ref{eq:2} to obtain

\begin{align}
    log(R_{1,r,t}) &= log(smolts_{r,t-1}) + log(Pr_{r,t}) - Z_{1,r,t} \label{eq:3}, \\
    log(R_{2,r,t+1}) &= log(smolts_{r,t-1}) - Z_{1,r,t} + log(1 - Pr_{r,t})  - Z_{2,r,t+1} \label{eq:4} 
\end{align}

where $Z_{1,r,t}$ and $Z_{2,r,t+1}$ are the instantaneous mortality rates.
Process error was included as the standard deviation of the log-transformed
return estimates from equation~\ref{eq:4}:

\begin{align}
log(R_{obs,1,t,r}) &\sim Normal(log(R_{1,t,r}), \epsilon_{1,r}), \\
log(R_{obs,2,t,r}) &\sim Normal(log(R_{2,t,r}), \epsilon_{2,r}) \label{eq:5} 
\end{align}

where $R_{obs,1,t,r}$ and $R_{obs,2,t,r}$ are the observed return estimates
for year $t$ and river $r$ of 1SW and 2SW fish, respectively, $\epsilon_{1,r}$
and $\epsilon_{2,r}$ are the process error terms. 
These error terms are normally distributed with a standard deviation
of 0.01, which are almost equivalent to using a coefficient of variation in
return estimates 1\%, given that return estimates in equations~\ref{eq:3}
and~\ref{eq:4} are log-transformed:

\begin{align}
\epsilon_{1,r} &\sim Normal(0, 0.01) \\
\epsilon_{2,r} &\sim Normal(0, 0.01).
\end{align}

Furthermore, we use instantaneous mortality rates in the model instead of survival probabilities
as the model is more efficient in its parameter search in log-space, and instantaneous rates
are easy to interpret. Instantaneous rates are easily converted to survival probabilities by 

\begin{align}
 S_{1} &= e^{-Z_1}, \\
 S_{2} &= e^{-Z_2}. 
\end{align}

We estimate population-level mean \Pg values around which the yearly \Pg
values are normally distributed. We specify different informative hyperpriors
for \prmu and \prsig based on whether the population is 1SW-dominated or not:

\begin{align}
    logit(P_{r,t}) &\sim Normal(\mu_r, \sigma_r) \\
    \mu_r &\sim 
    \begin{cases}
       Normal(2.3, 0.4),  &\text{for 1SW-dominated populations} \\
       Normal(0, 2.8), &\text{for non-1SW-dominated populations} \\
   \end{cases} \\
    \sigma_r &\sim halfNormal(0, 1).
\end{align}

The priors for $Z_1$ and $Z_2$ are specified as log-normal distributions:

\begin{align}
Z_1 &\sim logNormal(1, 0.22),   \\ 
Z_2 &\sim logNormal(0.2, 0.3).
\end{align}

The model was written in Stan \citep{Carpenter2017} and run in R version 3.6.1
\citep{RCoreTeam2019} using the \texttt{rstan} package version 2.19.2
\citep{StanDevelopmentTeam2019}.
The model was run with three chains and 3,000 iterations, with the first 1,500
discarded as a burn-in. The models were considered to have converged when the
$\hat R$ of all parameters were lower than 1.03 and the effective sample size 
were higher than 500.

\subsection*{Correlations among trends in survival}

%\comment{How do were quantitatively compare trends among rivers? DFA is an
%option, but so is a t-test of averages before and after a certain year}

We looked at the correlation of trends in \So by calculating the Pearson's
correlation coefficient for the Z-scores of these trends. Given that the time
series do not cover the same years, and that some rivers have missing years in
the middle of the time series, pair-wise Pearson's tests were done using only
the years where there is data for both rivers.

\section*{Results}

%\subsubsection*{Model convergence}

\subsubsection*{Trends in marine survival parameters}

Trends in estimates of \So were highly variable within and among rivers
(Fig.~\ref{fig:s1-dual}). The highest median posterior estimates of \So
were for the Nashwaak River in 2006 and 2008, with values of 0.18 and 0.21,
respectively. The lowest median \So estimate was in the Trinit\'{e} in 2001,
with an estimate of \So of 0.007, while the Conne, LaHave, and Trinit\'{e} had
years where estimates of \So varied between 0.01 and 0.02 (Fig.~\ref{fig:s1-faceted}).

Trends among populations also varied: Campbellton,
Saint-Jean, and Western Arm Brook populations showed increases in \So
over time, Trinit\'{e}, Conne, and La Have populations showed decreases,
while at Nashwaak there was an increase in median \So during the early
2000s but a decrease in the 2010s. Yearly estimates of \So had very little
variability for one sea-winter dominated populations (Conne, Campbellton, and
WAB), but were more variable (i.e. wider credible intervals) in the other
populations.

\begin{figure}[htbp] \centering
    \includegraphics[width=0.95\linewidth]{figures/s1-trends-dual.png}
    \caption{Trends in survival in the first year at sea (\So). a) Posterior
        estimates of \So for the seven rivers examined, error bars indicate
        the 90\% credible intervals, and dashed lines denote years of commercial
        fishing moratoria for each province. b) Z-scores of median posterior
        estimates.} \label{fig:s1-dual} \end{figure}

\begin{figure}[htbp] \centering
    \includegraphics[width=0.95\linewidth]{figures/s1-trends-faceted.png}
    \caption{Posterior estimates of \So for the seven rivers examined, error
        bars indicate the 90\% credible intervals.} \label{fig:s1-faceted}
\end{figure}

Estimates of \St were highly uncertain in all rivers, and trends 
were not apparent in most rivers given the large range of the credible
intervals in the yearly estimates (Fig.~\ref{fig:s2-faceted}). The estimates
of \St for the Saint-Jean and Trinit\'{e} were considerably less uncertain
than for the other populations, but showed no apparent trends through time.

\begin{figure}[htbp] \centering
    \includegraphics[width=0.95\linewidth]{figures/s2-trends-faceted.png}
    \caption{Posterior estimates of \St for the seven rivers examined, error
        bars indicate the 90\% credible intervals.} \label{fig:s2-faceted}
\end{figure}

Estimates of \Pg were mostly stable across time, except for the LaHave and
Nashwaak Rivers; The estimates of \Pg were slightly lower in the last four
years than in the previous ones, while in the Nashwaak the posterior estimates
of \Pg in 2012 were much lower than in all other years
(Fig.~\ref{fig:s2-faceted}). Uncertainty in yearly estimates was highest in
the LaHave, Nashwaak, and Trinit\'{e}, and lowest in the 1SW-dominated
populations.

\begin{figure}[htbp] \centering
    \includegraphics[width=0.95\linewidth]{figures/pr-trends-faceted.png}
    \caption{Posterior estimates of \Pg for the seven rivers examined, error
        bars indicate the 90\% credible intervals.} \label{fig:pr-faceted}
\end{figure}

Population-level estimates of \prmu and \prsig varied considerably among rivers. For all
three 1SW-dominated rivers (Campbellton, Conne, and WAB), estimates of \prmu
were very close to 1.0 and had little variability in \prsig (Fig~\ref{fig:prmu-post}).
Estimates of \prmu were the lowest for the two QC rivers, particularly the Saint-Jean (median \prmu = 0.11).
Estimates of \prmu for the Nashwaak and the LaHave Rivers were close to 0.5, with these two rivers having 
the highest estimated values of \prsig, particularly the Nashwaak (Fig~\ref{fig:prmu-post}).

\begin{figure}[htbp] \centering
    \includegraphics[width=1.0\linewidth]{figures/pr-mu-posteriors.png}
    \caption{Posterior estimates of the population-level parameters $Pr_{\mu}$
       $logit(Pr_{\mu})$, and $logit(Pr_{\sigma})$. Dots denote median estimates, while the thick and thing error bars indicate
       the 50\% and 90\% credible intervals, respectively.} 
   \label{fig:prmu-post} 
\end{figure}

\subsubsection*{Correlations}

Most correlations between z-scored trends were not statistically significant, with only 
three out of the 21 pair-wise comparisons having a p-value below 0.01 (Fig.~\ref{fig:s1-corr}a).
When looking at the direction of the correlation, regardless whether they were
significant or not, these spanned both positive and negative coefficients
(Fig.~\ref{fig:s1-corr}b).

\begin{figure}[htbp] \centering
    \includegraphics[width=0.95\linewidth]{figures/corr-s1.png} \caption{
        Correlation among Z-scores of median estimated trends in \So among
        rivers. a) Pearson's correlation coefficients are shown in each square,
        while colouring denotes significance of the correlation ($p \leq 0.01$), b)
        colours denote direction and magnitude of correlation, while asterisks denote significance.}
\label{fig:s1-corr} 
\end{figure}

\section*{Discussion} 

% 1. Main findings
% 2. How they compare to other publications
% 3. Potential reasons
% 4. Caveats
% 5. Future directions

% 1. Main findings
Our results challenge the narrative that marine survival, specifically survival in the first year at sea, is declining
uniformly throughout the range of Atlantic salmon in the northwest Atlantic.
Temporal trends are not consistent among populations. 
Over the time periods for which data were available, some rivers show positive trends in survival in the
first winter at sea (\So) while other exhibit highly variable yet stable trends, and some show
declines. We could not assess trends in the second winter at sea (\St) or
proportion returning as grilse (\Pg), as these parameter estimates were highly
uncertain and were strongly influenced by the priors.
Perhaps there is a need to rethink our understanding of Atlantic salmon
population dynamics in light of the possibilities that (1) any real and persistent decline in marine survival was
experienced by some but not necessarily all populations, (2) reductions in survival might
have occurred over a relatively brief period of time and have not persisted, and (3) marine survival has
been relatively stable, or increasing in some populations for one or more decades.

% 2. How they compare to other publications
Our results are contrary to those of \citet{Olmos2019}, who detected positive
correlations in post-smolt survival among spatially broad stock units. These differences
could be due to a number of reasons: different model specifications and
structure, different methods for estimating covariance, difference in the
spatial scales of data sources (i.e. river vs province scales), and perhaps most importantly the use stock-recruitment relationships
rather than empirical smolt count data to estimate marine survival.
As trends in marine survival during the first winter at sea are highly independent
among rivers on relatively small spatial scale, trends from broader
geographical areas (i.e. province, state, or country-wide estimates) may not
be representative.
Interestingly, our estimates of \So and \St are very similar to those produced
by \citet{Chaput2003b}, and our trends are almost identical for the
overlapping time period that marine survival was estimated for in their study
(1984-1998). 
While \citet{Chaput2003b} separated abundance data for males and females
and assumed their survival rates were the same (to be able to reach an
analytical solution), our study reached almost the same results (albeit with
slightly higher uncertainty), using a Bayesian approach with informative
priors. These overlapping trends obtained with two different methods 
suggest that our method is effective at estimating marine survival.

Trends in marine survival
among populations were compared by \citet{Chaput2012a} using adult return rates.
He found that for 4 of 6 populations examined, return rates in the 1990s 
were lower than those during the 1970s.
\citet{Friedland1993} compared return rates for a number of rivers in eastern
North America between 1973 and 1988, and suggested there are similar trends among these. 
However, the similarity in these trends was driven primarily by two years, 1977 and 1978, which
show concurrent low and high relative return rates across rivers,
respectively. Other years are much more variable relative to each other.
\citeauthor{Friedland1993}'s \citeyear{Friedland1993} time series ends in  
1988; thus there are only a few years for which to assess overlap with the
time series in our study.
In any event, we caution that the pooling of adult return rates \citep{Chaput2012a, Friedland1993} 
can mask inter-annual variability in marine survival,
and hence might not produce an accurate depiction of marine survival trends.
\citet{Dempson2003} described a general declining trend in marine survival for
Newfoundland rivers (except WAB); we drew the same conclusion for 
Conne River but not Campbellton River or WAB. It is not possible to draw broader conclusions
with data from only three Newfoundland rivers, but it seems that among index rivers,
those in Newfoundland are among those with the highest marine survival rates.

% 3. Potential reasons
There are a variety of potential explanations for the lack of synchronous
trends in estimates of \So. 
Marine survival in the first winter at sea could be highly variable between
populations because of the predominance of spatially local environmental drivers of survival (e.g., temperature, predation) 
relative to broader-scale, even ocean-wide, drivers.
The synchrony reported for marine survival trends at broader spatial scales \citep{Olmos2019}
might be attributable to the use of stock-recruitment relationships to estimate survival,
relationships that may have been confounded by changes in recruitment dynamics.
There is some evidence of a correlation between return rate and growth (as
indicated by inter-circuli spacing on scales), where years of poor growth
tended to also be years of poor survival \citep{Friedland1993}, supporting the
idea that environmental variability can affect marine survival.
Furthermore, among European salmon, there is evidence of a positive correlation
between spring temperature in the Norwegian and North Seas and population abundance, suggesting warmer
conditions favour post-smolts \citep{Friedland1998}, based on mapping the
extent of area of suitable temperature (7-13 \textdegree C).

Nonetheless, the causal mechanisms for why warming should affect post-smolt
survival almost certainly differs depending on the difference between
temperature experienced by the post-smolts and their respective
population-specific thermal optima. 
This difference could explain why populations in eastern North America are
declining in the southern part of their range but potentially increasing
further north, and also why some studies find positive correlations between
temperature and abundance \citep{Friedland1998, Friedland1998b, Jonsson2004}
while others find negative ones \citep{Friedland1993, Todd2008}.
Putative associations between temperature and direct estimates of marine
survival warrants further study at the population level.

While there is little evidence that marine survival is density-dependent in
Atlantic salmon \citep{Jonsson1998,Gibson2006}, there could potentially be
some density-dependent processes during parts of the post-smolt migration
period, particularly for populations that are likely to be subjected to
declining per capita population growth rates ($r$) generated by Allee effects.
Exploring relationships between survival and population size could potentially
shed light about the processes that have caused many of the population
declines that have been documented.

Oceanic conditions have been correlated with abundance trends and growth
\citep{Todd2008}, however, the mechanism by which such bottom-up effects, \
mediated by changes in food availability,
affect population dynamics beyond marine survival needs to
be thoroughly reassessed. If marine survival on its own cannot fully explain
trends in abundance, then there are potential carry-on effects of oceanic
conditions that manifest with regards to fresh production. 
For example, adults
that return to spawn after spending suboptimal conditions at sea might be less
likely to make it to their spawning grounds, successfully secure a mate,
produce fewer eggs, or produce eggs with lower per capita fitness than those
produced by adults which grew in optimal oceanic conditions.
As larger females tend to be more productive, in terms of fecundity and total
reproductive energy, than the same weight's worth of smaller females
\citep{Barneche2018}, a small decrease in body condition resulting from bottom-up
impacts on food availability could potentially have disproportionate effects on fecundity
and fitness of the offspring.

Egg-to-smolt survival in Atlantic salmon is highly variable \citep{Klemetsen2003,Chaput2015}
and changes in the oceanic conditions that spawners experience could be
contributing to this variability.
Obviously, there would be a time lag (perhaps as much as a generation) in how such effects
might be manifest at the adult stage.
However, given that most correlations are between relatively monotonic declines
in abundance coupled also monotonic increases in climatic indices
over decadal time scales \citep[e.g.,][]{Friedland1998, Todd2008,
    Beaugrand2012}, it would be expected that this correlations would be
maintained even if salmon abundances were lagged by a generation length.


% 4. Caveats
As with all novel modelling approaches, there are caveats to acknowledge.
The seven populations explored in the present study might not be representative
of regional trends in marine survival. However, there are no other
long-term time series of smolts and adult returns to draw inferences from.
While there are analytical issues associated with the estimation of \So, \St, and \Pg,
the assumption that \St is additive to \So could produce unrealistic results.
We know there is a period of a few months where 1SW
returns are subject to a different environment than those salmon that will
return as 2SW the next year. 
While this is not ideal,
overcoming this assumption would require an additional parameter to be
estimated, or an additional assumption as to what proportion of \So is not
additive to \St (as the returning 1SW adults do not experience the same
environment when they return to their natal streams as those fish who stayed
at sea for an additional winter before returning to spawn).

Secondly, the assumed hierarchical structure might not be the most appropriate
for modelling smolt abundances. This approach results in shrinkage to the
mean, which means that the variability of yearly smolt estimates is less than
it would be if not modelled hierarchically. 
That said, this assumption is likely to result in more conservative trends in marine survival, as the
variation is smolt estimates is reduced.
It is important to note that the hierarchical structure in the estimation of
\Pg seems like a reasonable assumption given that the probability of returning
as grilse has a genetic component associated to it \citep{Aykanat2019} and is
not expected to vary much, within a population, among years.

% 5. Future directions
Perhaps a reframing of the issue of marine survival is key to furthering our
understanding of Atlantic salmon population dynamics. Marine survival may not
have declined consistently, and over the same time periods, across all
populations. 
But the fact that it has remained at roughly similar levels as it was
previously \emph{despite} reduced commercial fishing mortalities, suggests
that there may well be an interaction between small population size (small
relative to unfished population size or carrying capacity), recovery
potential, and environmental stochasticity that has not been fully explored in
Atlantic salmon. 
All else being equal, relatively small populations are more vulnerable to
demographic, environmental, and genetic stochasticity than large populations
\citep{Lande1993, Hutchings2015}. Interactions between population size and the
demographic consequences of environmental stochasticity appear to have
affected recovery in many marine fishes that have exhibited impaired recovery
since mitigation of the threat posed by fishing mortality
\citep{Hutchings2017, Hutchings2020}. The possibility that similar
interactions may be impairing the recovery of wild Atlantic salmon merits
study.

\section*{Acknowledgements}

% We would like to thank Geir Bolstad, G\'{e}rald Chaput, Brian Dempson, and
% Martha Robertson for their useful discussions on estimating marine survival
% in Atlantic salmon, and Amanda Kissel for her helpful comments on the
% manuscript. Brian Dempson and Geoff Venoit for providing the data Conne
% River data.
We would like to thank G\'{e}rald Chaput for his useful discussions on
estimating marine survival in Atlantic salmon, Carmen David for her comments
on the manuscript, and Sean Anderson for his help with implementing the
non-centered parameterization of the model. This research was supported by the
Atlantic Salmon Conservation Foundation and the Atlantic Salmon Research Joint
Venture.

\section*{Conflicts of Interest}

The authors declare no conflicts of interest.
 
\bibliography{subset}

%\documentclass[12pt]{article}
\usepackage[top=0.85in,left=1.0in,right=1.0in,footskip=0.75in]{geometry}
%\usepackage[parfill]{parskip}
\usepackage{setspace}
\usepackage{lineno}
\usepackage[hidelinks]{hyperref}
%\onehalfspacing
\doublespacing

\usepackage[round,sectionbib]{natbib}
\setcitestyle{authoryear}
\bibpunct{(}{)}{;}{a}{}{;}
\bibliographystyle{fishfishnourl}

%\usepackage{textcomp}
%\usepackage{libertine}
%\usepackage{inconsolata} % sans serif typewriter

\usepackage{mathtools} % for dcases
\usepackage{xcolor} % for textcolor
\usepackage{makecell} % for \makecell and within cell line breaks (\thead)

\usepackage[T1]{fontenc}

%\makeatletter\let\expandableinput\@@input\makeatother % expandable input for \input inside tables

% Linux Libertine:
\usepackage{textcomp}
\usepackage[sb]{libertine}
\usepackage[varqu,varl]{inconsolata}% sans serif typewriter
\usepackage[libertine,bigdelims,vvarbb]{newtxmath} % bb from STIX
\usepackage[vvarbb]{newtxmath} % bb from STIX, removed bigdelims for ScholarOne rendering
\usepackage[cal=boondoxo]{mathalfa} % mathcal
%\useosf % osf for text, not math
%\usepackage[supstfm=libertinesups,%
%  supscaled=1.2,%
%  raised=-.13em]{superiors}

\usepackage{xspace}
\usepackage{xfrac} % for diagonal inline fractions in text
%\usepackage{array} % for making whole row bold in table
\usepackage{colortbl} % for background colours in table rows
\usepackage{longtable}
\usepackage{amssymb} % for \checkmark 
\usepackage{amsmath} % for \checkmark 
\usepackage{rotating}
\usepackage[nolists,tablesfirst]{endfloat} % for putting figs and tables at end of document
\DeclareDelayedFloatFlavor{sidewaystable}{table}
\usepackage{makecell} % for \makecell in tables
\usepackage{doi}

%\usepackage{xr} % to obtain label references from supp materials file
%\externaldocument[S-]{suppmat}

\usepackage{tabularx}
\usepackage{booktabs}
\usepackage{array} % for table wrapping of columns
\newcolumntype{L}[1]{>{\raggedright\let\newline\\\arraybackslash\hspace{0pt}}m{#1}}
\newcolumntype{C}[1]{>{\centering\let\newline\\\arraybackslash\hspace{0pt}}m{#1}}
\newcolumntype{R}[1]{>{\raggedleft\let\newline\\\arraybackslash\hspace{0pt}}m{#1}}
%\usepackage[detect-all]{siunitx} % for SI units

% custom hyphenation:
\hyphenation{inverse}
\hyphenation{At-lantic}
\hyphenation{elasmo-branchs}


% Macros
\newcommand{\So}{$S_{1}$\xspace}
\newcommand{\St}{$S_{2}$\xspace}
\newcommand{\Pg}{$P_r$\xspace}
\newcommand{\prmu}{$\mu_r$\xspace}
\newcommand{\prsig}{$\sigma_r$\xspace}
\newcommand{\Linf}{$L_{\infty}$}
\newcommand{\DWinf}{$DW_{\infty}$}
\newcommand{\alphat}{$\tilde{\alpha}$}
\newcommand{\lamat}{$l_{\alpha_{mat}}$}
\newcommand{\lamatb}{$l_{\alpha_{mat}}b$}
\newcommand{\rmax}{$r_{max}$\xspace}
\newcommand{\ageratio}{$\alpha_{mat}/\alpha_{max}$}
\newcommand{\yr}{year\textsuperscript{-1}}
\newcommand{\rsq}{$R^2$\xspace}
\newcommand{\mytilde}{\raise.17ex\hbox{$\scriptstyle\mathtt{\sim}$}}
% Select what to do with command \comment:  
%\newcommand{\comment}[1]{}  % comment not shown
\newcommand{\comment}[1]{\par {\bfseries \color{blue} #1 \par}} % comment shown
%% END MACROS SECTION

\begin{document}
\linenumbers


\section*{Trends in marine survival of Atlantic salmon in eastern Canada}

\textbf{Sebasti\'{a}n A. Pardo\textsuperscript{1*}, 
        Geir H. Bolstad\textsuperscript{2}, 
        J. Brian Dempson\textsuperscript{3}, 
        Julien April\textsuperscript{4}, 
        Ross A. Jones\textsuperscript{5}, 
        Martha J. Robertson\textsuperscript{3}, 
        Dustin Raab\textsuperscript{6}, 
Jeffrey A. Hutchings\textsuperscript{1}} 

\noindent\small{\textsuperscript{1} Department of Biology, Dalhousie University, Halifax, NS, Canada\\}
\small{\textsuperscript{2} Norwegian Institute for Nature Research (NINA), Trondheim, Norway\\}
\small{\textsuperscript{3} Fisheries and Oceans Canada, St. John's, NL, Canada\\}
\small{\textsuperscript{4} Minist\`{e}re des For\^{e}ts, de la Faune et des Parcs, Qu\'{e}bec, QC, Canada\\}
\small{\textsuperscript{5} Fisheries and Oceans Canada, Moncton, NB, Canada\\}
\small{\textsuperscript{6} Fisheries and Oceans Canada, Dartmouth, NS, Canada\\}
\small{\textsuperscript{*} Corresponding author: spardo@dal.ca}

\section*{Abstract}

Declines in wild Atlantic salmon (\emph{Salmo salar}) throughout the north
Atlantic are primarily attributed to declining survival at sea. This
hypothesis has proven challenging to test on a river-by-river basis because of
the need to model data on both migrating smolts and returning adults and to
simultaneously estimate multiple parameters, especially for salmon spending
more than one winter at sea (1SW) before spawning. We fit a hierarchical Bayesian
maturity schedule model to data (19-42 years) for seven populations in  
Newfoundland and Labrador (NL), Qu\'{e}bec (QC), New Brunswick (NB), and Nova
Scotia (NS), Canada. We estimate survival in the first (\So) and second year at sea (\St),
and the proportion returning as 1SW adults (\Pg). Trends in  were not consistent
among rivers. Since 1990, \So increased at Western Arm Brook (NL)
and Rivi\`{e}re Saint-Jean (QC), but declined at Conne River (NL) and Rivière
de la Trinité (QC). Since the mid-1990s, \So increased at Campbellton River (NL),
declined at LaHave River (NS), and fluctuated at Nashwaak
River (NB). Estimates of \St were highly uncertain, particularly for
1SW-dominated populations; \Pg was generally stable. These results challenge the
narrative that marine survival has changed in a temporally consistent manner
among spatially disparate populations. Our findings suggest that, at the
population level, changes in abundance are attributable to temporal shifts in
multiple components of individual fitness, including at-sea survival. If
salmon populations do not respond in a consistent, uniform manner to changing
environmental conditions throughout their range, future research initiatives
should explore why.

% The marine phase of anadromous Atlantic salmon (\emph{Salmo salar}) is the
% least known yet one of the most crucial with regards to population
% persistence. Declines in many Atlantic salmon populations in eastern
% Canada have often been attributed to changes in conditions at sea, negatively
% affecting their survival. However, marine survival estimates are difficult to
% obtain given that many individuals spend multiple winters in the ocean before
% returning to freshwater to spawn, necessitating the estimation of multiple
% parameters. To do so, we fit a hierarchical Bayesian maturity schedule model
% to smolt and adult abundance time series for seven populations located in
% Newfoundland and Labrador (NL), Qu\'{e}bec (QC), New Brunswick (NB), and Nova
% Scotia (NS). The datasets ranged between 19 and 42 years in length. We
% estimated three components of marine survival: survival in the first year at
% sea (\So), survival in the second year at sea (\St), and proportion returning
% as one sea-winter adults (\Pg). Controlling for time frame, trends in
% estimates of \So were not consistent among rivers.
% In the four populations for which data extended to 1990, marine survival
% during the first year at sea predominantly increased over time in Western Arm
% Brook (NL) and Rivi\`{e}re Saint-Jean (QC) but largely declined in Conne River
% (NL) and Rivi\`{e}re Trinit\'{e} (QC). In the three other populations, \So
% exhibited an increase in Campbellton River (NL), a decline in LaHave River
% (NS), and fluctuating stability in Nashwaak River (NB) since approximately the
% mid-1990s. 
% Estimates of \St were highly uncertain, particularly for 1SW-dominated rivers
% (Conne, Campbellton and WAB, where the abundance of 2SW returns is very low)
% and thus the posterior distributions matched our choice of prior and trends
% could not be assessed. 
% \Pg was temporally stable within rivers, except for LaHave and Nashwaak Rivers. 
% Our findings challenge the assumption that marine survival has changed in a consistent
% manner among populations across a broad geographic scale, and suggest that
% trends can be river-specific. 
% The well established correlations between climate variables and abundance are
% not being mediated solely by marine survival, and there can potentially be
% some important indirect effects on fecundity.
% Further work exploring potential correlates of marine survival and the
% potential non-lethal effects of climate effects is warranted.


Keywords: salmonid, survival at sea, natural mortality, marine mortality

%Running Head: 

\section*{Introduction} % (4-5 paragraphs)

%1. Declining salmon numbers

Reductions in fishing mortality, albeit necessary, are not always sufficient
to facilitate population recovery. Experience with numerous commercially
exploited marine fisheries since the early 1990s has shown that not all
populations respond as favourably as anticipated to major reductions in
exploitation \citep{Hutchings2017}. Gradual efforts to close commercial
Atlantic salmon (\emph{Salmo salar}) fisheries in eastern Canada culminated in
full moratoria in all regions, beginning in the Maritime provinces (1984) and
following in Newfoundland (1992), Labrador (1998), and Qu\'{e}bec (2000). Since
these closures, many populations have not increased as 
expected \citep{Dempson2004, ICES2019}; some 
have been assessed as considered threatened or endangered by the 
Committee on the Status of Endangered Wildlife in Canada \citep[][]{Cosewic2010}, 
Canada's national science advisory body (to the national government) on
species risk of extinction.
While it is not fully understood what is driving population declines, the potential
drivers of these are many \citep[see ][for a detailed discussion of possible
causes]{Cairns2001}, including but not limited to: fishing mortality \citep{Dempson2004}, 
damming of waterways and changes in the freshwater habitat \citep{Dunfield1985}, acidification
\citep[particularly in the Southern Uplands region of
NS, see][]{Gibson2010}, predation by seals and birds \citep{Cairns2000}, negative
effects of interbreeding or interactions with escaped farmed salmon
\citep{Keyser2018}, and climate-driven changes in survival and productivity \citep{Mills2013}.

% 2.
Over the past three decades, a narrative has emerged that marine survival of
Atlantic salmon has declined throughout the North Atlantic \citep{ICES2019}.
%\citep{Hansen1998,OMaoileidigh2003,Chaput2012a}.
Based on multiple lines of evidence that climate conditions can directly and
indirectly influence the abundance and productivity of Atlantic salmon
populations \citep{Mills2013,Almodovar2019}, it has been presumed that oceanic climate effects are
driving population dynamics primarily through changes in marine survival.
An implicit assumption is that any trend in
survival in ocean habitat that is shared by multiple populations during their
seaward migration period will be experienced similarly. 
Put another way, given that salmon from different rivers 
are hypothesized to share marine habitat during some of their time at sea, it
has been presumed that populations share similar temporal trends in
at-sea mortality \citep{Friedland1993, Friedland1998, Russell2012}.  
A recent study by \citet{Olmos2019} suggested that trends in post-smolt
survival, when estimated at the stock level, are synchronously declining
for all Atlantic salmon in eastern North America.

% This perception is widespread both in the scientific
% literature as well as federal reports. The latest status report on Atlantic
% salmon by COSEWIC states that ``While the mechanism(s) of marine mortality is
% uncertain, what is clear is that the recent period of poor sea survival is
% occurring in parallel with many widespread changes in the North Atlantic
% ecosystem.'' \citep{Cosewic2010}. Consequently, there have several lines of
% inquiry as to what the causes behind these declines might be attributable to
% \citep{Friedland1993, Friedland1998}.


In contrast to the narrative of widespread, demographically similar increases
in at-sea mortality, the conservation status of Canadian salmon populations differs
considerably. Populations in the southern part of their range are more
likely to be assessed as being of conservation concern than those in more
northerly regions \citep{Cosewic2010}. This geographical disparity in status
suggests that if marine survival has been, or is, a key factor responsible for
most population declines, these changes are not uniformly distributed across
all populations. 

% Nonetheless, most of the populations declines assessed by COSEWIC have
% occurred in populations in the southern extent of its distribution:
% populations in the Bay of Fundy, Anticosti Island, and the Atlantic coast of
% Nova Scotia being assessed as Endangered, populations in the south coast of
% Newfoundland assessed as Threatened, And the populations in the New Brunswick
% and Qu\`{e}bec coasts of the Gulf of St. Lawrence assessed as Special Concern
% \citep{Cosewic2010}. On the other hand, northermost populations have shown
% stable, or even increasing population trends, suggesting that, if marine
% survival were to be a factor in many of these declines, these changes are not
% uniformly distributed across all populations.

% While the purported decline of marine survival is mentioned widely in the 
% scientific literature, only relatively few studies have quantified it in detail.
% The basis for this premise is a limited number of time series that exhibit
% temporal declines in a proxy of marine survival (i.e. return rates or
% post-smolt survival).

% 3. Nonetheless, this premise has never been examined in detail
Given the logistical challenges associated with estimating at-sea survival, it
is not surprising that the number of studies that have estimated temporal
trends has been limited. An additional limitation has been the derivation of
proxies (e.g., return rates), rather than direct model-based estimates, of
marine survival.
\citet{Chaput2012a}, for example, examined the return rate of smolts to adult salmon 
as a metric of marine survival, finding that most Canadian populations 
had experienced declining return rates. 
However, examination of trends in return rates alone
can mask changes in differential survival during different years at sea, as well
as changes in the proportion of adults returning after one or two years at sea.
Recently \citet{Olmos2019} suggested that trends in post-smolt
survival, when estimated at the stock unit level, are synchronously declining
for all Atlantic salmon in Eastern North America and Canada, a conclusion ultimately 
grounded on the veracity of highly variable stock-recruitment relationships.

%4. 
In the present study, we compile data on the number of migrating smolts and number of returning adults 
for seven wild Canadian populations of Atlantic salmon to model trends in marine survival.
While some studies have previously used maturity-schedule models to estimate marine
survival for a limited number of salmon populations \citep{Chaput2003b}, none
have incorporated data extending over multiple decades, nor have they examined
trends among more than two or three populations. 
Here, we develop a hierarchical Bayesian model that uses Murphy's maturity
schedule method, in conjunction with informative priors, to estimate yearly
marine survival in salmon. In addition to accounting for observation error in
smolt and return estimates, we estimate the proportion of salmon returning
after one winter hierarchically.

\section*{Methods}

\subsection*{Data}

We obtained time series data of outmigrating smolt and returning adult
abundances for seven Atlantic salmon populations in eastern Canada, encompassing a
wide range of the species' distribution (Fig.~\ref{fig:map}). 
Populations included the LaHave River in the Southern Uplands region of Nova
Scotia (NS), Nashwaak River, New Brunswick (NB), Rivi\`{e}re de la Trinit\'{e} (Trinit\'{e}) and
Saint-Jean rivers in Qu\'{e}bec (QC), and  Western Arm Brook (also referred to as  WAB), Campbellton, and
Conne River, Newfoundland (NL). 
Data were collected in NS, NB, and Newfoundland
and Labrador (NL) by Fisheries and Oceans Canada (DFO) and in QC
by the Minist\`{e}re des For\^{e}ts, de la Faune et des Parcs, Qu\'{e}bec.

\begin{figure}[htbp] \centering
    \includegraphics[width=0.85\linewidth]{figures/rivers-map2.png}
    \caption{Locations of the seven rivers in eastern Canada with time series abundance data of outmigrating smolts and 
    returning adults.} \label{fig:map} 
\end{figure}

\subsubsection*{Smolt and adult return abundance data}

Smolt and adult return abundance estimates originate from a variety of
sources. Smolt estimates from the Trinit\'{e}, Saint-Jean, and Conne populations were obtained using a
mark-recapture approach, while estimates from the ..., WAB, and Campbellton populations
were obtained by direct counts using fish counting fences.
For further details on the data collection methodologies refer to 
\citet{Dempson1991, Venoitt2018} for NL populations, 
\citet{April2018}  for QC populations,
\citet{Jones2014} for NB populations,
and \citet{Gibson2009} for NS populations. 

Yearly return data are often recorded in terms of two size groups: small ($< 63$ cm
FL) and large ($\geq 63$ cm FL) salmon, as these closely represent different
life-history strategies (i.e. 1SW and 2SW), but can be confounded with repeat
spawners of different sizes. To correct for this in returns 
reported as small and large salmon, we estimated the abundance
of 1SW and 2SW returns using yearly scale age data of a subsample of returns:

\begin{equation}
    p_{r,t,a} = \frac{\sum_{s}{(\frac{n_{r,t,s,a}}{n_{r,t,s}} * N_{r,y,s})}}{\sum_{s}{N_{r,t,s}}}
\end{equation}

where $p_{r,t,a}$ is the proportion of annual returns in river $r$, year $t$,
and of spawning history $a$ (either 1SW or 2SW returns); $n_{r,t,s,a}$ is the
number of samples in river $r$, year $t$, of spawning history $a$, and of size
group $s$; $n_{r,t,s}$ is the total number of samples in river $r$, year $t$,
and of size group $s$; and $N_{r,t,s}$ is the returns of salmon
in river $r$, year $t$, and of size group $s$.

\subsection*{Bayesian model}

We developed a hierarchical Bayesian model that uses Murphy's maturity
schedule method, in conjunction with informative priors, to estimate yearly
marine survival in seven populations of Atlantic salmon. We account for
observation error in smolt and return estimates, as well as estimating the
proportion returning after one winter (i.e. \Pg) hierarchically.
There is an identifiability problem in the maturity schedule equations where
the parameter estimates cannot be optimally solved \citep{Chaput2003a}.
However, this issue can be mathematically overcome, at least partially, by
using informative priors for all three marine survival parameters in a
Bayesian framework.
This method requires abundance estimates of smolts as well as abundance estimates
of returning one-sea-winter (1SW) and two-sea-winter adults. With these data,
it estimates three parameters: survival in the first year at sea (\So), survival
in the second year at sea (\St), and the proportion of fish returning after one
year at sea (\Pg). 

Our model does not include repeat spawners and assumes that no fish spend
three or more winters at sea before returning to spawn for the first time.
The model also assumes that mortality in the second winter at sea (\St)
is additional to mortality in the first winter at sea in the previous year, 
and therefore does not account for differences in environmental conditions experienced
between 1SW and 2SW fish of the same smolt cohort during their overlapping first year at sea.
In other words, our model assumes that the decision of returning occurs just before
actually being counted as returns and that \St is any additional mortality in
the subsequent year. 

%\comment{Other assumptions...}

Observed smolt estimates were modelled hierarchically and included
observation error:

\begin{equation}
log(smolts_{obs,t,r}) = log(smolts_{true,t,r}) + \epsilon_{t,r}
\end{equation}

where $smolts_{true,t,r}$ are the true smolt abundances for year $t$ and river
$r$, and $\epsilon_{t,r}$ is the error term, which is empirically derived from
by calculating the yearly coefficient of variation in the empirically derived
smolt estimates (see Table S1 in the Supplementary material). 
Where available, we used population-specific measurement error estimates for smolt abundances; if not 
available, we set measurement error at 5\%. 
The log-transformed true smolt abundances are
normally distributed around a population-level mean and standard deviation:

\begin{equation}
log(smolts_{true,t,r}) \sim Normal(\mu_{smolts,r}, \sigma_{smolts,r})
\end{equation}

where $\mu_{smolts,r}$ and $\sigma_{smolts,r}$ are parameters estimated by the
model for each river.

Once we have yearly estimates of smolt, 1SW, and 2SW abundances, we estimate
marine survival parameters using Murphy's maturity schedule method
\citep{Murphy1952, Ricker1975}:

\begin{align}
    R_{1,t} &= smolts_{true,t-1} * S_{1,t} * Pr_t \label{eq:1}, \\
    R_{2,t+1} &= smolts_{true,t-1} * S_{1,t} * (1 - Pr_t) * S_{2,t+1} \label{eq:2}
\end{align}

where $R_{1,t}$ and $R_{2,t+1}$ are the estimated abundances of 1SW and 2SW
salmon returning in years $t$ and $t+1$, respectively, $smolts_{true,t-1}$ is the
estimated number of outmigrating smolts in year $t-1$, $S_{1,t}$ is the proportion of
salmon surviving in their first year ($t$) at sea, $Pr_t$ is the proportion of
salmon that return to spawn at year $t$, $S_{2,t+1}$ is the survival in their
second year at sea of the same cohort of salmon who did not return to spawn at
year $t$.

However, to allow for normally and log-normally-distributed parameters we
log-transform equations~\ref{eq:1} and~\ref{eq:2} to obtain

\begin{align}
    log(R_{1,r,t}) &= log(smolts_{r,t-1}) + log(Pr_{r,t}) - Z_{1,r,t} \label{eq:3}, \\
    log(R_{2,r,t+1}) &= log(smolts_{r,t-1}) - Z_{1,r,t} + log(1 - Pr_{r,t})  - Z_{2,r,t+1} \label{eq:4} 
\end{align}

where $Z_{1,r,t}$ and $Z_{2,r,t+1}$ are the instantaneous mortality rates.
Process error was included as the standard deviation of the log-transformed
return estimates from equation~\ref{eq:4}:

\begin{align}
log(R_{obs,1,t,r}) &\sim Normal(log(R_{1,t,r}), \epsilon_{1,r}), \\
log(R_{obs,2,t,r}) &\sim Normal(log(R_{2,t,r}), \epsilon_{2,r}) \label{eq:5} 
\end{align}

where $R_{obs,1,t,r}$ and $R_{obs,2,t,r}$ are the observed return estimates
for year $t$ and river $r$ of 1SW and 2SW fish, respectively, $\epsilon_{1,r}$
and $\epsilon_{2,r}$ are the process error terms. 
These error terms are normally distributed with a standard deviation
of 0.01, which are almost equivalent to using a coefficient of variation in
return estimates 1\%, given that return estimates in equations~\ref{eq:3}
and~\ref{eq:4} are log-transformed:

\begin{align}
\epsilon_{1,r} &\sim Normal(0, 0.01) \\
\epsilon_{2,r} &\sim Normal(0, 0.01).
\end{align}

Furthermore, we use instantaneous mortality rates in the model instead of survival probabilities
as the model is more efficient in its parameter search in log-space, and instantaneous rates
are easy to interpret. Instantaneous rates are easily converted to survival probabilities by 

\begin{align}
 S_{1} &= e^{-Z_1}, \\
 S_{2} &= e^{-Z_2}. 
\end{align}

We estimate population-level mean \Pg values around which the yearly \Pg
values are normally distributed. We specify different informative hyperpriors
for \prmu and \prsig based on whether the population is 1SW-dominated or not:

\begin{align}
    logit(P_{r,t}) &\sim Normal(\mu_r, \sigma_r) \\
    \mu_r &\sim 
    \begin{cases}
       Normal(2.3, 0.4),  &\text{for 1SW-dominated populations} \\
       Normal(0, 2.8), &\text{for non-1SW-dominated populations} \\
   \end{cases} \\
    \sigma_r &\sim halfNormal(0, 1).
\end{align}

The priors for $Z_1$ and $Z_2$ are specified as log-normal distributions:

\begin{align}
Z_1 &\sim logNormal(1, 0.22),   \\ 
Z_2 &\sim logNormal(0.2, 0.3).
\end{align}

The model was written in Stan \citep{Carpenter2017} and run in R version 3.6.1
\citep{RCoreTeam2019} using the \texttt{rstan} package version 2.19.2
\citep{StanDevelopmentTeam2019}.
The model was run with three chains and 3,000 iterations, with the first 1,500
discarded as a burn-in. The models were considered to have converged when the
$\hat R$ of all parameters were lower than 1.03 and the effective sample size 
were higher than 500.

\subsection*{Correlations among trends in survival}

%\comment{How do were quantitatively compare trends among rivers? DFA is an
%option, but so is a t-test of averages before and after a certain year}

We looked at the correlation of trends in \So by calculating the Pearson's
correlation coefficient for the Z-scores of these trends. Given that the time
series do not cover the same years, and that some rivers have missing years in
the middle of the time series, pair-wise Pearson's tests were done using only
the years where there is data for both rivers.

\section*{Results}

%\subsubsection*{Model convergence}

\subsubsection*{Trends in marine survival parameters}

Trends in estimates of \So were highly variable within and among rivers
(Fig.~\ref{fig:s1-dual}). The highest median posterior estimates of \So
were for the Nashwaak River in 2006 and 2008, with values of 0.18 and 0.21,
respectively. The lowest median \So estimate was in the Trinit\'{e} in 2001,
with an estimate of \So of 0.007, while the Conne, LaHave, and Trinit\'{e} had
years where estimates of \So varied between 0.01 and 0.02 (Fig.~\ref{fig:s1-faceted}).

Trends among populations also varied: Campbellton,
Saint-Jean, and Western Arm Brook populations showed increases in \So
over time, Trinit\'{e}, Conne, and La Have populations showed decreases,
while at Nashwaak there was an increase in median \So during the early
2000s but a decrease in the 2010s. Yearly estimates of \So had very little
variability for one sea-winter dominated populations (Conne, Campbellton, and
WAB), but were more variable (i.e. wider credible intervals) in the other
populations.

\begin{figure}[htbp] \centering
    \includegraphics[width=0.95\linewidth]{figures/s1-trends-dual.png}
    \caption{Trends in survival in the first year at sea (\So). a) Posterior
        estimates of \So for the seven rivers examined, error bars indicate
        the 90\% credible intervals, and dashed lines denote years of commercial
        fishing moratoria for each province. b) Z-scores of median posterior
        estimates.} \label{fig:s1-dual} \end{figure}

\begin{figure}[htbp] \centering
    \includegraphics[width=0.95\linewidth]{figures/s1-trends-faceted.png}
    \caption{Posterior estimates of \So for the seven rivers examined, error
        bars indicate the 90\% credible intervals.} \label{fig:s1-faceted}
\end{figure}

Estimates of \St were highly uncertain in all rivers, and trends 
were not apparent in most rivers given the large range of the credible
intervals in the yearly estimates (Fig.~\ref{fig:s2-faceted}). The estimates
of \St for the Saint-Jean and Trinit\'{e} were considerably less uncertain
than for the other populations, but showed no apparent trends through time.

\begin{figure}[htbp] \centering
    \includegraphics[width=0.95\linewidth]{figures/s2-trends-faceted.png}
    \caption{Posterior estimates of \St for the seven rivers examined, error
        bars indicate the 90\% credible intervals.} \label{fig:s2-faceted}
\end{figure}

Estimates of \Pg were mostly stable across time, except for the LaHave and
Nashwaak Rivers; The estimates of \Pg were slightly lower in the last four
years than in the previous ones, while in the Nashwaak the posterior estimates
of \Pg in 2012 were much lower than in all other years
(Fig.~\ref{fig:s2-faceted}). Uncertainty in yearly estimates was highest in
the LaHave, Nashwaak, and Trinit\'{e}, and lowest in the 1SW-dominated
populations.

\begin{figure}[htbp] \centering
    \includegraphics[width=0.95\linewidth]{figures/pr-trends-faceted.png}
    \caption{Posterior estimates of \Pg for the seven rivers examined, error
        bars indicate the 90\% credible intervals.} \label{fig:pr-faceted}
\end{figure}

Population-level estimates of \prmu and \prsig varied considerably among rivers. For all
three 1SW-dominated rivers (Campbellton, Conne, and WAB), estimates of \prmu
were very close to 1.0 and had little variability in \prsig (Fig~\ref{fig:prmu-post}).
Estimates of \prmu were the lowest for the two QC rivers, particularly the Saint-Jean (median \prmu = 0.11).
Estimates of \prmu for the Nashwaak and the LaHave Rivers were close to 0.5, with these two rivers having 
the highest estimated values of \prsig, particularly the Nashwaak (Fig~\ref{fig:prmu-post}).

\begin{figure}[htbp] \centering
    \includegraphics[width=1.0\linewidth]{figures/pr-mu-posteriors.png}
    \caption{Posterior estimates of the population-level parameters $Pr_{\mu}$
       $logit(Pr_{\mu})$, and $logit(Pr_{\sigma})$. Dots denote median estimates, while the thick and thing error bars indicate
       the 50\% and 90\% credible intervals, respectively.} 
   \label{fig:prmu-post} 
\end{figure}

\subsubsection*{Correlations}

Most correlations between z-scored trends were not statistically significant, with only 
three out of the 21 pair-wise comparisons having a p-value below 0.01 (Fig.~\ref{fig:s1-corr}a).
When looking at the direction of the correlation, regardless whether they were
significant or not, these spanned both positive and negative coefficients
(Fig.~\ref{fig:s1-corr}b).

\begin{figure}[htbp] \centering
    \includegraphics[width=0.95\linewidth]{figures/corr-s1.png} \caption{
        Correlation among Z-scores of median estimated trends in \So among
        rivers. a) Pearson's correlation coefficients are shown in each square,
        while colouring denotes significance of the correlation ($p \leq 0.01$), b)
        colours denote direction and magnitude of correlation, while asterisks denote significance.}
\label{fig:s1-corr} 
\end{figure}

\section*{Discussion} 

% 1. Main findings
% 2. How they compare to other publications
% 3. Potential reasons
% 4. Caveats
% 5. Future directions

% 1. Main findings
Our results challenge the narrative that marine survival, specifically survival in the first year at sea, is declining
uniformly throughout the range of Atlantic salmon in the northwest Atlantic.
Temporal trends are not consistent among populations. 
Over the time periods for which data were available, some rivers show positive trends in survival in the
first winter at sea (\So) while other exhibit highly variable yet stable trends, and some show
declines. We could not assess trends in the second winter at sea (\St) or
proportion returning as grilse (\Pg), as these parameter estimates were highly
uncertain and were strongly influenced by the priors.
Perhaps there is a need to rethink our understanding of Atlantic salmon
population dynamics in light of the possibilities that (1) any real and persistent decline in marine survival was
experienced by some but not necessarily all populations, (2) reductions in survival might
have occurred over a relatively brief period of time and have not persisted, and (3) marine survival has
been relatively stable, or increasing in some populations for one or more decades.

% 2. How they compare to other publications
Our results are contrary to those of \citet{Olmos2019}, who detected positive
correlations in post-smolt survival among spatially broad stock units. These differences
could be due to a number of reasons: different model specifications and
structure, different methods for estimating covariance, difference in the
spatial scales of data sources (i.e. river vs province scales), and perhaps most importantly the use stock-recruitment relationships
rather than empirical smolt count data to estimate marine survival.
As trends in marine survival during the first winter at sea are highly independent
among rivers on relatively small spatial scale, trends from broader
geographical areas (i.e. province, state, or country-wide estimates) may not
be representative.
Interestingly, our estimates of \So and \St are very similar to those produced
by \citet{Chaput2003b}, and our trends are almost identical for the
overlapping time period that marine survival was estimated for in their study
(1984-1998). 
While \citet{Chaput2003b} separated abundance data for males and females
and assumed their survival rates were the same (to be able to reach an
analytical solution), our study reached almost the same results (albeit with
slightly higher uncertainty), using a Bayesian approach with informative
priors. These overlapping trends obtained with two different methods 
suggest that our method is effective at estimating marine survival.

Trends in marine survival
among populations were compared by \citet{Chaput2012a} using adult return rates.
He found that for 4 of 6 populations examined, return rates in the 1990s 
were lower than those during the 1970s.
\citet{Friedland1993} compared return rates for a number of rivers in eastern
North America between 1973 and 1988, and suggested there are similar trends among these. 
However, the similarity in these trends was driven primarily by two years, 1977 and 1978, which
show concurrent low and high relative return rates across rivers,
respectively. Other years are much more variable relative to each other.
\citeauthor{Friedland1993}'s \citeyear{Friedland1993} time series ends in  
1988; thus there are only a few years for which to assess overlap with the
time series in our study.
In any event, we caution that the pooling of adult return rates \citep{Chaput2012a, Friedland1993} 
can mask inter-annual variability in marine survival,
and hence might not produce an accurate depiction of marine survival trends.
\citet{Dempson2003} described a general declining trend in marine survival for
Newfoundland rivers (except WAB); we drew the same conclusion for 
Conne River but not Campbellton River or WAB. It is not possible to draw broader conclusions
with data from only three Newfoundland rivers, but it seems that among index rivers,
those in Newfoundland are among those with the highest marine survival rates.

% 3. Potential reasons
There are a variety of potential explanations for the lack of synchronous
trends in estimates of \So. 
Marine survival in the first winter at sea could be highly variable between
populations because of the predominance of spatially local environmental drivers of survival (e.g., temperature, predation) 
relative to broader-scale, even ocean-wide, drivers.
The synchrony reported for marine survival trends at broader spatial scales \citep{Olmos2019}
might be attributable to the use of stock-recruitment relationships to estimate survival,
relationships that may have been confounded by changes in recruitment dynamics.
There is some evidence of a correlation between return rate and growth (as
indicated by inter-circuli spacing on scales), where years of poor growth
tended to also be years of poor survival \citep{Friedland1993}, supporting the
idea that environmental variability can affect marine survival.
Furthermore, among European salmon, there is evidence of a positive correlation
between spring temperature in the Norwegian and North Seas and population abundance, suggesting warmer
conditions favour post-smolts \citep{Friedland1998}, based on mapping the
extent of area of suitable temperature (7-13 \textdegree C).

Nonetheless, the causal mechanisms for why warming should affect post-smolt
survival almost certainly differs depending on the difference between
temperature experienced by the post-smolts and their respective
population-specific thermal optima. 
This difference could explain why populations in eastern North America are
declining in the southern part of their range but potentially increasing
further north, and also why some studies find positive correlations between
temperature and abundance \citep{Friedland1998, Friedland1998b, Jonsson2004}
while others find negative ones \citep{Friedland1993, Todd2008}.
Putative associations between temperature and direct estimates of marine
survival warrants further study at the population level.

While there is little evidence that marine survival is density-dependent in
Atlantic salmon \citep{Jonsson1998,Gibson2006}, there could potentially be
some density-dependent processes during parts of the post-smolt migration
period, particularly for populations that are likely to be subjected to
declining per capita population growth rates ($r$) generated by Allee effects.
Exploring relationships between survival and population size could potentially
shed light about the processes that have caused many of the population
declines that have been documented.

Oceanic conditions have been correlated with abundance trends and growth
\citep{Todd2008}, however, the mechanism by which such bottom-up effects, \
mediated by changes in food availability,
affect population dynamics beyond marine survival needs to
be thoroughly reassessed. If marine survival on its own cannot fully explain
trends in abundance, then there are potential carry-on effects of oceanic
conditions that manifest with regards to fresh production. 
For example, adults
that return to spawn after spending suboptimal conditions at sea might be less
likely to make it to their spawning grounds, successfully secure a mate,
produce fewer eggs, or produce eggs with lower per capita fitness than those
produced by adults which grew in optimal oceanic conditions.
As larger females tend to be more productive, in terms of fecundity and total
reproductive energy, than the same weight's worth of smaller females
\citep{Barneche2018}, a small decrease in body condition resulting from bottom-up
impacts on food availability could potentially have disproportionate effects on fecundity
and fitness of the offspring.

Egg-to-smolt survival in Atlantic salmon is highly variable \citep{Klemetsen2003,Chaput2015}
and changes in the oceanic conditions that spawners experience could be
contributing to this variability.
Obviously, there would be a time lag (perhaps as much as a generation) in how such effects
might be manifest at the adult stage.
However, given that most correlations are between relatively monotonic declines
in abundance coupled also monotonic increases in climatic indices
over decadal time scales \citep[e.g.,][]{Friedland1998, Todd2008,
    Beaugrand2012}, it would be expected that this correlations would be
maintained even if salmon abundances were lagged by a generation length.


% 4. Caveats
As with all novel modelling approaches, there are caveats to acknowledge.
The seven populations explored in the present study might not be representative
of regional trends in marine survival. However, there are no other
long-term time series of smolts and adult returns to draw inferences from.
While there are analytical issues associated with the estimation of \So, \St, and \Pg,
the assumption that \St is additive to \So could produce unrealistic results.
We know there is a period of a few months where 1SW
returns are subject to a different environment than those salmon that will
return as 2SW the next year. 
While this is not ideal,
overcoming this assumption would require an additional parameter to be
estimated, or an additional assumption as to what proportion of \So is not
additive to \St (as the returning 1SW adults do not experience the same
environment when they return to their natal streams as those fish who stayed
at sea for an additional winter before returning to spawn).

Secondly, the assumed hierarchical structure might not be the most appropriate
for modelling smolt abundances. This approach results in shrinkage to the
mean, which means that the variability of yearly smolt estimates is less than
it would be if not modelled hierarchically. 
That said, this assumption is likely to result in more conservative trends in marine survival, as the
variation is smolt estimates is reduced.
It is important to note that the hierarchical structure in the estimation of
\Pg seems like a reasonable assumption given that the probability of returning
as grilse has a genetic component associated to it \citep{Aykanat2019} and is
not expected to vary much, within a population, among years.

% 5. Future directions
Perhaps a reframing of the issue of marine survival is key to furthering our
understanding of Atlantic salmon population dynamics. Marine survival may not
have declined consistently, and over the same time periods, across all
populations. 
But the fact that it has remained at roughly similar levels as it was
previously \emph{despite} reduced commercial fishing mortalities, suggests
that there may well be an interaction between small population size (small
relative to unfished population size or carrying capacity), recovery
potential, and environmental stochasticity that has not been fully explored in
Atlantic salmon. 
All else being equal, relatively small populations are more vulnerable to
demographic, environmental, and genetic stochasticity than large populations
\citep{Lande1993, Hutchings2015}. Interactions between population size and the
demographic consequences of environmental stochasticity appear to have
affected recovery in many marine fishes that have exhibited impaired recovery
since mitigation of the threat posed by fishing mortality
\citep{Hutchings2017, Hutchings2020}. The possibility that similar
interactions may be impairing the recovery of wild Atlantic salmon merits
study.

\section*{Acknowledgements}

% We would like to thank Geir Bolstad, G\'{e}rald Chaput, Brian Dempson, and
% Martha Robertson for their useful discussions on estimating marine survival
% in Atlantic salmon, and Amanda Kissel for her helpful comments on the
% manuscript. Brian Dempson and Geoff Venoit for providing the data Conne
% River data.
We would like to thank G\'{e}rald Chaput for his useful discussions on
estimating marine survival in Atlantic salmon, Carmen David for her comments
on the manuscript, and Sean Anderson for his help with implementing the
non-centered parameterization of the model. This research was supported by the
Atlantic Salmon Conservation Foundation and the Atlantic Salmon Research Joint
Venture.

\section*{Conflicts of Interest}

The authors declare no conflicts of interest.
 
\bibliography{subset}

%\input{ms.bbl}

\end{document}




\end{document}




\end{document}




\end{document}


