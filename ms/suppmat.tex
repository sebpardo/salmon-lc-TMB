\documentclass[12pt]{article}
\usepackage[top=0.85in,left=1.0in,right=1.0in,footskip=0.75in]{geometry}
\usepackage[parfill]{parskip}
\usepackage{setspace}
\usepackage{lineno}
\usepackage[hidelinks]{hyperref}
\onehalfspacing

\usepackage[round]{natbib}
\setcitestyle{authoryear}
\bibpunct{(}{)}{;}{a}{}{;}

% Linux Libertine:
\usepackage{textcomp}
\usepackage[sb]{libertine}
\usepackage[varqu,varl]{inconsolata}% sans serif typewriter
\usepackage[libertine,bigdelims,vvarbb]{newtxmath} % bb from STIX
\usepackage[cal=boondoxo]{mathalfa} % mathcal
\useosf % osf for text, not math
\usepackage[supstfm=libertinesups,%
  supscaled=1.2,%
  raised=-.13em]{superiors}

\usepackage[utf8]{inputenc}

\usepackage{xspace}
\usepackage{xfrac} % for diagonal inline fractions in text
%\usepackage{array} % for making whole row bold in table
\usepackage{colortbl} % for background colours in table rows
\usepackage{longtable}
\usepackage{amssymb} % for \checkmark 
\usepackage{rotating}
%\usepackage[nolists,tablesfirst]{endfloat}
%\DeclareDelayedFloatFlavor{sidewaystable}{table}
\usepackage{makecell} % for \makecell in tables

% omit bibliography (\nobibliography{...})
\usepackage{bibentry}

\usepackage{tabularx}
\usepackage{booktabs}
\usepackage{array} % for table wrapping of columns
\newcolumntype{L}[1]{>{\raggedright\let\newline\\\arraybackslash\hspace{0pt}}m{#1}}
\newcolumntype{C}[1]{>{\centering\let\newline\\\arraybackslash\hspace{0pt}}m{#1}}
\newcolumntype{R}[1]{>{\raggedleft\let\newline\\\arraybackslash\hspace{0pt}}m{#1}}
\usepackage[detect-all]{siunitx} % for SI units

\makeatletter\let\expandableinput\@@input\makeatother % expandable input for \input inside tables
\setlength{\LTcapwidth}{\textwidth}

% custom hyphenation:
\hyphenation{elasmo-branchs}

% Macros
\newcommand{\So}{$S_{1}$\xspace}
\newcommand{\St}{$S_{2}$\xspace}
\newcommand{\Pg}{$P_r$\xspace}
\newcommand{\Linf}{$L_{\infty}$}
\newcommand{\DWinf}{$DW_{\infty}$}
\newcommand{\alphat}{$\tilde{\alpha}$}
\newcommand{\lamat}{$l_{\alpha_{mat}}$}
\newcommand{\lamatb}{$l_{\alpha_{mat}}b$}
\newcommand{\rmax}{$r_{max}$\xspace}
\newcommand{\ageratio}{$\alpha_{mat}/\alpha_{max}$}
\newcommand{\yr}{year\textsuperscript{-1}}
\newcommand{\rsq}{$R^2$\xspace}
% Select what to do with command \comment:  
\newcommand{\comment}[1]{}  % comment not shown
%\newcommand{\comment}[1]{\par {\bfseries \color{blue} #1 \par}} % comment shown
%% END MACROS SECTION

%\bibliographystyle{model2-names}
\bibliographystyle{fishfishnourl}

%%% MY MACROS

% Make supplementary tables and figures start with S1 and T2
% use \beginsupplement to reset count
\newcommand{\beginsupplement}{%
        \setcounter{table}{0}
        \renewcommand{\thetable}{S\arabic{table}}%
        \setcounter{figure}{0}
        \renewcommand{\thefigure}{S\arabic{figure}}%
     }

\begin{document}
\linenumbers


\noindent
{\Large Supplementary Materials}
\newline

\noindent
{\Large for}
\newline

\noindent
{\large Trends in marine survival of Atlantic salmon populations in eastern Canada}
\newline

\noindent
Sebasti\'{a}n A. Pardo\textsuperscript{1*}, Geir H. Bolstad\textsuperscript{2}, J. Brian Dempson\textsuperscript{3}, 
        Julien April\textsuperscript{4}, Ross A. Jones\textsuperscript{5}, %Martha J. Robertson\textsuperscript{3}, 
        Dustin Raab\textsuperscript{6}, Jeffrey A. Hutchings\textsuperscript{1}
\newline

\noindent
\small{\textsuperscript{1} Department of Biology, Dalhousie University, Halifax, NS, Canada\\}
\small{\textsuperscript{2} Norwegian Institute for Nature Research (NINA), Trondheim, Norway\\}
\small{\textsuperscript{3} Fisheries and Oceans Canada, St. John's, NL, Canada\\}
\small{\textsuperscript{4} Minist\`{e}re des For\^{e}ts, de la Faune et des Parcs, Qu\'{e}bec, QC, Canada\\}
\small{\textsuperscript{5} Fisheries and Oceans Canada, Moncton, NB, Canada\\}
\small{\textsuperscript{6} Fisheries and Oceans Canada, Dartmouth, NS, Canada\\}
\small{\textsuperscript{*} Corresponding author: spardo@dal.ca}
\beginsupplement

This Supplementary Material includes additional tables to accompany the publication. 

\clearpage

\subsection*{Raw smolt data used in the model}

% latex table generated in R 3.6.3 by xtable 1.8-4 package
% Tue May 12 20:59:00 2020
\begingroup\footnotesize
\begin{longtable}{lrrrrrrr}
\caption{Empirical annual smolt estimates for the seven populations examined.} \\ 
  \hline
Year & Campbellton & Conne & LaHave & Nashwaak & Saint-Jean & Trinité & WAB \\ 
  \hline
1974 & - & - & - & - & - & - & 8,484 \\ 
  1975 & - & - & - & - & - & - & 11,854 \\ 
  1976 & - & - & - & - & - & - & 9,600 \\ 
  1977 & - & - & - & - & - & - & 6,232 \\ 
  1978 & - & - & - & - & - & - & 9,899 \\ 
  1979 & - & - & - & - & - & - & 13,071 \\ 
  1980 & - & - & - & - & - & - & 8,349 \\ 
  1981 & - & - & - & - & - & - & 15,665 \\ 
  1982 & - & - & - & - & - & - & 13,981 \\ 
  1983 & - & - & - & - & - & - & 12,477 \\ 
  1984 & - & - & - & - & - & - & 10,552 \\ 
  1985 & - & - & - & - & - & 68,208 & 20,653 \\ 
  1986 & - & - & - & - & - & 66,069 & 13,417 \\ 
  1987 & - & - & - & - & - & 96,545 & 17,719 \\ 
  1988 & - & 74,585 & - & - & - & 77,617 & 17,029 \\ 
  1989 & - & 65,692 & - & - & - & 51,879 & 15,321 \\ 
  1990 & - & 73,724 & - & - & 92,665 & 80,057 & 11,407 \\ 
  1991 & - & 56,943 & - & - & 97,992 & 50,328 & 10,563 \\ 
  1992 & - & 74,645 & - & - & 113,927 & 40,863 & 13,453 \\ 
  1993 & - & 68,208 & - & - & 154,980 & 50,869 & 15,405 \\ 
  1994 & 31,577 & 55,765 & - & - & 142,972 & 86,226 & 13,435 \\ 
  1995 & 41,663 & 60,762 & - & - & 74,285 & 55,913 & 9,283 \\ 
  1996 & 39,715 & 62,749 & - & - & 60,227 & 71,899 & 15,144 \\ 
  1997 & 58,369 & 94,088 & 20,511 & - & 104,973 & 61,092 & 14,502 \\ 
  1998 & 62,050 & 100,983 & 16,550 & - & - & 31,892 & 23,845 \\ 
  1999 & 50,441 & 69,841 & 15,600 & 22,750 & 95,843 & 28,962 & 17,139 \\ 
  2000 & 47,256 & 63,658 & 10,420 & 28,500 & 114,255 & 56,557 & 13,500 \\ 
  2001 & 35,596 & 60,777 & 16,300 & 15,800 & 50,993 & 39,744 & 12,706 \\ 
  2002 & 37,170 & 86,898 & 15,700 & 11,000 & 109,845 & 70,318 & 16,013 \\ 
  2003 & 32,630 & 81,806 & 11,860 & 15,000 & 71,839 & 44,264 & 14,999 \\ 
  2004 & 35,089 & 71,479 & 17,845 & 9,000 & 60,259 & 53,030 & 12,086 \\ 
  2005 & 32,780 & 79,667 & 20,613 & 13,600 & 54,821 & 27,051 & 17,323 \\ 
  2006 & 30,123 & 66,196 & 5,270 & 5,200 & 96,002 & 34,867 & 8,607 \\ 
  2007 & 33,304 & 35,146 & 22,971 & 25,400 & 102,939 & - & 20,826 \\ 
  2008 & 35,742 & 63,738 & 24,430 & 21,550 & 135,360 & 42,923 & 16,621 \\ 
  2009 & 40,390 & 68,242 & 14,450 & 7,300 & 45,978 & 35,036 & 17,444 \\ 
  2010 & 36,705 & 71,085 & 8,644 & 15,900 & 37,297 & 32,680 & 18,492 \\ 
  2011 & 41,069 & 54,392 & 16,215 & 12,500 & 48,187 & 37,500 & 19,044 \\ 
  2012 & 37,033 & 50,701 & - & 8,750 & 45,214 & 44,400 & 20,544 \\ 
  2013 & 44,193 & 51,220 & - & 11,060 & 40,787 & 45,108 & 13,573 \\ 
  2014 & 40,355 & 66,261 & 7,159 & 10,120 & 36,849 & 42,378 & 19,710 \\ 
  2015 & 45,630 & 56,224 & 29,175 & 11,100 & 56,456 & 30,741 & 19,771 \\ 
  2016 & - & - & 6,664 & 7,900 & - & 47,566 & - \\ 
  2017 & - & - & 25,849 & 7,150 & - & - & - \\ 
   \hline
\hline
\end{longtable}
\endgroup


% latex table generated in R 3.6.3 by xtable 1.8-4 package
% Tue May 12 21:00:57 2020
\begingroup\footnotesize
\begin{longtable}{lrrrrrrr}
\caption{Annual coefficient of variation (CV, \%) estimates for the seven populations
                    examined. Where possible, we estimated CV directly
                    from uncertainty in smolts estimates.} \\ 
  \hline
Year & Campbellton & Conne & LaHave & Nashwaak & Saint-Jean & Trinité & WAB \\ 
  \hline
1974 & - & - & - & - & - & - & 5.0 \\ 
  1975 & - & - & - & - & - & - & 5.0 \\ 
  1976 & - & - & - & - & - & - & 5.0 \\ 
  1977 & - & - & - & - & - & - & 5.0 \\ 
  1978 & - & - & - & - & - & - & 5.0 \\ 
  1979 & - & - & - & - & - & - & 5.0 \\ 
  1980 & - & - & - & - & - & - & 5.0 \\ 
  1981 & - & - & - & - & - & - & 5.0 \\ 
  1982 & - & - & - & - & - & - & 5.0 \\ 
  1983 & - & - & - & - & - & - & 5.0 \\ 
  1984 & - & - & - & - & - & - & 5.0 \\ 
  1985 & - & - & - & - & - & 10.0 & 5.0 \\ 
  1986 & - & - & - & - & - & 10.0 & 5.0 \\ 
  1987 & - & - & - & - & - & 10.0 & 5.0 \\ 
  1988 & - & 10.0 & - & - & - & 10.0 & 5.0 \\ 
  1989 & - & 10.0 & - & - & - & 10.0 & 5.0 \\ 
  1990 & - & 10.0 & - & - & 10.0 & 10.0 & 5.0 \\ 
  1991 & - & 10.0 & - & - & 10.0 & 10.0 & 5.0 \\ 
  1992 & - & 10.0 & - & - & 10.0 & 10.0 & 5.0 \\ 
  1993 & - & 10.0 & - & - & 10.0 & 10.0 & 5.0 \\ 
  1994 & 5.0 & 10.0 & - & - & 10.0 & 10.0 & 5.0 \\ 
  1995 & 5.0 & 10.0 & - & - & 10.0 & 10.0 & 5.0 \\ 
  1996 & 5.0 & 10.0 & - & - & 10.0 & 10.0 & 5.0 \\ 
  1997 & 5.0 & 10.0 & 1.5 & - & 10.0 & 10.0 & 5.0 \\ 
  1998 & 5.0 & 10.0 & 1.7 & - & - & 10.0 & 5.0 \\ 
  1999 & 5.0 & 10.0 & 3.1 & 16.8 & 10.0 & 10.0 & 5.0 \\ 
  2000 & 5.0 & 10.0 & 3.2 & 7.1 & 10.0 & 10.0 & 5.0 \\ 
  2001 & 5.0 & 10.0 & 1.2 & 10.2 & 10.0 & 10.0 & 5.0 \\ 
  2002 & 5.0 & 10.0 & 1.4 & 21.6 & 10.0 & 10.0 & 5.0 \\ 
  2003 & 5.0 & 10.0 & 1.5 & 11.4 & 10.0 & 10.0 & 5.0 \\ 
  2004 & 5.0 & 10.0 & 25.8 & 18.1 & 10.0 & 10.0 & 5.0 \\ 
  2005 & 5.0 & 10.0 & 2.4 & 20.1 & 10.0 & 10.0 & 5.0 \\ 
  2006 & 5.0 & 10.0 & 6.1 & 46.1 & 10.0 & 10.0 & 5.0 \\ 
  2007 & 5.0 & 10.0 & 6.8 & 8.2 & 10.0 & - & 5.0 \\ 
  2008 & 5.0 & 10.0 & 5.7 & 16.0 & 10.0 & 10.0 & 5.0 \\ 
  2009 & 5.0 & 10.0 & 3.5 & 19.9 & 10.0 & 10.0 & 5.0 \\ 
  2010 & 5.0 & 10.0 & 5.6 & 17.2 & 10.0 & 10.0 & 5.0 \\ 
  2011 & 5.0 & 10.0 & 3.3 & 13.9 & 10.0 & 10.0 & 5.0 \\ 
  2012 & 5.0 & 10.0 & - & 12.2 & 10.0 & 10.0 & 5.0 \\ 
  2013 & 5.0 & 10.0 & - & 22.4 & 10.0 & 10.0 & 5.0 \\ 
  2014 & 5.0 & 10.0 & 17.9 & 7.5 & 10.0 & 10.0 & 5.0 \\ 
  2015 & 5.0 & 10.0 & 12.3 & 20.8 & 10.0 & 10.0 & 5.0 \\ 
  2016 & - & - & 5.4 & 11.2 & - & 10.0 & - \\ 
  2017 & - & - & 5.4 & 15.5 & - & - & - \\ 
   \hline
\hline
\end{longtable}
\endgroup



\subsection*{Annual smolt abundances estimates derived from the Bayesian hierarchical model}

%\begin{table}[ht]
%\centering
%\begin{longtable}{lrrrrrr}
%\caption{Posterior annual smolt estimates for the seven populations examined. These
%                    estimates differ from the smolt data used as they include both observation error and shrinkage from
%                    the hierarchical model specification.} 
%\hline
% latex table generated in R 4.0.5 by xtable 1.8-4 package
% Sat Apr 10 12:54:00 2021
\begingroup\footnotesize
\begin{longtable}{llrrrrr}
\caption{Posterior annual smolt estimates for the seven populations examined. These
                    estimates differ from the smolt data used as they include both observation error and shrinkage from
                    the hierarchical model specification.} \\ 
  \hline
Population & Year & Median & 2.5\% & 25\% & 75\% & 97.5\% \\ 
  \hline
Campbellton River & 1993 & 33,511.1 & 28,214.8 & 31,656.1 & 35,583.8 & 39,718.8 \\ 
  Campbellton River & 1994 & 40,938.7 & 34,716.9 & 38,709.4 & 43,412.0 & 48,578.5 \\ 
  Campbellton River & 1995 & 39,635.9 & 33,609.3 & 37,468.9 & 41,997.9 & 46,713.4 \\ 
  Campbellton River & 1996 & 51,791.3 & 43,468.0 & 48,649.6 & 55,177.6 & 62,303.0 \\ 
  Campbellton River & 1997 & 54,216.7 & 45,333.8 & 51,047.4 & 57,772.1 & 65,491.2 \\ 
  Campbellton River & 1998 & 46,716.5 & 39,521.7 & 44,083.7 & 49,564.8 & 55,667.3 \\ 
  Campbellton River & 1999 & 44,435.3 & 37,512.8 & 41,967.7 & 47,153.5 & 53,000.9 \\ 
  Campbellton River & 2000 & 36,482.0 & 30,688.7 & 34,374.8 & 38,605.9 & 43,290.2 \\ 
  Campbellton River & 2001 & 37,594.9 & 31,777.3 & 35,539.9 & 39,810.1 & 44,482.2 \\ 
  Campbellton River & 2002 & 34,368.2 & 28,994.2 & 32,413.2 & 36,394.2 & 40,587.6 \\ 
  Campbellton River & 2003 & 36,264.6 & 30,649.3 & 34,232.0 & 38,391.0 & 42,735.3 \\ 
  Campbellton River & 2004 & 34,874.9 & 29,490.1 & 32,936.6 & 36,896.0 & 41,223.1 \\ 
  Campbellton River & 2005 & 32,605.1 & 27,295.7 & 30,628.3 & 34,576.0 & 38,417.4 \\ 
  Campbellton River & 2006 & 34,873.3 & 29,627.1 & 32,950.5 & 36,913.9 & 40,996.9 \\ 
  Campbellton River & 2007 & 36,958.2 & 31,359.2 & 34,911.8 & 39,141.6 & 43,591.2 \\ 
  Campbellton River & 2008 & 40,282.4 & 34,149.5 & 38,069.0 & 42,682.9 & 47,584.2 \\ 
  Campbellton River & 2009 & 37,745.9 & 31,843.6 & 35,569.5 & 39,985.7 & 44,558.4 \\ 
  Campbellton River & 2010 & 41,087.9 & 35,050.0 & 38,812.4 & 43,488.2 & 48,314.0 \\ 
  Campbellton River & 2011 & 37,692.1 & 31,901.3 & 35,569.3 & 39,892.7 & 44,340.3 \\ 
  Campbellton River & 2012 & 42,858.5 & 36,384.8 & 40,562.7 & 45,439.9 & 50,768.4 \\ 
  Campbellton River & 2013 & 40,293.3 & 34,259.5 & 38,030.8 & 42,707.7 & 47,782.9 \\ 
  Campbellton River & 2014 & 43,933.4 & 37,176.7 & 41,446.8 & 46,464.4 & 51,993.3 \\ 
  Conne River & 1987 & 72,982.0 & 61,825.4 & 68,908.0 & 77,483.5 & 87,108.8 \\ 
  Conne River & 1988 & 65,482.2 & 55,017.2 & 61,773.5 & 69,429.0 & 77,519.6 \\ 
  Conne River & 1989 & 71,732.9 & 60,341.2 & 67,662.7 & 76,033.1 & 85,289.3 \\ 
  Conne River & 1990 & 58,053.5 & 48,954.3 & 54,734.6 & 61,583.5 & 68,852.1 \\ 
  Conne River & 1991 & 71,593.5 & 60,209.2 & 67,412.4 & 76,016.6 & 85,593.2 \\ 
  Conne River & 1992 & 66,899.1 & 56,410.1 & 62,961.7 & 70,896.4 & 79,446.1 \\ 
  Conne River & 1993 & 56,871.2 & 47,419.4 & 53,513.0 & 60,376.3 & 67,765.0 \\ 
  Conne River & 1994 & 61,300.4 & 51,790.9 & 57,941.6 & 65,006.4 & 72,732.1 \\ 
  Conne River & 1995 & 62,942.2 & 52,860.8 & 59,265.1 & 66,731.8 & 74,858.5 \\ 
  Conne River & 1996 & 85,514.1 & 71,929.6 & 80,495.6 & 91,094.4 & 102,970.9 \\ 
  Conne River & 1997 & 90,293.7 & 75,537.8 & 84,603.9 & 96,231.9 & 109,310.0 \\ 
  Conne River & 1998 & 67,795.0 & 57,258.1 & 63,907.7 & 71,829.2 & 80,229.2 \\ 
  Conne River & 1999 & 64,146.0 & 54,015.2 & 60,541.4 & 68,102.9 & 76,325.7 \\ 
  Conne River & 2000 & 60,765.2 & 51,087.5 & 57,176.4 & 64,449.2 & 72,241.7 \\ 
  Conne River & 2001 & 80,523.2 & 67,730.5 & 75,800.6 & 85,482.6 & 95,987.6 \\ 
  Conne River & 2002 & 76,531.0 & 64,183.8 & 72,076.1 & 81,264.1 & 91,233.2 \\ 
  Conne River & 2003 & 69,567.4 & 58,373.8 & 65,535.7 & 73,768.5 & 82,631.0 \\ 
  Conne River & 2004 & 74,936.7 & 63,054.6 & 70,592.6 & 79,691.8 & 89,260.1 \\ 
  Conne River & 2005 & 65,108.8 & 54,609.6 & 61,358.3 & 69,098.2 & 77,264.2 \\ 
  Conne River & 2006 & 39,790.6 & 32,651.6 & 37,242.8 & 42,388.9 & 48,028.5 \\ 
  Conne River & 2007 & 63,414.6 & 53,671.0 & 59,987.0 & 67,355.5 & 75,256.3 \\ 
  Conne River & 2008 & 66,547.6 & 55,851.4 & 62,751.8 & 70,573.2 & 79,212.3 \\ 
  Conne River & 2009 & 68,588.7 & 57,863.5 & 64,750.5 & 72,899.3 & 81,584.0 \\ 
  Conne River & 2010 & 55,698.9 & 46,824.9 & 52,400.3 & 59,057.4 & 66,438.4 \\ 
  Conne River & 2011 & 53,062.2 & 44,445.4 & 49,865.9 & 56,381.2 & 62,950.1 \\ 
  Conne River & 2012 & 53,644.7 & 45,173.6 & 50,484.3 & 56,884.6 & 63,825.4 \\ 
  Conne River & 2013 & 64,846.4 & 54,684.0 & 61,110.6 & 68,840.8 & 77,030.1 \\ 
  Conne River & 2014 & 57,410.4 & 48,248.9 & 54,152.3 & 60,921.2 & 68,039.9 \\ 
  LaHave River & 1996 & 19,860.3 & 16,421.4 & 18,574.1 & 21,196.2 & 23,976.9 \\ 
  LaHave River & 1997 & 16,411.3 & 13,670.1 & 15,366.6 & 17,473.1 & 19,721.7 \\ 
  LaHave River & 1998 & 15,341.6 & 12,657.5 & 14,321.7 & 16,385.1 & 18,430.4 \\ 
  LaHave River & 1999 & 10,593.3 & 8,756.4 & 9,924.6 & 11,300.5 & 12,823.4 \\ 
  LaHave River & 2000 & 15,897.2 & 13,076.2 & 14,858.5 & 17,020.8 & 19,165.0 \\ 
  LaHave River & 2001 & 15,506.0 & 12,834.7 & 14,522.8 & 16,556.9 & 18,719.9 \\ 
  LaHave River & 2002 & 11,798.6 & 9,731.7 & 11,016.6 & 12,624.3 & 14,322.4 \\ 
  LaHave River & 2003 & 17,355.9 & 14,334.1 & 16,279.0 & 18,526.2 & 21,016.5 \\ 
  LaHave River & 2004 & 19,914.3 & 16,494.4 & 18,652.0 & 21,286.8 & 24,141.3 \\ 
  LaHave River & 2005 & 5,562.0 & 4,582.0 & 5,208.2 & 5,938.7 & 6,738.9 \\ 
  LaHave River & 2006 & 22,216.1 & 18,249.6 & 20,740.3 & 23,714.1 & 26,890.1 \\ 
  LaHave River & 2007 & 23,551.9 & 19,496.5 & 22,071.8 & 25,150.2 & 28,447.8 \\ 
  LaHave River & 2008 & 14,177.1 & 11,678.0 & 13,305.7 & 15,149.4 & 17,154.7 \\ 
  LaHave River & 2009 & 8,785.0 & 7,266.7 & 8,222.3 & 9,394.8 & 10,595.1 \\ 
  LaHave River & 2010 & 15,904.5 & 12,991.2 & 14,824.6 & 16,974.7 & 19,313.4 \\ 
  LaHave River & 2013 & 7,217.8 & 5,963.2 & 6,763.5 & 7,705.0 & 8,771.4 \\ 
  LaHave River & 2014 & 27,648.3 & 22,952.4 & 25,822.7 & 29,515.6 & 33,631.1 \\ 
  LaHave River & 2015 & 6,723.6 & 5,554.4 & 6,291.1 & 7,182.6 & 8,138.3 \\ 
  LaHave River & 2016 & 24,634.7 & 20,283.0 & 23,069.6 & 26,385.0 & 29,870.9 \\ 
  Nashwaak River & 1998 & 22,059.2 & 18,220.4 & 20,622.2 & 23,525.3 & 26,653.3 \\ 
  Nashwaak River & 1999 & 27,233.3 & 22,559.8 & 25,536.7 & 29,032.7 & 33,194.3 \\ 
  Nashwaak River & 2000 & 15,349.1 & 12,652.7 & 14,385.4 & 16,418.0 & 18,620.9 \\ 
  Nashwaak River & 2001 & 11,010.5 & 9,102.4 & 10,332.7 & 11,773.6 & 13,381.8 \\ 
  Nashwaak River & 2002 & 14,807.7 & 12,256.2 & 13,883.4 & 15,833.2 & 17,964.8 \\ 
  Nashwaak River & 2003 & 9,271.2 & 7,701.4 & 8,719.8 & 9,890.7 & 11,149.8 \\ 
  Nashwaak River & 2004 & 13,663.5 & 11,283.8 & 12,829.9 & 14,538.0 & 16,414.6 \\ 
  Nashwaak River & 2005 & 5,642.6 & 4,676.9 & 5,291.3 & 6,028.9 & 6,801.5 \\ 
  Nashwaak River & 2006 & 24,270.0 & 20,093.5 & 22,754.7 & 25,924.0 & 29,437.3 \\ 
  Nashwaak River & 2007 & 21,265.8 & 17,551.2 & 19,865.2 & 22,711.7 & 25,733.6 \\ 
  Nashwaak River & 2008 & 7,539.6 & 6,265.0 & 7,055.0 & 8,047.1 & 9,098.3 \\ 
  Nashwaak River & 2009 & 16,725.7 & 13,975.8 & 15,717.4 & 17,826.2 & 20,162.9 \\ 
  Nashwaak River & 2010 & 12,627.3 & 10,488.9 & 11,837.5 & 13,474.3 & 15,183.2 \\ 
  Nashwaak River & 2011 & 8,741.1 & 7,196.1 & 8,160.6 & 9,345.1 & 10,702.7 \\ 
  Nashwaak River & 2012 & 10,898.7 & 9,063.2 & 10,234.3 & 11,663.2 & 13,198.7 \\ 
  Nashwaak River & 2013 & 10,045.4 & 8,347.5 & 9,399.5 & 10,757.4 & 12,150.1 \\ 
  Nashwaak River & 2014 & 11,055.1 & 9,177.2 & 10,387.0 & 11,782.4 & 13,342.9 \\ 
  Nashwaak River & 2015 & 8,123.0 & 6,739.3 & 7,617.9 & 8,655.1 & 9,807.0 \\ 
  Nashwaak River & 2016 & 7,231.6 & 5,999.0 & 6,770.2 & 7,733.1 & 8,754.0 \\ 
  Saint-Jean River & 1989 & 90,849.5 & 75,391.4 & 85,199.7 & 97,089.9 & 109,430.0 \\ 
  Saint-Jean River & 1990 & 95,207.4 & 78,702.6 & 89,150.1 & 101,775.1 & 115,067.6 \\ 
  Saint-Jean River & 1991 & 110,311.5 & 90,636.2 & 103,244.3 & 117,815.1 & 133,665.9 \\ 
  Saint-Jean River & 1992 & 147,047.9 & 121,445.5 & 137,591.5 & 157,185.4 & 178,454.6 \\ 
  Saint-Jean River & 1993 & 136,044.2 & 112,819.0 & 127,298.4 & 145,492.5 & 164,797.9 \\ 
  Saint-Jean River & 1994 & 73,088.8 & 60,708.5 & 68,565.0 & 77,924.4 & 88,160.5 \\ 
  Saint-Jean River & 1995 & 60,199.5 & 50,085.1 & 56,422.4 & 64,181.2 & 72,481.0 \\ 
  Saint-Jean River & 1996 & 101,200.3 & 83,420.4 & 94,640.3 & 108,245.9 & 122,640.6 \\ 
  Saint-Jean River & 1998 & 92,842.3 & 76,377.3 & 86,985.9 & 99,253.9 & 112,829.1 \\ 
  Saint-Jean River & 1999 & 109,916.6 & 90,975.3 & 102,980.2 & 117,263.8 & 133,172.5 \\ 
  Saint-Jean River & 2000 & 51,453.8 & 42,541.2 & 48,108.9 & 55,000.7 & 62,197.9 \\ 
  Saint-Jean River & 2001 & 106,355.9 & 87,924.6 & 99,362.4 & 113,710.1 & 129,084.3 \\ 
  Saint-Jean River & 2002 & 71,137.1 & 58,914.2 & 66,731.6 & 75,999.3 & 86,290.4 \\ 
  Saint-Jean River & 2003 & 60,435.6 & 49,835.5 & 56,587.0 & 64,485.2 & 73,163.3 \\ 
  Saint-Jean River & 2004 & 55,402.3 & 45,380.5 & 51,698.0 & 58,986.8 & 66,998.2 \\ 
  Saint-Jean River & 2005 & 93,190.9 & 77,042.6 & 87,104.8 & 99,594.0 & 112,431.8 \\ 
  Saint-Jean River & 2006 & 99,511.0 & 81,480.9 & 93,037.2 & 106,326.5 & 120,683.0 \\ 
  Saint-Jean River & 2007 & 129,333.3 & 107,290.6 & 121,299.0 & 138,199.9 & 156,103.4 \\ 
  Saint-Jean River & 2008 & 46,719.6 & 38,588.0 & 43,755.2 & 49,818.0 & 56,299.0 \\ 
  Saint-Jean River & 2009 & 38,421.9 & 31,849.0 & 36,021.2 & 41,036.0 & 46,362.8 \\ 
  Saint-Jean River & 2010 & 48,919.7 & 40,473.8 & 45,823.3 & 52,140.6 & 59,061.6 \\ 
  Saint-Jean River & 2011 & 45,848.4 & 38,172.9 & 42,977.7 & 49,032.6 & 55,461.2 \\ 
  Saint-Jean River & 2012 & 41,371.8 & 34,424.0 & 38,749.6 & 44,140.8 & 49,946.9 \\ 
  Saint-Jean River & 2013 & 38,128.5 & 31,559.8 & 35,691.0 & 40,665.0 & 46,160.6 \\ 
  Saint-Jean River & 2014 & 57,314.9 & 47,225.8 & 53,807.1 & 61,036.0 & 69,388.8 \\ 
  Trinité River & 1984 & 65,585.1 & 54,618.8 & 61,499.8 & 69,783.6 & 78,802.4 \\ 
  Trinité River & 1985 & 64,090.3 & 53,148.7 & 60,063.3 & 68,364.0 & 77,072.5 \\ 
  Trinité River & 1986 & 90,005.9 & 74,381.7 & 84,432.8 & 95,873.3 & 108,671.1 \\ 
  Trinité River & 1987 & 74,036.0 & 61,404.2 & 69,398.6 & 79,043.5 & 89,222.4 \\ 
  Trinité River & 1988 & 51,695.5 & 43,071.0 & 48,710.5 & 55,003.8 & 62,347.8 \\ 
  Trinité River & 1989 & 76,359.4 & 63,401.6 & 71,590.4 & 81,527.1 & 91,879.4 \\ 
  Trinité River & 1990 & 50,082.2 & 41,694.5 & 46,960.7 & 53,430.3 & 60,171.0 \\ 
  Trinité River & 1991 & 40,866.9 & 34,141.2 & 38,378.2 & 43,554.2 & 49,206.9 \\ 
  Trinité River & 1992 & 49,755.8 & 41,204.5 & 46,709.2 & 52,945.9 & 59,834.8 \\ 
  Trinité River & 1993 & 80,653.4 & 66,791.5 & 75,620.4 & 85,887.8 & 97,572.7 \\ 
  Trinité River & 1994 & 54,204.0 & 44,939.9 & 50,855.7 & 57,789.1 & 65,125.2 \\ 
  Trinité River & 1995 & 68,400.9 & 56,725.6 & 63,985.9 & 72,975.7 & 82,084.1 \\ 
  Trinité River & 1996 & 58,718.0 & 48,836.9 & 55,096.5 & 62,698.3 & 70,631.1 \\ 
  Trinité River & 1997 & 32,682.9 & 26,953.1 & 30,592.8 & 34,974.1 & 39,628.6 \\ 
  Trinité River & 1998 & 29,863.0 & 24,690.7 & 28,027.5 & 31,868.2 & 36,028.9 \\ 
  Trinité River & 1999 & 54,591.5 & 45,410.9 & 51,207.0 & 58,246.9 & 65,661.8 \\ 
  Trinité River & 2000 & 39,404.7 & 32,753.8 & 36,976.7 & 41,995.4 & 47,683.8 \\ 
  Trinité River & 2001 & 66,649.0 & 55,330.6 & 62,615.0 & 71,258.0 & 80,738.4 \\ 
  Trinité River & 2002 & 43,825.1 & 36,390.9 & 41,108.6 & 46,761.3 & 52,940.9 \\ 
  Trinité River & 2003 & 51,479.4 & 42,635.8 & 48,168.4 & 54,842.8 & 62,275.2 \\ 
  Trinité River & 2004 & 27,958.8 & 23,126.4 & 26,188.4 & 29,797.1 & 33,719.3 \\ 
  Trinité River & 2005 & 35,407.5 & 29,159.4 & 33,140.7 & 37,743.1 & 42,625.3 \\ 
  Trinité River & 2007 & 42,810.1 & 35,284.4 & 40,092.3 & 45,715.1 & 51,726.2 \\ 
  Trinité River & 2008 & 35,277.0 & 29,200.7 & 32,970.4 & 37,687.7 & 42,597.3 \\ 
  Trinité River & 2009 & 33,515.9 & 27,785.1 & 31,389.2 & 35,740.9 & 40,537.2 \\ 
  Trinité River & 2010 & 38,126.1 & 31,551.9 & 35,771.2 & 40,787.9 & 46,344.1 \\ 
  Trinité River & 2011 & 43,796.7 & 36,083.5 & 41,060.1 & 46,791.3 & 52,950.4 \\ 
  Trinité River & 2012 & 44,379.5 & 36,849.6 & 41,603.5 & 47,287.9 & 53,342.6 \\ 
  Trinité River & 2013 & 41,945.6 & 34,919.5 & 39,396.4 & 44,666.4 & 50,667.3 \\ 
  Trinité River & 2014 & 31,686.3 & 26,202.4 & 29,625.1 & 33,754.9 & 38,189.1 \\ 
  Trinité River & 2015 & 46,944.5 & 38,984.4 & 44,016.0 & 50,125.9 & 56,574.4 \\ 
  Western Arm Brook & 1973 & 8,912.1 & 7,374.7 & 8,360.5 & 9,511.9 & 10,727.8 \\ 
  Western Arm Brook & 1974 & 11,970.1 & 9,957.9 & 11,222.1 & 12,767.0 & 14,421.4 \\ 
  Western Arm Brook & 1975 & 9,954.3 & 8,228.2 & 9,334.4 & 10,611.1 & 11,929.1 \\ 
  Western Arm Brook & 1976 & 6,776.7 & 5,637.5 & 6,354.7 & 7,230.2 & 8,176.6 \\ 
  Western Arm Brook & 1977 & 10,131.7 & 8,478.0 & 9,492.5 & 10,774.2 & 12,096.0 \\ 
  Western Arm Brook & 1978 & 13,374.1 & 11,156.6 & 12,580.1 & 14,220.6 & 15,982.2 \\ 
  Western Arm Brook & 1979 & 8,762.7 & 7,296.4 & 8,238.8 & 9,334.6 & 10,565.8 \\ 
  Western Arm Brook & 1980 & 15,189.6 & 12,650.5 & 14,269.1 & 16,205.3 & 18,303.2 \\ 
  Western Arm Brook & 1981 & 13,755.0 & 11,522.0 & 12,938.7 & 14,662.0 & 16,491.8 \\ 
  Western Arm Brook & 1982 & 12,677.0 & 10,586.0 & 11,917.8 & 13,499.6 & 15,134.5 \\ 
  Western Arm Brook & 1983 & 10,689.7 & 8,897.2 & 10,045.5 & 11,360.5 & 12,873.0 \\ 
  Western Arm Brook & 1984 & 19,397.6 & 16,194.6 & 18,194.4 & 20,670.0 & 23,349.2 \\ 
  Western Arm Brook & 1985 & 13,305.2 & 11,063.3 & 12,486.5 & 14,175.7 & 16,021.2 \\ 
  Western Arm Brook & 1986 & 16,937.7 & 14,112.3 & 15,887.0 & 18,095.0 & 20,487.2 \\ 
  Western Arm Brook & 1987 & 16,333.9 & 13,606.3 & 15,330.0 & 17,394.8 & 19,662.0 \\ 
  Western Arm Brook & 1988 & 14,942.3 & 12,413.9 & 14,021.2 & 15,859.6 & 17,876.4 \\ 
  Western Arm Brook & 1989 & 11,515.6 & 9,562.8 & 10,789.5 & 12,284.0 & 13,843.3 \\ 
  Western Arm Brook & 1990 & 10,702.8 & 8,898.5 & 10,024.1 & 11,396.0 & 12,827.2 \\ 
  Western Arm Brook & 1991 & 13,318.3 & 11,038.5 & 12,518.1 & 14,192.4 & 15,972.3 \\ 
  Western Arm Brook & 1992 & 15,173.7 & 12,647.7 & 14,246.0 & 16,143.5 & 18,172.0 \\ 
  Western Arm Brook & 1993 & 13,494.3 & 11,185.2 & 12,654.9 & 14,353.6 & 16,200.6 \\ 
  Western Arm Brook & 1994 & 9,754.7 & 8,089.7 & 9,158.0 & 10,370.2 & 11,708.4 \\ 
  Western Arm Brook & 1995 & 15,057.9 & 12,530.7 & 14,137.8 & 16,012.9 & 18,060.4 \\ 
  Western Arm Brook & 1996 & 14,237.9 & 11,880.7 & 13,385.5 & 15,180.7 & 17,127.6 \\ 
  Western Arm Brook & 1997 & 22,461.5 & 18,646.8 & 21,038.6 & 23,950.7 & 27,168.6 \\ 
  Western Arm Brook & 1998 & 16,688.0 & 13,881.4 & 15,670.4 & 17,751.1 & 20,032.6 \\ 
  Western Arm Brook & 1999 & 13,705.4 & 11,481.9 & 12,868.4 & 14,626.2 & 16,464.0 \\ 
  Western Arm Brook & 2000 & 12,677.4 & 10,582.5 & 11,912.4 & 13,512.5 & 15,193.5 \\ 
  Western Arm Brook & 2001 & 15,871.0 & 13,231.2 & 14,886.4 & 16,845.7 & 19,014.5 \\ 
  Western Arm Brook & 2002 & 14,987.4 & 12,492.0 & 14,111.7 & 15,918.4 & 17,955.2 \\ 
  Western Arm Brook & 2003 & 12,359.2 & 10,258.1 & 11,622.3 & 13,194.8 & 14,865.6 \\ 
  Western Arm Brook & 2004 & 16,788.9 & 13,941.3 & 15,773.6 & 17,855.9 & 20,185.4 \\ 
  Western Arm Brook & 2005 & 9,308.6 & 7,781.9 & 8,747.6 & 9,898.2 & 11,118.0 \\ 
  Western Arm Brook & 2006 & 19,637.6 & 16,317.7 & 18,452.2 & 20,917.7 & 23,577.4 \\ 
  Western Arm Brook & 2007 & 16,539.1 & 13,735.8 & 15,549.1 & 17,587.0 & 19,852.0 \\ 
  Western Arm Brook & 2008 & 16,911.1 & 14,155.6 & 15,895.9 & 18,016.7 & 20,279.3 \\ 
  Western Arm Brook & 2009 & 18,092.8 & 15,184.2 & 16,986.1 & 19,293.2 & 21,588.2 \\ 
  Western Arm Brook & 2010 & 18,371.4 & 15,321.1 & 17,262.9 & 19,581.1 & 22,107.3 \\ 
  Western Arm Brook & 2011 & 19,572.9 & 16,310.4 & 18,365.6 & 20,836.5 & 23,535.1 \\ 
  Western Arm Brook & 2012 & 13,485.1 & 11,226.6 & 12,665.5 & 14,340.3 & 16,102.4 \\ 
  Western Arm Brook & 2013 & 18,975.3 & 15,786.0 & 17,824.5 & 20,222.1 & 22,818.5 \\ 
  Western Arm Brook & 2014 & 19,072.5 & 15,967.7 & 17,942.7 & 20,266.1 & 22,854.7 \\ 
  LaHave River & 2011 & - & - & - & - & - \\ 
  LaHave River & 2012 & - & - & - & - & - \\ 
  Saint-Jean River & 1997 & - & - & - & - & - \\ 
  Trinité River & 2006 & - & - & - & - & - \\ 
   \hline
\hline
\end{longtable}
\endgroup

%\end{longtable}
%\end{table}
\clearpage

\subsection*{Population-level estimates of proportion returning after one winter at sea}

% latex table generated in R 3.6.3 by xtable 1.8-4 package
% Tue May 12 21:41:30 2020
\begin{table}[ht]
\centering
\caption{Posterior estimates of the population-level 
                    mean smolt abundance for the seven populations examined.} 
\begin{tabular}{lrrrrr}
  \hline
Population & Median & 2.5\% & 25\% & 75\% & 97.5\% \\ 
  \hline
Campbellton River & 39,543.0 & 36,454.6 & 38,473.1 & 40,660.1 & 42,952.6 \\ 
  Conne River & 65,103.2 & 59,884.4 & 63,386.0 & 66,934.9 & 70,790.3 \\ 
  LaHave River & 14,361.5 & 11,372.3 & 13,325.8 & 15,515.3 & 18,462.7 \\ 
  Nashwaak River & 12,356.7 & 9,823.7 & 11,470.8 & 13,280.6 & 15,587.3 \\ 
  Saint-Jean River & 73,731.8 & 61,204.6 & 69,429.6 & 78,505.3 & 89,053.0 \\ 
  Trinité River & 48,167.3 & 42,627.7 & 46,225.9 & 50,213.1 & 54,736.8 \\ 
  Western Arm Brook & 13,882.8 & 12,631.4 & 13,450.6 & 14,328.1 & 15,203.3 \\ 
   \hline
\end{tabular}
\end{table}


\hspace{8em}

% latex table generated in R 3.6.3 by xtable 1.8-4 package
% Thu Oct  8 10:51:43 2020
\begin{table}[ht]
\centering
\caption{Posterior estimates of the population-level 
                    mean log-smolt abundance for the seven populations examined.} 
\begin{tabular}{lrrrrr}
  \hline
Population & Median & 2.5\% & 25\% & 75\% & 97.5\% \\ 
  \hline
Campbellton River & 10.587 & 10.501 & 10.559 & 10.614 & 10.672 \\ 
  Conne River & 11.083 & 11.002 & 11.057 & 11.111 & 11.167 \\ 
  LaHave River & 9.575 & 9.341 & 9.497 & 9.650 & 9.809 \\ 
  Nashwaak River & 9.421 & 9.190 & 9.346 & 9.497 & 9.650 \\ 
  Saint-Jean River & 11.209 & 11.021 & 11.146 & 11.270 & 11.399 \\ 
  Trinité River & 10.781 & 10.657 & 10.742 & 10.824 & 10.906 \\ 
  Western Arm Brook & 9.537 & 9.444 & 9.507 & 9.569 & 9.632 \\ 
   \hline
\end{tabular}
\end{table}


% latex table generated in R 4.0.5 by xtable 1.8-4 package
% Sun Apr 18 20:40:06 2021
\begin{table}[ht]
\centering
\caption{Posterior estimates of the population-level standard 
                    deviation of log-smolt abundance for the seven populations examined.} 
\begin{tabular}{lrrrrr}
  \hline
Population & Median & 2.5\% & 25\% & 75\% & 97.5\% \\ 
  \hline
Campbellton River & 0.164 & 0.105 & 0.142 & 0.191 & 0.255 \\ 
  Conne River & 0.191 & 0.135 & 0.169 & 0.215 & 0.273 \\ 
  LaHave River & 0.490 & 0.355 & 0.435 & 0.559 & 0.728 \\ 
  Nashwaak River & 0.473 & 0.343 & 0.421 & 0.539 & 0.707 \\ 
  Saint-Jean River & 0.444 & 0.336 & 0.402 & 0.492 & 0.621 \\ 
  Trinité River & 0.333 & 0.254 & 0.303 & 0.367 & 0.448 \\ 
  Western Arm Brook & 0.285 & 0.223 & 0.262 & 0.311 & 0.368 \\ 
   \hline
\end{tabular}
\end{table}


% \begin{figure}[htbp] \centering
%     \includegraphics[width=0.9\linewidth]{simul_timeseries_declining.pdf} \caption{Simulated
%         time series of returning adult salmon abundance in the six scenarios. The lines denote the
%     simulated abundance estimates without observation error while the points are the same estimates
% including observation error.}
% \label{fig:simul-ts-dec} 
% \end{figure}

% \begin{figure}[htbp] \centering
%     \includegraphics[width=0.95\linewidth]{sir_out_18_declining.pdf} \caption{Yearly
%         estimated \So, \St, and \Pg values in the six scenarios. True values
%         are denoted by blue circles, black circles show median estimates,
%         error bars indicate the 25\% and 75\% quantiles.}
% \label{fig:sir-out-dec} 
% \end{figure}

% \begin{figure}[htbp] \centering
%     \includegraphics[width=0.7\linewidth]{simul_plots_6_declining.pdf}
%     \caption{Comparison of estimated and true \So values in the six 
%         scenarios. True \So values are deterministic, black circles show
%         median \So estimates, error bars indicate the  25\% and 75\%
%         quantiles, while the blue line denotes a linear model fit of the
%         medians. The one-to-one relationship is shown by the gray dashed
%         line.} 
% \label{fig:simul-six-plots-dec} 
% \end{figure}


%\clearpage
%\subsection*{References}
\nobibliography{subset}

\end{document}

