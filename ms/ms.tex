\documentclass[12pt]{article}
\usepackage[top=0.85in,left=1.0in,right=1.0in,footskip=0.75in]{geometry}
%\usepackage[parfill]{parskip}
\usepackage{setspace}
\usepackage{lineno}
\usepackage[hidelinks]{hyperref}
%\onehalfspacing
\doublespacing

\usepackage[round,sectionbib]{natbib}
\setcitestyle{authoryear}
\bibpunct{(}{)}{;}{a}{}{;}
\bibliographystyle{mee}

%\usepackage{textcomp}
%\usepackage{libertine}
%\usepackage{inconsolata} % sans serif typewriter

\usepackage{mathtools} % for dcases
\usepackage{xcolor} % for textcolor
\usepackage{makecell} % for \makecell and within cell line breaks (\thead)

\usepackage[T1]{fontenc}

%\makeatletter\let\expandableinput\@@input\makeatother % expandable input for \input inside tables

% Linux Libertine:
\usepackage{textcomp}
\usepackage[sb]{libertine}
\usepackage[varqu,varl]{inconsolata}% sans serif typewriter
\usepackage[libertine,bigdelims,vvarbb]{newtxmath} % bb from STIX
\usepackage[vvarbb]{newtxmath} % bb from STIX, removed bigdelims for ScholarOne rendering
\usepackage[cal=boondoxo]{mathalfa} % mathcal
%\useosf % osf for text, not math
%\usepackage[supstfm=libertinesups,%
%  supscaled=1.2,%
%  raised=-.13em]{superiors}

\usepackage{xspace}
\usepackage{xfrac} % for diagonal inline fractions in text
%\usepackage{array} % for making whole row bold in table
\usepackage{colortbl} % for background colours in table rows
\usepackage{longtable}
\usepackage{amssymb} % for \checkmark 
\usepackage{amsmath} % for \checkmark 
\usepackage{rotating}
\usepackage[nolists,tablesfirst]{endfloat} % for putting figs and tables at end of document
\DeclareDelayedFloatFlavor{sidewaystable}{table}
\usepackage{makecell} % for \makecell in tables
\usepackage{doi}

%\usepackage{xr} % to obtain label references from supp materials file
%\externaldocument[S-]{suppmat}

\usepackage{tabularx}
\usepackage{booktabs}
\usepackage{array} % for table wrapping of columns
\newcolumntype{L}[1]{>{\raggedright\let\newline\\\arraybackslash\hspace{0pt}}m{#1}}
\newcolumntype{C}[1]{>{\centering\let\newline\\\arraybackslash\hspace{0pt}}m{#1}}
\newcolumntype{R}[1]{>{\raggedleft\let\newline\\\arraybackslash\hspace{0pt}}m{#1}}
%\usepackage[detect-all]{siunitx} % for SI units

% custom hyphenation:
\hyphenation{inverse}
\hyphenation{At-lantic}
\hyphenation{elasmo-branchs}


% Macros
\newcommand{\So}{$S_{1}$\xspace}
\newcommand{\St}{$S_{2}$\xspace}
\newcommand{\Pg}{$P_r$\xspace}
\newcommand{\prmu}{$\mu_r$\xspace}
\newcommand{\prsig}{$\sigma_r$\xspace}
\newcommand{\Linf}{$L_{\infty}$}
\newcommand{\DWinf}{$DW_{\infty}$}
\newcommand{\alphat}{$\tilde{\alpha}$}
\newcommand{\lamat}{$l_{\alpha_{mat}}$}
\newcommand{\lamatb}{$l_{\alpha_{mat}}b$}
\newcommand{\rmax}{$r_{max}$\xspace}
\newcommand{\ageratio}{$\alpha_{mat}/\alpha_{max}$}
\newcommand{\yr}{year\textsuperscript{-1}}
\newcommand{\rsq}{$R^2$\xspace}
\newcommand{\mytilde}{\raise.17ex\hbox{$\scriptstyle\mathtt{\sim}$}}
% Select what to do with command \comment:  
%\newcommand{\comment}[1]{}  % comment not shown
\newcommand{\comment}[1]{\par {\bfseries \color{blue} #1 \par}} % comment shown
%% END MACROS SECTION

\begin{document}
\linenumbers


\section*{Trends in marine survival of Altantic salmon in eastern Canada}

\textbf{Sebasti\'{a}n A. Pardo\textsuperscript{1*}, 
        Geir H. Bolstad\textsuperscript{2}, 
        J. Brian Dempson\textsuperscript{3}, 
        Julien April\textsuperscript{4}, 
        Ross A. Jones\textsuperscript{5}, 
        Martha J. Robertson\textsuperscript{3}, 
        Dustin Raab\textsuperscript{6}, 
Jeffrey A. Hutchings\textsuperscript{1}} 

\noindent\small{\textsuperscript{1} Department of Biology, Dalhousie University, Halifax, NS, B3H 4R2, Canada\\}
\small{\textsuperscript{2} Norwegian Institute for Nature Research (NINA), Trondheim, Norway\\}
\small{\textsuperscript{3} Department of Fisheries and Oceans Canada, St. John's, NL, Canada\\}
\small{\textsuperscript{4} Minist\`{e}re des For\^{e}ts, de la Faune et des Parcs, Qu\`{e}bec, QC, Canada\\}
\small{\textsuperscript{5} Department of Fisheries and Oceans Canada, Moncton, NB, Canada\\}
\small{\textsuperscript{6} Department of Fisheries and Oceans Canada, Dartmouth, NS, Canada\\}
\small{\textsuperscript{*} Corresponding author: spardo@dal.ca}

\section*{Abstract}

The marine phase of anadromous Atlantic salmon (\emph{Salmo salar}) is the
least known yet one of the most crucial with regards to population
persistence. Recently, declines in many Atlantic salmon populations in eastern
Canada have been attributed to changes in conditions at sea, negatively
affecting their survival. However, marine survival estimates are difficult to
obtain given that many individuals spend multiple winters in the ocean before
returning to freshwater to spawn, necessitating the estimation of multiple
parameters. To do so, we fit a hierarchical Bayesian maturity schedule model
to smolt and adult abundance time series for seven populations located in
Newfoundland and Labrador (NL), Qu\'{e}bec (QC), New Brunswick (NB), and Nova
Scotia (NS). The datasets ranged between 19 and 42 years in length. We
estimated three components of marine survival: survival in the first year at
sea (\So), survival in the second year at sea (\St), and proportion returning
as one sea-winter adults (\Pg). Controlling for time frame, trends in
estimates of \So were not consistent among rivers.
In the four populations for which data extended to 1990, marine survival
during the first year at sea predominantly increased over time in Western Arm
Brook (NL) and Rivi\`{e}re Saint-Jean (QC) but largely declined in Conne River
(NL) and Rivi\`{e}re Trinit\'{e} (QC). In the three other populations, \So
exhibited an increase in Campbellton River (NL), a decline in LaHave River
(NS), and fluctuating stability in Nashwaak River (NB) since approximately the
mid-1990s. 
Estimates of \St were highly uncertain and trends could not be assessed. \Pg
was temporally stable, except for LaHave and Nashwaak Rivers. Our findings
challenge the assumption that marine survival has changed in a consistent
manner among populations across a broad geographic scale, and suggest that
trends can be river-specific. Further work exploring potential correlates of
these trends in marine survival is warranted.

Keywords: salmonid, survival at sea, natural mortality, marine mortality

%Running Head: 

\section*{Introduction} % (4-5 paragraphs)

%1. Declining salmon numbers


Reductions in fishing mortality, albeit necessary, are not always sufficient
to facilitate population recovery. Experience with numerous commercially
exploited marine fisheries since the early 1990s has shown that not all
populations respond as favourably as anticipated to considerable reductions in
exploitation \citep{Hutchings2017}. Efforts to close commercial
Atlantic salmon (\emph{Salmo salar}) fisheries in eastern Canada culminated in
moratoria in all regions, beginning in the Maritime provinces (1984) and
following in Newfoundland (1992), Labrador (1998), and Québec (2000). Since
these closures, many populations have not increased as expected (REF): some 
have been assessed as considered threatened or endangered by the 
Committee on the Status of Endangered Wildlife in Canada \citep[COSEWIC, ][]{Cosewic2010}, 
Canada’s national science advisory body (to the national government) on species risk of extinction.
While it is not fully understood what is driving population declines, the potential
drivers of these are many \citep[see ][for a detailed discussion of possible
causes]{Cairns2001}: fishing mortality, damming of waterways
and changes in their freshwater habitats \citep{Dunfield1985}, acidification
of freshwater habitats \citep[particularly in the Southern Uplands region of
NS, see][]{Gibson2010}, predation by seals and birds \citep{Cairns2000}, negative
effects of interbreeding or interactions with escaped farmed salmon
\citep{Keyser2018}, and changes in their marine survival.


% 2.
Over the past three decades, a narrative has emerged that marine survival of Atlantic salmon has
declined throughout the North Atlantic \citep{Hansen1998,OMaoileidigh2003,Chaput2012a}. 
An implicit assumption, albeit infrequently acknowledged, is that any trend in
survival in ocean habitat that is shared by multiple populations during their
seaward migration period will be experienced similarly by multiple
populations. Put another way, given that salmon from different river systems
are hypothesized to share marine habitat during some of their time at sea, it
has been implicitly assumed that populations share similar temporal trends in
at-sea mortality \citep{Friedland1993, Friedland1998}.  
A recent study by \citet{Olmos2019} suggested that trends in post-smolt
survival, when estimated at the stock unit level, are synchronously declining
for all Atlantic salmon in eastern North America.

% This perception is widespread both in the scientific
% literature as well as federal reports. The latest status report on Atlantic
% salmon by COSEWIC states that ``While the mechanism(s) of marine mortality is
% uncertain, what is clear is that the recent period of poor sea survival is
% occurring in parallel with many widespread changes in the North Atlantic
% ecosystem.'' \citep{Cosewic2010}. Consequently, there have several lines of
% inquiry as to what the causes behind these declines might be attributable to
% \citep{Friedland1993, Friedland1998}.


In contrast to the narrative of widespread, demographically similar increases
in at-sea mortality, the conservation status of Canadian populations differs
considerably. Populations in the southern part of the Canadian range are more
likely to be assessed as being below conservation targets than those in more
northerly regions \citep{Cosewic2010}. This geographical disparity in status
suggests that if marine survival has been, or is, a key factor responsible for
most population declines, these changes are not uniformly distributed across
all populations.  

% Nonetheless, most of the populations declines assessed by COSEWIC have
% occurred in populations in the southern extent of its distribution:
% populations in the Bay of Fundy, Anticosti Island, and the Atlantic coast of
% Nova Scotia being assessed as Endangered, populations in the south coast of
% Newfoundland assessed as Threatened, And the populations in the New Brunswick
% and Qu\`{e}bec coasts of the Gulf of St. Lawrence assessed as Special Concern
% \citep{Cosewic2010}. On the other hand, northermost populations have shown
% stable, or even increasing population trends, suggesting that, if marine
% survival were to be a factor in many of these declines, these changes are not
% uniformly distributed across all populations.

% While the purported decline of marine survival is mentioned widely in the 
% scientific literature, only relatively few studies have quantified it in detail.
% The basis for this premise is a limited number of time series that exhibit
% temporal declines in a proxy of marine survival (i.e. return rates or
% post-smolt survival).

% 3. Nonetheless, this premise has never been examined in detail
Given the logistical challenges associated with estimating at-sea survival, it
is not surprising that the number of studies that have estimated temporal
trends has been limited. An additional limitation has been the derivation of
proxies (e.g., return rates), rather than direct model-based estimates, of
marine survival.
\citet{Chaput2012a}, for example, examined return rates of smolts to adult salmon to assess
declines in marine survival and found that most populations from Canadian
rivers had declining return rates. 
However, looking at trends in return rates
can mask changes in differential survival in different years at sea, as well
as changes in the proportion of adults returning after one or two years at sea.
A more recent study by \citet{Olmos2019} suggested that trends in post-smolt
survival, when estimated at the stock unit level, are synchronously declining
for all Atlantic salmon in Eastern North America and Canada. Nonetheless,
the aforementioned study by \citet{Olmos2019} relied on the veracity of stock-recruitment relationships.

%4. 
In the present study, we compile smolt and adult returns times series data for seven wild
Canadian populations of Atlantic salmon to model trends in marine survival.
While some studies have used maturity-schedule models to estimate marine
survival for a limited number of salmon populations \citep{Chaput2003b}, none
have incorporated data extending over decates, nor have they examined
trends among more than two or three populations. 
Here, we develop a hierarchical Bayesian model that uses Murphy's maturity
schedule method, in conjunction with informative priors, to estimate yearly
marine survival in salmon. In addition to accounting for observation error in
smolt and return estimates, we estimate the proportion of salmon returning
after one winter hierarchically.

\section*{Methods}

\subsection*{Data}

We obtained time series data of outmigrating smolt and returning adult
abundances for seven Atlantic salmon populations in eastern Canada, encompassing a
wide range of the species' distribution (Fig.~\ref{fig:map}). These
populations included LaHave River in the Southern Uplands region of Nova
Scotia (NS), Nashwaak River in New Brunswick (NS) which connects to the
Bay of Fundy, Trinit\'{e} and Saint-Jean Rivers in Qu\'{e}bec (QC), which
flow to the St. Lawrence River and Bay of St. Lawrence, respectively, and
Western Arm Brook (also referred to as WAB), Campbellton, and Conne Rivers in Newfoundland, which flow to the
Strait of Belle Isle, Labrador Sea, and Gulf of St. Lawrence, respectively.
Data were collected in NS, NB, and Newfoundland
and Labrador (NL) by Fisheries and Oceans Canada (DFO) and in QC
by the Minist\`{e}re des For\^{e}ts, de la Faune et des Parcs, Qu\'{e}bec.

\begin{figure}[htbp] \centering
    \includegraphics[width=0.85\linewidth]{figures/rivers-map2.png}
    \caption{Locations of the seven rivers in eastern Canada with time series abundance data of outmigrating smolts and 
    returning adults.} \label{fig:map} 
\end{figure}

\subsubsection*{Smolt data}

\comment{Co-authors: please check data sources}

Smolt estimates originate from a variety of sources.
Estimates from the XXXX populations were obtained using a capture-mark-recapture approach, while
Estimates from the XXXX populations were obtained by direct counts using a XXXX.
Refer to ... for more details on the data collection details. 

\subsubsection*{Returns data}

Yearly return data is often recorded in terms of two size groups: small ($< 63$ cm
FL) and large ($\geq 63$ cm FL) salmon, as these closely represent different
life-history strategies (i.e. 1SW and 2SW), but can be conflated with repeat
spawners of different sizes. To correct for this, we estimated the abundance
of 1SW and 2SW returns using yearly scale data of a subsample of returns:

\begin{equation}
    p_{r,y,a} = \frac{\sum_{s}{(\frac{n_{r,y,s,a}}{n_{r,y,s}} * N_{r,y,s})}}{\sum_{s}{N_{r,y,s}}}
\end{equation}

where $p_{r,y,a}$ is the proportion of annual returns in river $r$, year $y$,
and of spawning history $a$ (either 1SW or 2SW returns); $n_{r,y,s,a}$ is the
number of samples in river $r$, year $y$, of spawning history $a$, and of size
group $s$; $n_{r,y,s}$ is the total number of samples in river $r$, year $y$,
and of size group $s$; and $N_{r,y,s}$ is the returns of salmon
in river $r$, year $y$, and of size group $s$.

\subsection*{Bayesian model}

We developed a hierarchical Bayesian model that uses Murphy's maturity
schedule method, in conjunction with informative priors, to estimate yearly
marine survival in seven populations of Atlantic salmon. We account for
observation error in smolt and return estimates, as well as estimating the
proportion returning after one winter (i.e. \Pg) hierarchically.
There is an identifiability problem in the maturity schedule equations where
the parameter estimates cannot be optimally solved \citep{Chaput2003a}.
However, this issue can be mathematically overcome, at least partially, by
using informative priors for all three marine survival parameters in a
Bayesian framework.
This method requires abundance estimates of smolts as well as abundance estimates
of returning one-sea-winter (1SW) and two-sea-winter adults. With these data,
it estimates three parameters: survival in the first year at sea (\So), survival
in the second year at sea (\St), and the proportion of fish returning after one
year at sea (\Pg). 
Our model does not include repeat spawners and assumes that no fish spend
three or more winters at sea before returning to spawn for the first time.

The model also assumes that mortality in the second winter at sea 
is additional to mortality in the first winter at sea in the previous year, 
and therefore does not account for differences
the returning (Pr) occurs just before
actually being counted as returns and that S2 is any additional mortality in
the subsequent year. I agree this is far from ideal, given that we know there
is a period of a few months where 1SW returns are subject to a different
environment than those 2SW+ fishes. 

%\comment{Other assumptions...}

Observed smolt estimates were modelled hierarchically and included
observation error:

\begin{equation}
log(smolts_{obs,t,r}) = log(smolts_{true,t,r}) + \epsilon_{t,r}
\end{equation}

where $smolts_{true,t,r}$ are the true smolt abundances for year $t$ and river
$r$, and $\epsilon_{t,r}$ is the error term, which is empirically derived from
by calculating the yearly coefficient of variation in the empirically derived
smolt estimates (see Table S1 in the Supplementary material). 
The log-transformed true smolt abundances are
normally distributed around a population-level mean and standard deviation:

\begin{equation}
log(smolts_{true,t,r}) \sim Normal(\mu_{smolts,r}, \sigma_{smolts,r})
\end{equation}

where $\mu_{smolts,r}$ and $\sigma_{smolts,r}$ are parameters estimated by the
model for each river.

Once we have yearly estimates of smolt, 1SW, and 2SW abundances, we estimate
marine survival parameters using Murphy's maturity schedule method
\citep{Murphy1952, Ricker1975}:

\begin{align}
    R_{1,t} &= smolts_{t-1} * S1_t * Pr_t \label{eq:1}, \\
    R_{2,t+1} &= smolts_{t-1} * S1_t * (1 - Pr_t) * S2_{t+1} \label{eq:2}
\end{align}

where $R_{1,t}$ and $R_{2,t+1}$ are the estimated abundances of 1SW and 2SW
salmon returning in years $t$ and $t+1$, respectively, $smolts_{t-1}$ is the
number of outmigrating smolts in year $t-1$, $S1_t$ is the proportion of
salmon surviving in their first year ($t$) at sea, $Pr_t$ is the proportion of
salmon that return to spawn at year $t$, $S2_{t+1}$ is the survival in their
second year at sea of the same cohort of salmon who did not return to spawn at
year $t$.

However, to allow for normally and log-normally-distributed parameters we
log-transform equations~\ref{eq:1} and~\ref{eq:2} to obtain

\begin{align}
    log(R_{1,r,t}) &= log(smolts_{r,t-1}) + log(Pr_{r,t}) - Z_{1,r,t} + \epsilon_{1,r} \label{eq:3}, \\
    log(R_{2,r,t+1}) &= log(smolts_{r,t-1}) - Z_{1,r,t} + log(1 - Pr_{r,t})  - Z_{2,r,t+1} + \epsilon_{2,r} \label{eq:4} 
\end{align}

where $Z_{1,r,t}$ and $Z_{2,r,t+1}$ are the instantaneous mortality rates and $\epsilon_{1,r}$ and $\epsilon_{2,r}$
are the process error terms. 
These error terms are normally distributed with a standard deviation
of 0.01, which are almost equivalent to using a coefficient of variation in
return estimates 1\%, given that return estimates in equations~\ref{eq:3}
and~\ref{eq:4} are log-transformed:

\begin{align}
\epsilon_{1,r} &\sim Normal(0, 0.01) \\
\epsilon_{2,r} &\sim Normal(0, 0.01)
\end{align}

Furthermore, we use instantaneous mortality rates in the model instead of survival probabilities
as the model is more efficient in its parameter search in log-space, and instantaneous rates
are easy to interpret. Instantaneous rates are easily converted to survival probabilities by 

\begin{align}
 S_{1} &= e^{-Z_1}, \\
 S_{2} &= e^{-Z_2} 
\end{align}

We estimate population-level mean \Pg values around which the yearly \Pg
values are normally distributed. We specify different informative hyperpriors
for \prmu and \prsig based on whether the population is 1SW-dominated or not:

\begin{align}
    logit(P_{r,t}) &\sim Normal(\mu_r, \sigma_r) \\
    \mu_r &\sim 
    \begin{cases}
       Normal(2.3, 0.4),  &\text{for 1SW-dominated populations} \\
       Normal(0, 2.8), &\text{for non-1SW-dominated populations} \\
   \end{cases} \\
    \sigma_r &\sim halfNormal(0, 1)
\end{align}

The priors for \So and \St are specified as log-normal distributions around
the instantaneous mortality rates $Z_1$ and $Z_2$:

\begin{align}
Z_1 &\sim logNormal(1, 0.22),   \\ 
Z_2 &\sim logNormal(0.2, 0.3)
\end{align}

The model was written in Stan \citep{Carpenter2017} and run in R version 3.6.1
\citep{RCoreTeam2019} using the \texttt{rstan} package version 2.19.2
\citep{StanDevelopmentTeam2019}.
The model was run with three chains and 3,000 iterations, with the first 1,500
discarded as a burn-in. The models were considered to have converged when the
$\hat R$ of all parameters were lower than 1.03 and the effective sample size 
were higher than 500.

\subsection*{Correlations among trends in survival}

%\comment{How do were quantitatively compare trends among rivers? DFA is an
%option, but so is a t-test of averages before and after a certain year}

We looked at the correlation of trends in \So by calculating the Pearson's
correlation coefficient for the Z-scores of these trends. Given that the time
series do not cover the same years, and that some rivers have missing years in
the middle of the time series, pair-wise Pearson's tests were done using only
the years where there is data for both rivers.

\section*{Results}

%\subsubsection*{Model convergence}

\subsubsection*{Trends in marine survival parameters}

Trends in estimates of \So were highly variable within among rivers
(Fig.~\ref{fig:s1-dual}). The highest median posterior estimates of \So
were for the Nashwaak River in 2006 and 2008, with values of 0.18 and 0.21,
respectively. The lowest median \So estimate was in the Trinit\'{e} in 2001,
with an estimate of \So of 0.007, while the Conne, LaHave, and Trinit\'{e} had
years of estimates of \So between 0.01 and 0.02 (Fig.~\ref{fig:s1-faceted}).

Trends among populations were variable as well: roughly, the Campbellton,
Saint-Jean, and Western Arm Brook populations showed increases in \So
over time, the Trinit\'{e}, Conne, and La Have populations showing decreases,
and the Nashwaak population showing an increase in median \So during the early
2000s but a decrease in the 2010s. Yearly estimates of \So had very little
variability for one sea-winter dominated populations (Conne, Campbellton, and
WAB), but were more variable (i.e. wider credible intervals) in the other
populations.

\begin{figure}[htbp] \centering
    \includegraphics[width=0.95\linewidth]{figures/s1-trends-dual.png}
    \caption{Trends in survival in the first year at sea (\So). a) Posterior
        estimates of \So for the seven rivers examined, error bars indicate
        the 90\% credible intervals. b) Z-scores of median posterior
        estimates.} \label{fig:s1-dual} 
\end{figure}

\begin{figure}[htbp] \centering
    \includegraphics[width=0.95\linewidth]{figures/s1-trends-faceted.png}
    \caption{Posterior estimates of \So for the seven rivers examined, error
        bars indicate the 90\% credible intervals.} \label{fig:s1-faceted}
\end{figure}

Estimates of \St were highly uncertain in all rivers, and trends of any kind
were not apparent in most rivers given the large range of the credible
intervals in the yearly estimates (Fig.~\ref{fig:s2-faceted}). The estimates
of \St for the Saint-Jean and Trinit\'{e} were considerably less uncertain
than for the other populations, but showed no apparent trends through time.

\begin{figure}[htbp] \centering
    \includegraphics[width=0.95\linewidth]{figures/s2-trends-faceted.png}
    \caption{Posterior estimates of \St for the seven rivers examined, error
        bars indicate the 90\% credible intervals.} \label{fig:s2-faceted}
\end{figure}

Estimates of \Pg were mostly stable across time, except for the LaHave and
Nashwaak Rivers; The estimates of \Pg were slightly lower in the last four
years than in the previous ones, while in the Nashwaak the posterior estimates
of \Pg in 2012 were much lower than in all other years
(Fig.~\ref{fig:s2-faceted}). Uncertainty in yearly estimates was highest in
the LaHave, Nashwaak, and Trinit\'{e}, and lowest in the 1SW-dominated
populations.

\begin{figure}[htbp] \centering
    \includegraphics[width=0.95\linewidth]{figures/pr-trends-faceted.png}
    \caption{Posterior estimates of \Pg for the seven rivers examined, error
        bars indicate the 90\% credible intervals.} \label{fig:pr-faceted}
\end{figure}

Population-level estimates of \prmu and \prsig varied considerably among rivers. For all
three 1SW-dominated rivers (Campbellton, Conne, and WAB), estimates of \prmu
were very close to 1.0 and had little variability in \prsig (Fig~\ref{fig:prmu-post}).
Estimates of \prmu were the lowest for the two QC rivers, particularly the Saint-Jean (median \prmu = 0.11).
Estimates of \prmu for the Nashwaak and the LaHave Rivers were close to 0.5, with these two rivers having 
the highest estimated values of \prsig, particularly the Nashwaak (Fig~\ref{fig:prmu-post}).

\begin{figure}[htbp] \centering
    \includegraphics[width=1.0\linewidth]{figures/pr-mu-posteriors.png}
    \caption{Posterior estimates of the population-level parameters $Pr_{\mu}$
       $logit(Pr_{\mu})$, and $logit(Pr_{\sigma})$. Dots denote median estimates, while the thick and thing error bars indicate
       the 50\% and 90\% credible intervals, respectively.} 
   \label{fig:prmu-post} 
\end{figure}

\subsubsection*{Correlations}

Most correlations between z-scored trends were not significant, with only 
three out of the 21 pair-wise comparisons having a p-value below 0.01 (Fig.~\ref{fig:s1-corr}a).
When looking at the direction of the correlation, regardless whether they were
significant or not, these spanned both positive and negative coefficients
(Fig.~\ref{fig:s1-corr}b).

\begin{figure}[htbp] \centering
    \includegraphics[width=0.95\linewidth]{figures/corr-s1.png} \caption{
        Correlation among Z-scores of median estimated trends in \So among
        rivers. a) Pearson's correlation coefficients are shown in each square,
        while colouring denotes significance of the correlation ($p \leq 0.01$), b)
        colours denote direction and magnitude of correlation, while asterisks denote significance.}
\label{fig:s1-corr} 
\end{figure}

\section*{Discussion} 

% 1. Main findings
% 2. How they compare to other publications
% 3. Potential reasons
% 4. Caveats
% 5. Future directions

% 1. Main findings
Our findings challenge the narrative that marine survival is declining
uniformly throughout the range of Atlantic salmon in the northwest Atlantic.
Temporal trends are not consistent among populations 
. Over the time periods for which data were available, some rivers show positive trends in survival in the
first winter at sea (\So) while other show stable trends, and some even show
declines. We could not assess trends in the second winter at sea (\So) or
proportion returning as grilse (\Pg), as these parameter estimates were highly
uncertain and were strongly influenced by the priors used.
Perhaps there is a need to rethink our understanding of Atlantic salmon
population dynamics in light of the possibility that any real and persistent decline in marine survival was
experienced by some but not all populations, that reductions in survival might
have occurred over a relatively brief period of time and may not longer be ongoing, and that since that
reduction occurred (for a subset of populations) marine survival has
been stable, or even increasing.

% 2. How they compare to other publications
Our findings are contrary to those of \citet{Olmos2019}, who detected positive
correlations in post-smolt survival among spatially broad stock units. These differences
could be due to a number of reasons: different model specifications and
structure, different methods for estimating covariance, difference in the
spatial scales of data sources, and the use stock-recruitment relationships
rather than empirical smolt observations to estimate marine survival.
Interestingly, our estimates of \So and \St are very similar to those produced
by \citet{Chaput2003b}, and our trends are almost identical for the
overlapping time period that marine survival was estimated for in their study
(1984-1998). 
While \citet{Chaput2003b} separated abundance data for males and females
and assumed their survival rates were the same (to be able to reach an
analytical solution), our study reached almost the same results (albeit with
slightly higher uncertainty) using a Bayesian approach with informative
priors. These overlapping trends obtained with two different methods 
suggest that our method is effective at estimating marine survival.

Trends in marine survival
among populations were compared by \citet{Chaput2012a} using adult return rates, 
and found that for four out of the six population examined, return rates in the 1990s 
were less than those during the 1970s.
\citet{Friedland1993} compared return rates for a number rivers in eastern
North America between 1973 and 1988, and suggested there are similar trends among these. 
However, the similarity in trends is driven primarily by two years, 1977 and 1978, which
show concurrent low and high relative return rates across rivers,
respectively. Others years are much more variable relative to each other.
\citeauthor{Friedland1993}'s \citeyear{Friedland1993} time series stops at
1988, thus there are only a few years for which to assess overlap with the
time series in our study.
However, the pooling of adult return rates, as done by \citet{Chaput2012a} and
\citet{Friedland1993} can mask inter-annual variability in marine survival,
and hence might not produce an accurate depiction of marine survival trends.
\citet{Dempson2003} describe a general declining trend in marine survival for
Newfoundland rivers (except WAB); which in our study we observe in the
Conne but not in Campbellton River or WAB. It is not possible to draw conclusions
with data from only three Newfoundland rivers, but it seems that among index rivers,
those in Newfoundland are among those with the highest marine survival rates.

% 3. Potential reasons
There are multiple potential explanations for the lack of synchronous
trends in estimates of \So. 
Marine survival in the first winter at sea could be highly variable between
populations because of the predominance of spatially local environmental drivers of survival (e.g., temperature, predation) 
relative to broader-scale, even ocean-wide, drivers.
The synchrony reported for marine survival trends at broader spatial scales \citep{Olmos2019}
might be attributable to the use of stock-recruitment relationships to estimate survival,
relationships thay may have been confounded by changes in recruitment dynamics.

There is some evidence of a correlation between return rate and growth (as
indicated by intercirculi spacing on scales), where years of poor growth
tended to also be years of poor survival \citep{Friedland1993}, supporting the
idea that environmental variability can affect marine survival.
Furthermore, among European salmon, there is evidence of a positive correlation
between spring temperature in the Norwegian and North Seas and population abundance, suggesting warmer
conditions favour post-smolts \citep{Friedland1998} based on mapping the
extent of area of suitable temperature (7-13 \textdegree C).

Nonetheless, the causal mechanisms for why warming should affect post-smolt
survival almost certainly differs depending on the difference between
temperature experienced by the post-smolts and their respective
population-specific thermal optima, which could explain why populations in eastern North America are declining
in the southern part of their range but potentially increasing further north.

A potential association between temperature and survival warrants
further study.

While there is little evidence that marine survival is density-dependent in
Atlantic salmon \citep{Jonsson1998,Gibson2006}, there could potentially be
some density-dependent processes during parts of the post-smolt migration
period, particularly for populations that are likely to be subjected to
declining per capita population growth rates generated by Allee effects.
Exploring relationships between survival and population size could potentially
shed light about the processes that have caused many of the population
declines that have been documented.

% 4. Caveats
As with all novel modelling approaches, there are caveats to acknowledge.
While there are analytical issues with estimation of \So, \St, and \Pg,
the assumption that \St is additive to \So could produce unrealistic results.
Overcoming this assumption would require an additional parameter to be
estimated, or an additional assumption as to what proportion of \So is not
additive to \St (as the returning 1SW adults do not experience the same
environment when they return to their natal streams as those fish who stayed
at sea for an additional winter before returning to spawn).
Secondly, the assumed hierarchical structure might not be the most appropriate
for modelling smolt abundances. This approach results in shrinkage to the
mean, which means that the variability of yearly smolt estimates is less than
it would be if not modelled hierarchically. 
That said, this assumption is likely to result in more conservative trends in marine survival, as the
variation is smolt estimates is reduced.
It is important to note that the hierarchical structure in the estimation of
\Pg seems like a reasonable assumption given that the probability of returning
as grilse has a genetic component associated to it \citep{Aykanat2019} and is
not expected to vary much, within a population, among years.

% 5. Future directions
Perhaps a reframing of the issue of marine survival is key to furthering our
understanding of Atlantic salmon population dynamics. Marine
survival may not have declined consistently, and over the same time periods, across across populations, but the fact that it has stayed at
roughly similar levels as it was previously \emph{despite} reduced commercial fishing mortalities,
and that many populations are smaller in size compared to what they once were, is
what we should focus on understanding better.
Given the literature of correlates on marine survival in the past, if survival
has been stable or increasing, this clearly warrants reexamination of the
correlates and causes of temporal and spatial shifts in marine survival.



\section*{Acknowledgements}

% We would like to thank Geir Bolstad, G\'{e}rald Chaput, Brian Dempson, and Martha
% Robertson for their useful discussions on estimating marine survival in
% Atlantic salmon, and Amanda Kissel for her helpful comments on the manuscript.
% Brian Dempson and Geoff Venoit for providing the data Conne River data.
We would like to thank G\'{e}rald Chaput for his useful discussions on
estimating marine survival in Atlantic salmon, and Sean Anderson for his help
with implementing the non-centered parameterization of the model.
This research was supported by the Atlantic Salmon Conservation Foundation and
the Atlantic Salmon Research Joint Venture.

\section*{Conflicts of Interest}

The authors declare no conflicts of interest.
 
\bibliography{subset}

%\documentclass[12pt]{article}
\usepackage[top=0.85in,left=1.0in,right=1.0in,footskip=0.75in]{geometry}
%\usepackage[parfill]{parskip}
\usepackage{setspace}
\usepackage{lineno}
\usepackage[hidelinks]{hyperref}
%\onehalfspacing
\doublespacing

\usepackage[round,sectionbib]{natbib}
\setcitestyle{authoryear}
\bibpunct{(}{)}{;}{a}{}{;}
\bibliographystyle{fishfishnourl}

%\usepackage{textcomp}
%\usepackage{libertine}
%\usepackage{inconsolata} % sans serif typewriter

\usepackage{mathtools} % for dcases
\usepackage{xcolor} % for textcolor
\usepackage{makecell} % for \makecell and within cell line breaks (\thead)

\usepackage[T1]{fontenc}

%\makeatletter\let\expandableinput\@@input\makeatother % expandable input for \input inside tables

% Linux Libertine:
\usepackage{textcomp}
\usepackage[sb]{libertine}
\usepackage[varqu,varl]{inconsolata}% sans serif typewriter
\usepackage[libertine,bigdelims,vvarbb]{newtxmath} % bb from STIX
\usepackage[vvarbb]{newtxmath} % bb from STIX, removed bigdelims for ScholarOne rendering
\usepackage[cal=boondoxo]{mathalfa} % mathcal
%\useosf % osf for text, not math
%\usepackage[supstfm=libertinesups,%
%  supscaled=1.2,%
%  raised=-.13em]{superiors}

\usepackage{xspace}
\usepackage{xfrac} % for diagonal inline fractions in text
%\usepackage{array} % for making whole row bold in table
\usepackage{colortbl} % for background colours in table rows
\usepackage{longtable}
\usepackage{amssymb} % for \checkmark 
\usepackage{amsmath} % for \checkmark 
\usepackage{rotating}
\usepackage[nolists,tablesfirst]{endfloat} % for putting figs and tables at end of document
\DeclareDelayedFloatFlavor{sidewaystable}{table}
\usepackage{makecell} % for \makecell in tables
\usepackage{doi}

%\usepackage{xr} % to obtain label references from supp materials file
%\externaldocument[S-]{suppmat}

\usepackage{tabularx}
\usepackage{booktabs}
\usepackage{array} % for table wrapping of columns
\newcolumntype{L}[1]{>{\raggedright\let\newline\\\arraybackslash\hspace{0pt}}m{#1}}
\newcolumntype{C}[1]{>{\centering\let\newline\\\arraybackslash\hspace{0pt}}m{#1}}
\newcolumntype{R}[1]{>{\raggedleft\let\newline\\\arraybackslash\hspace{0pt}}m{#1}}
%\usepackage[detect-all]{siunitx} % for SI units

% custom hyphenation:
\hyphenation{inverse}
\hyphenation{At-lantic}
\hyphenation{elasmo-branchs}


% Macros
\newcommand{\So}{$S_{1}$\xspace}
\newcommand{\St}{$S_{2}$\xspace}
\newcommand{\Pg}{$P_r$\xspace}
\newcommand{\prmu}{$\mu_r$\xspace}
\newcommand{\prsig}{$\sigma_r$\xspace}
\newcommand{\Linf}{$L_{\infty}$}
\newcommand{\DWinf}{$DW_{\infty}$}
\newcommand{\alphat}{$\tilde{\alpha}$}
\newcommand{\lamat}{$l_{\alpha_{mat}}$}
\newcommand{\lamatb}{$l_{\alpha_{mat}}b$}
\newcommand{\rmax}{$r_{max}$\xspace}
\newcommand{\ageratio}{$\alpha_{mat}/\alpha_{max}$}
\newcommand{\yr}{year\textsuperscript{-1}}
\newcommand{\rsq}{$R^2$\xspace}
\newcommand{\mytilde}{\raise.17ex\hbox{$\scriptstyle\mathtt{\sim}$}}
% Select what to do with command \comment:  
%\newcommand{\comment}[1]{}  % comment not shown
\newcommand{\comment}[1]{\par {\bfseries \color{blue} #1 \par}} % comment shown
%% END MACROS SECTION

\begin{document}
\linenumbers


\section*{Trends in marine survival of Atlantic salmon in eastern Canada}

\textbf{Sebasti\'{a}n A. Pardo\textsuperscript{1*}, 
        Geir H. Bolstad\textsuperscript{2}, 
        J. Brian Dempson\textsuperscript{3}, 
        Julien April\textsuperscript{4}, 
        Ross A. Jones\textsuperscript{5}, 
        Martha J. Robertson\textsuperscript{3}, 
        Dustin Raab\textsuperscript{6}, 
Jeffrey A. Hutchings\textsuperscript{1}} 

\noindent\small{\textsuperscript{1} Department of Biology, Dalhousie University, Halifax, NS, Canada\\}
\small{\textsuperscript{2} Norwegian Institute for Nature Research (NINA), Trondheim, Norway\\}
\small{\textsuperscript{3} Fisheries and Oceans Canada, St. John's, NL, Canada\\}
\small{\textsuperscript{4} Minist\`{e}re des For\^{e}ts, de la Faune et des Parcs, Qu\'{e}bec, QC, Canada\\}
\small{\textsuperscript{5} Fisheries and Oceans Canada, Moncton, NB, Canada\\}
\small{\textsuperscript{6} Fisheries and Oceans Canada, Dartmouth, NS, Canada\\}
\small{\textsuperscript{*} Corresponding author: spardo@dal.ca}

\section*{Abstract}

Declines in wild Atlantic salmon (\emph{Salmo salar}) throughout the north
Atlantic are primarily attributed to declining survival at sea. This
hypothesis has proven challenging to test on a river-by-river basis because of
the need to model data on both migrating smolts and returning adults and to
simultaneously estimate multiple parameters, especially for salmon spending
more than one winter at sea (1SW) before spawning. We fit a hierarchical Bayesian
maturity schedule model to data (19-42 years) for seven populations in  
Newfoundland and Labrador (NL), Qu\'{e}bec (QC), New Brunswick (NB), and Nova
Scotia (NS), Canada. We estimate survival in the first (\So) and second year at sea (\St),
and the proportion returning as 1SW adults (\Pg). Trends in  were not consistent
among rivers. Since 1990, \So increased at Western Arm Brook (NL)
and Rivi\`{e}re Saint-Jean (QC), but declined at Conne River (NL) and Rivière
de la Trinité (QC). Since the mid-1990s, \So increased at Campbellton River (NL),
declined at LaHave River (NS), and fluctuated at Nashwaak
River (NB). Estimates of \St were highly uncertain, particularly for
1SW-dominated populations; \Pg was generally stable. These results challenge the
narrative that marine survival has changed in a temporally consistent manner
among spatially disparate populations. Our findings suggest that, at the
population level, changes in abundance are attributable to temporal shifts in
multiple components of individual fitness, including at-sea survival. If
salmon populations do not respond in a consistent, uniform manner to changing
environmental conditions throughout their range, future research initiatives
should explore why.

% The marine phase of anadromous Atlantic salmon (\emph{Salmo salar}) is the
% least known yet one of the most crucial with regards to population
% persistence. Declines in many Atlantic salmon populations in eastern
% Canada have often been attributed to changes in conditions at sea, negatively
% affecting their survival. However, marine survival estimates are difficult to
% obtain given that many individuals spend multiple winters in the ocean before
% returning to freshwater to spawn, necessitating the estimation of multiple
% parameters. To do so, we fit a hierarchical Bayesian maturity schedule model
% to smolt and adult abundance time series for seven populations located in
% Newfoundland and Labrador (NL), Qu\'{e}bec (QC), New Brunswick (NB), and Nova
% Scotia (NS). The datasets ranged between 19 and 42 years in length. We
% estimated three components of marine survival: survival in the first year at
% sea (\So), survival in the second year at sea (\St), and proportion returning
% as one sea-winter adults (\Pg). Controlling for time frame, trends in
% estimates of \So were not consistent among rivers.
% In the four populations for which data extended to 1990, marine survival
% during the first year at sea predominantly increased over time in Western Arm
% Brook (NL) and Rivi\`{e}re Saint-Jean (QC) but largely declined in Conne River
% (NL) and Rivi\`{e}re Trinit\'{e} (QC). In the three other populations, \So
% exhibited an increase in Campbellton River (NL), a decline in LaHave River
% (NS), and fluctuating stability in Nashwaak River (NB) since approximately the
% mid-1990s. 
% Estimates of \St were highly uncertain, particularly for 1SW-dominated rivers
% (Conne, Campbellton and WAB, where the abundance of 2SW returns is very low)
% and thus the posterior distributions matched our choice of prior and trends
% could not be assessed. 
% \Pg was temporally stable within rivers, except for LaHave and Nashwaak Rivers. 
% Our findings challenge the assumption that marine survival has changed in a consistent
% manner among populations across a broad geographic scale, and suggest that
% trends can be river-specific. 
% The well established correlations between climate variables and abundance are
% not being mediated solely by marine survival, and there can potentially be
% some important indirect effects on fecundity.
% Further work exploring potential correlates of marine survival and the
% potential non-lethal effects of climate effects is warranted.


Keywords: salmonid, survival at sea, natural mortality, marine mortality

%Running Head: 

\section*{Introduction} % (4-5 paragraphs)

%1. Declining salmon numbers

Reductions in fishing mortality, albeit necessary, are not always sufficient
to facilitate population recovery. Experience with numerous commercially
exploited marine fisheries since the early 1990s has shown that not all
populations respond as favourably as anticipated to major reductions in
exploitation \citep{Hutchings2017}. Gradual efforts to close commercial
Atlantic salmon (\emph{Salmo salar}) fisheries in eastern Canada culminated in
full moratoria in all regions, beginning in the Maritime provinces (1984) and
following in Newfoundland (1992), Labrador (1998), and Qu\'{e}bec (2000). Since
these closures, many populations have not increased as 
expected \citep{Dempson2004, ICES2019}; some 
have been assessed as considered threatened or endangered by the 
Committee on the Status of Endangered Wildlife in Canada \citep[][]{Cosewic2010}, 
Canada's national science advisory body (to the national government) on
species risk of extinction.
While it is not fully understood what is driving population declines, the potential
drivers of these are many \citep[see ][for a detailed discussion of possible
causes]{Cairns2001}, including but not limited to: fishing mortality \citep{Dempson2004}, 
damming of waterways and changes in the freshwater habitat \citep{Dunfield1985}, acidification
\citep[particularly in the Southern Uplands region of
NS, see][]{Gibson2010}, predation by seals and birds \citep{Cairns2000}, negative
effects of interbreeding or interactions with escaped farmed salmon
\citep{Keyser2018}, and climate-driven changes in survival and productivity \citep{Mills2013}.

% 2.
Over the past three decades, a narrative has emerged that marine survival of
Atlantic salmon has declined throughout the North Atlantic \citep{ICES2019}.
%\citep{Hansen1998,OMaoileidigh2003,Chaput2012a}.
Based on multiple lines of evidence that climate conditions can directly and
indirectly influence the abundance and productivity of Atlantic salmon
populations \citep{Mills2013,Almodovar2019}, it has been presumed that oceanic climate effects are
driving population dynamics primarily through changes in marine survival.
An implicit assumption is that any trend in
survival in ocean habitat that is shared by multiple populations during their
seaward migration period will be experienced similarly. 
Put another way, given that salmon from different rivers 
are hypothesized to share marine habitat during some of their time at sea, it
has been presumed that populations share similar temporal trends in
at-sea mortality \citep{Friedland1993, Friedland1998, Russell2012}.  
A recent study by \citet{Olmos2019} suggested that trends in post-smolt
survival, when estimated at the stock level, are synchronously declining
for all Atlantic salmon in eastern North America.

% This perception is widespread both in the scientific
% literature as well as federal reports. The latest status report on Atlantic
% salmon by COSEWIC states that ``While the mechanism(s) of marine mortality is
% uncertain, what is clear is that the recent period of poor sea survival is
% occurring in parallel with many widespread changes in the North Atlantic
% ecosystem.'' \citep{Cosewic2010}. Consequently, there have several lines of
% inquiry as to what the causes behind these declines might be attributable to
% \citep{Friedland1993, Friedland1998}.


In contrast to the narrative of widespread, demographically similar increases
in at-sea mortality, the conservation status of Canadian salmon populations differs
considerably. Populations in the southern part of their range are more
likely to be assessed as being of conservation concern than those in more
northerly regions \citep{Cosewic2010}. This geographical disparity in status
suggests that if marine survival has been, or is, a key factor responsible for
most population declines, these changes are not uniformly distributed across
all populations. 

% Nonetheless, most of the populations declines assessed by COSEWIC have
% occurred in populations in the southern extent of its distribution:
% populations in the Bay of Fundy, Anticosti Island, and the Atlantic coast of
% Nova Scotia being assessed as Endangered, populations in the south coast of
% Newfoundland assessed as Threatened, And the populations in the New Brunswick
% and Qu\`{e}bec coasts of the Gulf of St. Lawrence assessed as Special Concern
% \citep{Cosewic2010}. On the other hand, northermost populations have shown
% stable, or even increasing population trends, suggesting that, if marine
% survival were to be a factor in many of these declines, these changes are not
% uniformly distributed across all populations.

% While the purported decline of marine survival is mentioned widely in the 
% scientific literature, only relatively few studies have quantified it in detail.
% The basis for this premise is a limited number of time series that exhibit
% temporal declines in a proxy of marine survival (i.e. return rates or
% post-smolt survival).

% 3. Nonetheless, this premise has never been examined in detail
Given the logistical challenges associated with estimating at-sea survival, it
is not surprising that the number of studies that have estimated temporal
trends has been limited. An additional limitation has been the derivation of
proxies (e.g., return rates), rather than direct model-based estimates, of
marine survival.
\citet{Chaput2012a}, for example, examined the return rate of smolts to adult salmon 
as a metric of marine survival, finding that most Canadian populations 
had experienced declining return rates. 
However, examination of trends in return rates alone
can mask changes in differential survival during different years at sea, as well
as changes in the proportion of adults returning after one or two years at sea.
Recently \citet{Olmos2019} suggested that trends in post-smolt
survival, when estimated at the stock unit level, are synchronously declining
for all Atlantic salmon in Eastern North America and Canada, a conclusion ultimately 
grounded on the veracity of highly variable stock-recruitment relationships.

%4. 
In the present study, we compile data on the number of migrating smolts and number of returning adults 
for seven wild Canadian populations of Atlantic salmon to model trends in marine survival.
While some studies have previously used maturity-schedule models to estimate marine
survival for a limited number of salmon populations \citep{Chaput2003b}, none
have incorporated data extending over multiple decades, nor have they examined
trends among more than two or three populations. 
Here, we develop a hierarchical Bayesian model that uses Murphy's maturity
schedule method, in conjunction with informative priors, to estimate yearly
marine survival in salmon. In addition to accounting for observation error in
smolt and return estimates, we estimate the proportion of salmon returning
after one winter hierarchically.

\section*{Methods}

\subsection*{Data}

We obtained time series data of outmigrating smolt and returning adult
abundances for seven Atlantic salmon populations in eastern Canada, encompassing a
wide range of the species' distribution (Fig.~\ref{fig:map}). 
Populations included the LaHave River in the Southern Uplands region of Nova
Scotia (NS), Nashwaak River, New Brunswick (NB), Rivi\`{e}re de la Trinit\'{e} (Trinit\'{e}) and
Saint-Jean rivers in Qu\'{e}bec (QC), and  Western Arm Brook (also referred to as  WAB), Campbellton, and
Conne River, Newfoundland (NL). 
Data were collected in NS, NB, and Newfoundland
and Labrador (NL) by Fisheries and Oceans Canada (DFO) and in QC
by the Minist\`{e}re des For\^{e}ts, de la Faune et des Parcs, Qu\'{e}bec.

\begin{figure}[htbp] \centering
    \includegraphics[width=0.85\linewidth]{figures/rivers-map2.png}
    \caption{Locations of the seven rivers in eastern Canada with time series abundance data of outmigrating smolts and 
    returning adults.} \label{fig:map} 
\end{figure}

\subsubsection*{Smolt and adult return abundance data}

Smolt and adult return abundance estimates originate from a variety of
sources. Smolt estimates from the Trinit\'{e}, Saint-Jean, and Conne populations were obtained using a
mark-recapture approach, while estimates from the ..., WAB, and Campbellton populations
were obtained by direct counts using fish counting fences.
For further details on the data collection methodologies refer to 
\citet{Dempson1991, Venoitt2018} for NL populations, 
\citet{April2018}  for QC populations,
\citet{Jones2014} for NB populations,
and \citet{Gibson2009} for NS populations. 

Yearly return data are often recorded in terms of two size groups: small ($< 63$ cm
FL) and large ($\geq 63$ cm FL) salmon, as these closely represent different
life-history strategies (i.e. 1SW and 2SW), but can be confounded with repeat
spawners of different sizes. To correct for this in returns 
reported as small and large salmon, we estimated the abundance
of 1SW and 2SW returns using yearly scale age data of a subsample of returns:

\begin{equation}
    p_{r,t,a} = \frac{\sum_{s}{(\frac{n_{r,t,s,a}}{n_{r,t,s}} * N_{r,y,s})}}{\sum_{s}{N_{r,t,s}}}
\end{equation}

where $p_{r,t,a}$ is the proportion of annual returns in river $r$, year $t$,
and of spawning history $a$ (either 1SW or 2SW returns); $n_{r,t,s,a}$ is the
number of samples in river $r$, year $t$, of spawning history $a$, and of size
group $s$; $n_{r,t,s}$ is the total number of samples in river $r$, year $t$,
and of size group $s$; and $N_{r,t,s}$ is the returns of salmon
in river $r$, year $t$, and of size group $s$.

\subsection*{Bayesian model}

We developed a hierarchical Bayesian model that uses Murphy's maturity
schedule method, in conjunction with informative priors, to estimate yearly
marine survival in seven populations of Atlantic salmon. We account for
observation error in smolt and return estimates, as well as estimating the
proportion returning after one winter (i.e. \Pg) hierarchically.
There is an identifiability problem in the maturity schedule equations where
the parameter estimates cannot be optimally solved \citep{Chaput2003a}.
However, this issue can be mathematically overcome, at least partially, by
using informative priors for all three marine survival parameters in a
Bayesian framework.
This method requires abundance estimates of smolts as well as abundance estimates
of returning one-sea-winter (1SW) and two-sea-winter adults. With these data,
it estimates three parameters: survival in the first year at sea (\So), survival
in the second year at sea (\St), and the proportion of fish returning after one
year at sea (\Pg). 

Our model does not include repeat spawners and assumes that no fish spend
three or more winters at sea before returning to spawn for the first time.
The model also assumes that mortality in the second winter at sea (\St)
is additional to mortality in the first winter at sea in the previous year, 
and therefore does not account for differences in environmental conditions experienced
between 1SW and 2SW fish of the same smolt cohort during their overlapping first year at sea.
In other words, our model assumes that the decision of returning occurs just before
actually being counted as returns and that \St is any additional mortality in
the subsequent year. 

%\comment{Other assumptions...}

Observed smolt estimates were modelled hierarchically and included
observation error:

\begin{equation}
log(smolts_{obs,t,r}) = log(smolts_{true,t,r}) + \epsilon_{t,r}
\end{equation}

where $smolts_{true,t,r}$ are the true smolt abundances for year $t$ and river
$r$, and $\epsilon_{t,r}$ is the error term, which is empirically derived from
by calculating the yearly coefficient of variation in the empirically derived
smolt estimates (see Table S1 in the Supplementary material). 
Where available, we used population-specific measurement error estimates for smolt abundances; if not 
available, we set measurement error at 5\%. 
The log-transformed true smolt abundances are
normally distributed around a population-level mean and standard deviation:

\begin{equation}
log(smolts_{true,t,r}) \sim Normal(\mu_{smolts,r}, \sigma_{smolts,r})
\end{equation}

where $\mu_{smolts,r}$ and $\sigma_{smolts,r}$ are parameters estimated by the
model for each river.

Once we have yearly estimates of smolt, 1SW, and 2SW abundances, we estimate
marine survival parameters using Murphy's maturity schedule method
\citep{Murphy1952, Ricker1975}:

\begin{align}
    R_{1,t} &= smolts_{true,t-1} * S_{1,t} * Pr_t \label{eq:1}, \\
    R_{2,t+1} &= smolts_{true,t-1} * S_{1,t} * (1 - Pr_t) * S_{2,t+1} \label{eq:2}
\end{align}

where $R_{1,t}$ and $R_{2,t+1}$ are the estimated abundances of 1SW and 2SW
salmon returning in years $t$ and $t+1$, respectively, $smolts_{true,t-1}$ is the
estimated number of outmigrating smolts in year $t-1$, $S_{1,t}$ is the proportion of
salmon surviving in their first year ($t$) at sea, $Pr_t$ is the proportion of
salmon that return to spawn at year $t$, $S_{2,t+1}$ is the survival in their
second year at sea of the same cohort of salmon who did not return to spawn at
year $t$.

However, to allow for normally and log-normally-distributed parameters we
log-transform equations~\ref{eq:1} and~\ref{eq:2} to obtain

\begin{align}
    log(R_{1,r,t}) &= log(smolts_{r,t-1}) + log(Pr_{r,t}) - Z_{1,r,t} \label{eq:3}, \\
    log(R_{2,r,t+1}) &= log(smolts_{r,t-1}) - Z_{1,r,t} + log(1 - Pr_{r,t})  - Z_{2,r,t+1} \label{eq:4} 
\end{align}

where $Z_{1,r,t}$ and $Z_{2,r,t+1}$ are the instantaneous mortality rates.
Process error was included as the standard deviation of the log-transformed
return estimates from equation~\ref{eq:4}:

\begin{align}
log(R_{obs,1,t,r}) &\sim Normal(log(R_{1,t,r}), \epsilon_{1,r}), \\
log(R_{obs,2,t,r}) &\sim Normal(log(R_{2,t,r}), \epsilon_{2,r}) \label{eq:5} 
\end{align}

where $R_{obs,1,t,r}$ and $R_{obs,2,t,r}$ are the observed return estimates
for year $t$ and river $r$ of 1SW and 2SW fish, respectively, $\epsilon_{1,r}$
and $\epsilon_{2,r}$ are the process error terms. 
These error terms are normally distributed with a standard deviation
of 0.01, which are almost equivalent to using a coefficient of variation in
return estimates 1\%, given that return estimates in equations~\ref{eq:3}
and~\ref{eq:4} are log-transformed:

\begin{align}
\epsilon_{1,r} &\sim Normal(0, 0.01) \\
\epsilon_{2,r} &\sim Normal(0, 0.01).
\end{align}

Furthermore, we use instantaneous mortality rates in the model instead of survival probabilities
as the model is more efficient in its parameter search in log-space, and instantaneous rates
are easy to interpret. Instantaneous rates are easily converted to survival probabilities by 

\begin{align}
 S_{1} &= e^{-Z_1}, \\
 S_{2} &= e^{-Z_2}. 
\end{align}

We estimate population-level mean \Pg values around which the yearly \Pg
values are normally distributed. We specify different informative hyperpriors
for \prmu and \prsig based on whether the population is 1SW-dominated or not:

\begin{align}
    logit(P_{r,t}) &\sim Normal(\mu_r, \sigma_r) \\
    \mu_r &\sim 
    \begin{cases}
       Normal(2.3, 0.4),  &\text{for 1SW-dominated populations} \\
       Normal(0, 2.8), &\text{for non-1SW-dominated populations} \\
   \end{cases} \\
    \sigma_r &\sim halfNormal(0, 1).
\end{align}

The priors for $Z_1$ and $Z_2$ are specified as log-normal distributions:

\begin{align}
Z_1 &\sim logNormal(1, 0.22),   \\ 
Z_2 &\sim logNormal(0.2, 0.3).
\end{align}

The model was written in Stan \citep{Carpenter2017} and run in R version 3.6.1
\citep{RCoreTeam2019} using the \texttt{rstan} package version 2.19.2
\citep{StanDevelopmentTeam2019}.
The model was run with three chains and 3,000 iterations, with the first 1,500
discarded as a burn-in. The models were considered to have converged when the
$\hat R$ of all parameters were lower than 1.03 and the effective sample size 
were higher than 500.

\subsection*{Correlations among trends in survival}

%\comment{How do were quantitatively compare trends among rivers? DFA is an
%option, but so is a t-test of averages before and after a certain year}

We looked at the correlation of trends in \So by calculating the Pearson's
correlation coefficient for the Z-scores of these trends. Given that the time
series do not cover the same years, and that some rivers have missing years in
the middle of the time series, pair-wise Pearson's tests were done using only
the years where there is data for both rivers.

\section*{Results}

%\subsubsection*{Model convergence}

\subsubsection*{Trends in marine survival parameters}

Trends in estimates of \So were highly variable within and among rivers
(Fig.~\ref{fig:s1-dual}). The highest median posterior estimates of \So
were for the Nashwaak River in 2006 and 2008, with values of 0.18 and 0.21,
respectively. The lowest median \So estimate was in the Trinit\'{e} in 2001,
with an estimate of \So of 0.007, while the Conne, LaHave, and Trinit\'{e} had
years where estimates of \So varied between 0.01 and 0.02 (Fig.~\ref{fig:s1-faceted}).

Trends among populations also varied: Campbellton,
Saint-Jean, and Western Arm Brook populations showed increases in \So
over time, Trinit\'{e}, Conne, and La Have populations showed decreases,
while at Nashwaak there was an increase in median \So during the early
2000s but a decrease in the 2010s. Yearly estimates of \So had very little
variability for one sea-winter dominated populations (Conne, Campbellton, and
WAB), but were more variable (i.e. wider credible intervals) in the other
populations.

\begin{figure}[htbp] \centering
    \includegraphics[width=0.95\linewidth]{figures/s1-trends-dual.png}
    \caption{Trends in survival in the first year at sea (\So). a) Posterior
        estimates of \So for the seven rivers examined, error bars indicate
        the 90\% credible intervals, and dashed lines denote years of commercial
        fishing moratoria for each province. b) Z-scores of median posterior
        estimates.} \label{fig:s1-dual} \end{figure}

\begin{figure}[htbp] \centering
    \includegraphics[width=0.95\linewidth]{figures/s1-trends-faceted.png}
    \caption{Posterior estimates of \So for the seven rivers examined, error
        bars indicate the 90\% credible intervals.} \label{fig:s1-faceted}
\end{figure}

Estimates of \St were highly uncertain in all rivers, and trends 
were not apparent in most rivers given the large range of the credible
intervals in the yearly estimates (Fig.~\ref{fig:s2-faceted}). The estimates
of \St for the Saint-Jean and Trinit\'{e} were considerably less uncertain
than for the other populations, but showed no apparent trends through time.

\begin{figure}[htbp] \centering
    \includegraphics[width=0.95\linewidth]{figures/s2-trends-faceted.png}
    \caption{Posterior estimates of \St for the seven rivers examined, error
        bars indicate the 90\% credible intervals.} \label{fig:s2-faceted}
\end{figure}

Estimates of \Pg were mostly stable across time, except for the LaHave and
Nashwaak Rivers; The estimates of \Pg were slightly lower in the last four
years than in the previous ones, while in the Nashwaak the posterior estimates
of \Pg in 2012 were much lower than in all other years
(Fig.~\ref{fig:s2-faceted}). Uncertainty in yearly estimates was highest in
the LaHave, Nashwaak, and Trinit\'{e}, and lowest in the 1SW-dominated
populations.

\begin{figure}[htbp] \centering
    \includegraphics[width=0.95\linewidth]{figures/pr-trends-faceted.png}
    \caption{Posterior estimates of \Pg for the seven rivers examined, error
        bars indicate the 90\% credible intervals.} \label{fig:pr-faceted}
\end{figure}

Population-level estimates of \prmu and \prsig varied considerably among rivers. For all
three 1SW-dominated rivers (Campbellton, Conne, and WAB), estimates of \prmu
were very close to 1.0 and had little variability in \prsig (Fig~\ref{fig:prmu-post}).
Estimates of \prmu were the lowest for the two QC rivers, particularly the Saint-Jean (median \prmu = 0.11).
Estimates of \prmu for the Nashwaak and the LaHave Rivers were close to 0.5, with these two rivers having 
the highest estimated values of \prsig, particularly the Nashwaak (Fig~\ref{fig:prmu-post}).

\begin{figure}[htbp] \centering
    \includegraphics[width=1.0\linewidth]{figures/pr-mu-posteriors.png}
    \caption{Posterior estimates of the population-level parameters $Pr_{\mu}$
       $logit(Pr_{\mu})$, and $logit(Pr_{\sigma})$. Dots denote median estimates, while the thick and thing error bars indicate
       the 50\% and 90\% credible intervals, respectively.} 
   \label{fig:prmu-post} 
\end{figure}

\subsubsection*{Correlations}

Most correlations between z-scored trends were not statistically significant, with only 
three out of the 21 pair-wise comparisons having a p-value below 0.01 (Fig.~\ref{fig:s1-corr}a).
When looking at the direction of the correlation, regardless whether they were
significant or not, these spanned both positive and negative coefficients
(Fig.~\ref{fig:s1-corr}b).

\begin{figure}[htbp] \centering
    \includegraphics[width=0.95\linewidth]{figures/corr-s1.png} \caption{
        Correlation among Z-scores of median estimated trends in \So among
        rivers. a) Pearson's correlation coefficients are shown in each square,
        while colouring denotes significance of the correlation ($p \leq 0.01$), b)
        colours denote direction and magnitude of correlation, while asterisks denote significance.}
\label{fig:s1-corr} 
\end{figure}

\section*{Discussion} 

% 1. Main findings
% 2. How they compare to other publications
% 3. Potential reasons
% 4. Caveats
% 5. Future directions

% 1. Main findings
Our results challenge the narrative that marine survival, specifically survival in the first year at sea, is declining
uniformly throughout the range of Atlantic salmon in the northwest Atlantic.
Temporal trends are not consistent among populations. 
Over the time periods for which data were available, some rivers show positive trends in survival in the
first winter at sea (\So) while other exhibit highly variable yet stable trends, and some show
declines. We could not assess trends in the second winter at sea (\St) or
proportion returning as grilse (\Pg), as these parameter estimates were highly
uncertain and were strongly influenced by the priors.
Perhaps there is a need to rethink our understanding of Atlantic salmon
population dynamics in light of the possibilities that (1) any real and persistent decline in marine survival was
experienced by some but not necessarily all populations, (2) reductions in survival might
have occurred over a relatively brief period of time and have not persisted, and (3) marine survival has
been relatively stable, or increasing in some populations for one or more decades.

% 2. How they compare to other publications
Our results are contrary to those of \citet{Olmos2019}, who detected positive
correlations in post-smolt survival among spatially broad stock units. These differences
could be due to a number of reasons: different model specifications and
structure, different methods for estimating covariance, difference in the
spatial scales of data sources (i.e. river vs province scales), and perhaps most importantly the use stock-recruitment relationships
rather than empirical smolt count data to estimate marine survival.
As trends in marine survival during the first winter at sea are highly independent
among rivers on relatively small spatial scale, trends from broader
geographical areas (i.e. province, state, or country-wide estimates) may not
be representative.
Interestingly, our estimates of \So and \St are very similar to those produced
by \citet{Chaput2003b}, and our trends are almost identical for the
overlapping time period that marine survival was estimated for in their study
(1984-1998). 
While \citet{Chaput2003b} separated abundance data for males and females
and assumed their survival rates were the same (to be able to reach an
analytical solution), our study reached almost the same results (albeit with
slightly higher uncertainty), using a Bayesian approach with informative
priors. These overlapping trends obtained with two different methods 
suggest that our method is effective at estimating marine survival.

Trends in marine survival
among populations were compared by \citet{Chaput2012a} using adult return rates.
He found that for 4 of 6 populations examined, return rates in the 1990s 
were lower than those during the 1970s.
\citet{Friedland1993} compared return rates for a number of rivers in eastern
North America between 1973 and 1988, and suggested there are similar trends among these. 
However, the similarity in these trends was driven primarily by two years, 1977 and 1978, which
show concurrent low and high relative return rates across rivers,
respectively. Other years are much more variable relative to each other.
\citeauthor{Friedland1993}'s \citeyear{Friedland1993} time series ends in  
1988; thus there are only a few years for which to assess overlap with the
time series in our study.
In any event, we caution that the pooling of adult return rates \citep{Chaput2012a, Friedland1993} 
can mask inter-annual variability in marine survival,
and hence might not produce an accurate depiction of marine survival trends.
\citet{Dempson2003} described a general declining trend in marine survival for
Newfoundland rivers (except WAB); we drew the same conclusion for 
Conne River but not Campbellton River or WAB. It is not possible to draw broader conclusions
with data from only three Newfoundland rivers, but it seems that among index rivers,
those in Newfoundland are among those with the highest marine survival rates.

% 3. Potential reasons
There are a variety of potential explanations for the lack of synchronous
trends in estimates of \So. 
Marine survival in the first winter at sea could be highly variable between
populations because of the predominance of spatially local environmental drivers of survival (e.g., temperature, predation) 
relative to broader-scale, even ocean-wide, drivers.
The synchrony reported for marine survival trends at broader spatial scales \citep{Olmos2019}
might be attributable to the use of stock-recruitment relationships to estimate survival,
relationships that may have been confounded by changes in recruitment dynamics.
There is some evidence of a correlation between return rate and growth (as
indicated by inter-circuli spacing on scales), where years of poor growth
tended to also be years of poor survival \citep{Friedland1993}, supporting the
idea that environmental variability can affect marine survival.
Furthermore, among European salmon, there is evidence of a positive correlation
between spring temperature in the Norwegian and North Seas and population abundance, suggesting warmer
conditions favour post-smolts \citep{Friedland1998}, based on mapping the
extent of area of suitable temperature (7-13 \textdegree C).

Nonetheless, the causal mechanisms for why warming should affect post-smolt
survival almost certainly differs depending on the difference between
temperature experienced by the post-smolts and their respective
population-specific thermal optima. 
This difference could explain why populations in eastern North America are
declining in the southern part of their range but potentially increasing
further north, and also why some studies find positive correlations between
temperature and abundance \citep{Friedland1998, Friedland1998b, Jonsson2004}
while others find negative ones \citep{Friedland1993, Todd2008}.
Putative associations between temperature and direct estimates of marine
survival warrants further study at the population level.

While there is little evidence that marine survival is density-dependent in
Atlantic salmon \citep{Jonsson1998,Gibson2006}, there could potentially be
some density-dependent processes during parts of the post-smolt migration
period, particularly for populations that are likely to be subjected to
declining per capita population growth rates ($r$) generated by Allee effects.
Exploring relationships between survival and population size could potentially
shed light about the processes that have caused many of the population
declines that have been documented.

Oceanic conditions have been correlated with abundance trends and growth
\citep{Todd2008}, however, the mechanism by which such bottom-up effects, \
mediated by changes in food availability,
affect population dynamics beyond marine survival needs to
be thoroughly reassessed. If marine survival on its own cannot fully explain
trends in abundance, then there are potential carry-on effects of oceanic
conditions that manifest with regards to fresh production. 
For example, adults
that return to spawn after spending suboptimal conditions at sea might be less
likely to make it to their spawning grounds, successfully secure a mate,
produce fewer eggs, or produce eggs with lower per capita fitness than those
produced by adults which grew in optimal oceanic conditions.
As larger females tend to be more productive, in terms of fecundity and total
reproductive energy, than the same weight's worth of smaller females
\citep{Barneche2018}, a small decrease in body condition resulting from bottom-up
impacts on food availability could potentially have disproportionate effects on fecundity
and fitness of the offspring.

Egg-to-smolt survival in Atlantic salmon is highly variable \citep{Klemetsen2003,Chaput2015}
and changes in the oceanic conditions that spawners experience could be
contributing to this variability.
Obviously, there would be a time lag (perhaps as much as a generation) in how such effects
might be manifest at the adult stage.
However, given that most correlations are between relatively monotonic declines
in abundance coupled also monotonic increases in climatic indices
over decadal time scales \citep[e.g.,][]{Friedland1998, Todd2008,
    Beaugrand2012}, it would be expected that this correlations would be
maintained even if salmon abundances were lagged by a generation length.


% 4. Caveats
As with all novel modelling approaches, there are caveats to acknowledge.
The seven populations explored in the present study might not be representative
of regional trends in marine survival. However, there are no other
long-term time series of smolts and adult returns to draw inferences from.
While there are analytical issues associated with the estimation of \So, \St, and \Pg,
the assumption that \St is additive to \So could produce unrealistic results.
We know there is a period of a few months where 1SW
returns are subject to a different environment than those salmon that will
return as 2SW the next year. 
While this is not ideal,
overcoming this assumption would require an additional parameter to be
estimated, or an additional assumption as to what proportion of \So is not
additive to \St (as the returning 1SW adults do not experience the same
environment when they return to their natal streams as those fish who stayed
at sea for an additional winter before returning to spawn).

Secondly, the assumed hierarchical structure might not be the most appropriate
for modelling smolt abundances. This approach results in shrinkage to the
mean, which means that the variability of yearly smolt estimates is less than
it would be if not modelled hierarchically. 
That said, this assumption is likely to result in more conservative trends in marine survival, as the
variation is smolt estimates is reduced.
It is important to note that the hierarchical structure in the estimation of
\Pg seems like a reasonable assumption given that the probability of returning
as grilse has a genetic component associated to it \citep{Aykanat2019} and is
not expected to vary much, within a population, among years.

% 5. Future directions
Perhaps a reframing of the issue of marine survival is key to furthering our
understanding of Atlantic salmon population dynamics. Marine survival may not
have declined consistently, and over the same time periods, across all
populations. 
But the fact that it has remained at roughly similar levels as it was
previously \emph{despite} reduced commercial fishing mortalities, suggests
that there may well be an interaction between small population size (small
relative to unfished population size or carrying capacity), recovery
potential, and environmental stochasticity that has not been fully explored in
Atlantic salmon. 
All else being equal, relatively small populations are more vulnerable to
demographic, environmental, and genetic stochasticity than large populations
\citep{Lande1993, Hutchings2015}. Interactions between population size and the
demographic consequences of environmental stochasticity appear to have
affected recovery in many marine fishes that have exhibited impaired recovery
since mitigation of the threat posed by fishing mortality
\citep{Hutchings2017, Hutchings2020}. The possibility that similar
interactions may be impairing the recovery of wild Atlantic salmon merits
study.

\section*{Acknowledgements}

% We would like to thank Geir Bolstad, G\'{e}rald Chaput, Brian Dempson, and
% Martha Robertson for their useful discussions on estimating marine survival
% in Atlantic salmon, and Amanda Kissel for her helpful comments on the
% manuscript. Brian Dempson and Geoff Venoit for providing the data Conne
% River data.
We would like to thank G\'{e}rald Chaput for his useful discussions on
estimating marine survival in Atlantic salmon, Carmen David for her comments
on the manuscript, and Sean Anderson for his help with implementing the
non-centered parameterization of the model. This research was supported by the
Atlantic Salmon Conservation Foundation and the Atlantic Salmon Research Joint
Venture.

\section*{Conflicts of Interest}

The authors declare no conflicts of interest.
 
\bibliography{subset}

%\documentclass[12pt]{article}
\usepackage[top=0.85in,left=1.0in,right=1.0in,footskip=0.75in]{geometry}
%\usepackage[parfill]{parskip}
\usepackage{setspace}
\usepackage{lineno}
\usepackage[hidelinks]{hyperref}
%\onehalfspacing
\doublespacing

\usepackage[round,sectionbib]{natbib}
\setcitestyle{authoryear}
\bibpunct{(}{)}{;}{a}{}{;}
\bibliographystyle{fishfishnourl}

%\usepackage{textcomp}
%\usepackage{libertine}
%\usepackage{inconsolata} % sans serif typewriter

\usepackage{mathtools} % for dcases
\usepackage{xcolor} % for textcolor
\usepackage{makecell} % for \makecell and within cell line breaks (\thead)

\usepackage[T1]{fontenc}

%\makeatletter\let\expandableinput\@@input\makeatother % expandable input for \input inside tables

% Linux Libertine:
\usepackage{textcomp}
\usepackage[sb]{libertine}
\usepackage[varqu,varl]{inconsolata}% sans serif typewriter
\usepackage[libertine,bigdelims,vvarbb]{newtxmath} % bb from STIX
\usepackage[vvarbb]{newtxmath} % bb from STIX, removed bigdelims for ScholarOne rendering
\usepackage[cal=boondoxo]{mathalfa} % mathcal
%\useosf % osf for text, not math
%\usepackage[supstfm=libertinesups,%
%  supscaled=1.2,%
%  raised=-.13em]{superiors}

\usepackage{xspace}
\usepackage{xfrac} % for diagonal inline fractions in text
%\usepackage{array} % for making whole row bold in table
\usepackage{colortbl} % for background colours in table rows
\usepackage{longtable}
\usepackage{amssymb} % for \checkmark 
\usepackage{amsmath} % for \checkmark 
\usepackage{rotating}
\usepackage[nolists,tablesfirst]{endfloat} % for putting figs and tables at end of document
\DeclareDelayedFloatFlavor{sidewaystable}{table}
\usepackage{makecell} % for \makecell in tables
\usepackage{doi}

%\usepackage{xr} % to obtain label references from supp materials file
%\externaldocument[S-]{suppmat}

\usepackage{tabularx}
\usepackage{booktabs}
\usepackage{array} % for table wrapping of columns
\newcolumntype{L}[1]{>{\raggedright\let\newline\\\arraybackslash\hspace{0pt}}m{#1}}
\newcolumntype{C}[1]{>{\centering\let\newline\\\arraybackslash\hspace{0pt}}m{#1}}
\newcolumntype{R}[1]{>{\raggedleft\let\newline\\\arraybackslash\hspace{0pt}}m{#1}}
%\usepackage[detect-all]{siunitx} % for SI units

% custom hyphenation:
\hyphenation{inverse}
\hyphenation{At-lantic}
\hyphenation{elasmo-branchs}


% Macros
\newcommand{\So}{$S_{1}$\xspace}
\newcommand{\St}{$S_{2}$\xspace}
\newcommand{\Pg}{$P_r$\xspace}
\newcommand{\prmu}{$\mu_r$\xspace}
\newcommand{\prsig}{$\sigma_r$\xspace}
\newcommand{\Linf}{$L_{\infty}$}
\newcommand{\DWinf}{$DW_{\infty}$}
\newcommand{\alphat}{$\tilde{\alpha}$}
\newcommand{\lamat}{$l_{\alpha_{mat}}$}
\newcommand{\lamatb}{$l_{\alpha_{mat}}b$}
\newcommand{\rmax}{$r_{max}$\xspace}
\newcommand{\ageratio}{$\alpha_{mat}/\alpha_{max}$}
\newcommand{\yr}{year\textsuperscript{-1}}
\newcommand{\rsq}{$R^2$\xspace}
\newcommand{\mytilde}{\raise.17ex\hbox{$\scriptstyle\mathtt{\sim}$}}
% Select what to do with command \comment:  
%\newcommand{\comment}[1]{}  % comment not shown
\newcommand{\comment}[1]{\par {\bfseries \color{blue} #1 \par}} % comment shown
%% END MACROS SECTION

\begin{document}
\linenumbers


\section*{Trends in marine survival of Atlantic salmon in eastern Canada}

\textbf{Sebasti\'{a}n A. Pardo\textsuperscript{1*}, 
        Geir H. Bolstad\textsuperscript{2}, 
        J. Brian Dempson\textsuperscript{3}, 
        Julien April\textsuperscript{4}, 
        Ross A. Jones\textsuperscript{5}, 
        Martha J. Robertson\textsuperscript{3}, 
        Dustin Raab\textsuperscript{6}, 
Jeffrey A. Hutchings\textsuperscript{1}} 

\noindent\small{\textsuperscript{1} Department of Biology, Dalhousie University, Halifax, NS, Canada\\}
\small{\textsuperscript{2} Norwegian Institute for Nature Research (NINA), Trondheim, Norway\\}
\small{\textsuperscript{3} Fisheries and Oceans Canada, St. John's, NL, Canada\\}
\small{\textsuperscript{4} Minist\`{e}re des For\^{e}ts, de la Faune et des Parcs, Qu\'{e}bec, QC, Canada\\}
\small{\textsuperscript{5} Fisheries and Oceans Canada, Moncton, NB, Canada\\}
\small{\textsuperscript{6} Fisheries and Oceans Canada, Dartmouth, NS, Canada\\}
\small{\textsuperscript{*} Corresponding author: spardo@dal.ca}

\section*{Abstract}

Declines in wild Atlantic salmon (\emph{Salmo salar}) throughout the north
Atlantic are primarily attributed to declining survival at sea. This
hypothesis has proven challenging to test on a river-by-river basis because of
the need to model data on both migrating smolts and returning adults and to
simultaneously estimate multiple parameters, especially for salmon spending
more than one winter at sea (1SW) before spawning. We fit a hierarchical Bayesian
maturity schedule model to data (19-42 years) for seven populations in  
Newfoundland and Labrador (NL), Qu\'{e}bec (QC), New Brunswick (NB), and Nova
Scotia (NS), Canada. We estimate survival in the first (\So) and second year at sea (\St),
and the proportion returning as 1SW adults (\Pg). Trends in  were not consistent
among rivers. Since 1990, \So increased at Western Arm Brook (NL)
and Rivi\`{e}re Saint-Jean (QC), but declined at Conne River (NL) and Rivière
de la Trinité (QC). Since the mid-1990s, \So increased at Campbellton River (NL),
declined at LaHave River (NS), and fluctuated at Nashwaak
River (NB). Estimates of \St were highly uncertain, particularly for
1SW-dominated populations; \Pg was generally stable. These results challenge the
narrative that marine survival has changed in a temporally consistent manner
among spatially disparate populations. Our findings suggest that, at the
population level, changes in abundance are attributable to temporal shifts in
multiple components of individual fitness, including at-sea survival. If
salmon populations do not respond in a consistent, uniform manner to changing
environmental conditions throughout their range, future research initiatives
should explore why.

% The marine phase of anadromous Atlantic salmon (\emph{Salmo salar}) is the
% least known yet one of the most crucial with regards to population
% persistence. Declines in many Atlantic salmon populations in eastern
% Canada have often been attributed to changes in conditions at sea, negatively
% affecting their survival. However, marine survival estimates are difficult to
% obtain given that many individuals spend multiple winters in the ocean before
% returning to freshwater to spawn, necessitating the estimation of multiple
% parameters. To do so, we fit a hierarchical Bayesian maturity schedule model
% to smolt and adult abundance time series for seven populations located in
% Newfoundland and Labrador (NL), Qu\'{e}bec (QC), New Brunswick (NB), and Nova
% Scotia (NS). The datasets ranged between 19 and 42 years in length. We
% estimated three components of marine survival: survival in the first year at
% sea (\So), survival in the second year at sea (\St), and proportion returning
% as one sea-winter adults (\Pg). Controlling for time frame, trends in
% estimates of \So were not consistent among rivers.
% In the four populations for which data extended to 1990, marine survival
% during the first year at sea predominantly increased over time in Western Arm
% Brook (NL) and Rivi\`{e}re Saint-Jean (QC) but largely declined in Conne River
% (NL) and Rivi\`{e}re Trinit\'{e} (QC). In the three other populations, \So
% exhibited an increase in Campbellton River (NL), a decline in LaHave River
% (NS), and fluctuating stability in Nashwaak River (NB) since approximately the
% mid-1990s. 
% Estimates of \St were highly uncertain, particularly for 1SW-dominated rivers
% (Conne, Campbellton and WAB, where the abundance of 2SW returns is very low)
% and thus the posterior distributions matched our choice of prior and trends
% could not be assessed. 
% \Pg was temporally stable within rivers, except for LaHave and Nashwaak Rivers. 
% Our findings challenge the assumption that marine survival has changed in a consistent
% manner among populations across a broad geographic scale, and suggest that
% trends can be river-specific. 
% The well established correlations between climate variables and abundance are
% not being mediated solely by marine survival, and there can potentially be
% some important indirect effects on fecundity.
% Further work exploring potential correlates of marine survival and the
% potential non-lethal effects of climate effects is warranted.


Keywords: salmonid, survival at sea, natural mortality, marine mortality

%Running Head: 

\section*{Introduction} % (4-5 paragraphs)

%1. Declining salmon numbers

Reductions in fishing mortality, albeit necessary, are not always sufficient
to facilitate population recovery. Experience with numerous commercially
exploited marine fisheries since the early 1990s has shown that not all
populations respond as favourably as anticipated to major reductions in
exploitation \citep{Hutchings2017}. Gradual efforts to close commercial
Atlantic salmon (\emph{Salmo salar}) fisheries in eastern Canada culminated in
full moratoria in all regions, beginning in the Maritime provinces (1984) and
following in Newfoundland (1992), Labrador (1998), and Qu\'{e}bec (2000). Since
these closures, many populations have not increased as 
expected \citep{Dempson2004, ICES2019}; some 
have been assessed as considered threatened or endangered by the 
Committee on the Status of Endangered Wildlife in Canada \citep[][]{Cosewic2010}, 
Canada's national science advisory body (to the national government) on
species risk of extinction.
While it is not fully understood what is driving population declines, the potential
drivers of these are many \citep[see ][for a detailed discussion of possible
causes]{Cairns2001}, including but not limited to: fishing mortality \citep{Dempson2004}, 
damming of waterways and changes in the freshwater habitat \citep{Dunfield1985}, acidification
\citep[particularly in the Southern Uplands region of
NS, see][]{Gibson2010}, predation by seals and birds \citep{Cairns2000}, negative
effects of interbreeding or interactions with escaped farmed salmon
\citep{Keyser2018}, and climate-driven changes in survival and productivity \citep{Mills2013}.

% 2.
Over the past three decades, a narrative has emerged that marine survival of
Atlantic salmon has declined throughout the North Atlantic \citep{ICES2019}.
%\citep{Hansen1998,OMaoileidigh2003,Chaput2012a}.
Based on multiple lines of evidence that climate conditions can directly and
indirectly influence the abundance and productivity of Atlantic salmon
populations \citep{Mills2013,Almodovar2019}, it has been presumed that oceanic climate effects are
driving population dynamics primarily through changes in marine survival.
An implicit assumption is that any trend in
survival in ocean habitat that is shared by multiple populations during their
seaward migration period will be experienced similarly. 
Put another way, given that salmon from different rivers 
are hypothesized to share marine habitat during some of their time at sea, it
has been presumed that populations share similar temporal trends in
at-sea mortality \citep{Friedland1993, Friedland1998, Russell2012}.  
A recent study by \citet{Olmos2019} suggested that trends in post-smolt
survival, when estimated at the stock level, are synchronously declining
for all Atlantic salmon in eastern North America.

% This perception is widespread both in the scientific
% literature as well as federal reports. The latest status report on Atlantic
% salmon by COSEWIC states that ``While the mechanism(s) of marine mortality is
% uncertain, what is clear is that the recent period of poor sea survival is
% occurring in parallel with many widespread changes in the North Atlantic
% ecosystem.'' \citep{Cosewic2010}. Consequently, there have several lines of
% inquiry as to what the causes behind these declines might be attributable to
% \citep{Friedland1993, Friedland1998}.


In contrast to the narrative of widespread, demographically similar increases
in at-sea mortality, the conservation status of Canadian salmon populations differs
considerably. Populations in the southern part of their range are more
likely to be assessed as being of conservation concern than those in more
northerly regions \citep{Cosewic2010}. This geographical disparity in status
suggests that if marine survival has been, or is, a key factor responsible for
most population declines, these changes are not uniformly distributed across
all populations. 

% Nonetheless, most of the populations declines assessed by COSEWIC have
% occurred in populations in the southern extent of its distribution:
% populations in the Bay of Fundy, Anticosti Island, and the Atlantic coast of
% Nova Scotia being assessed as Endangered, populations in the south coast of
% Newfoundland assessed as Threatened, And the populations in the New Brunswick
% and Qu\`{e}bec coasts of the Gulf of St. Lawrence assessed as Special Concern
% \citep{Cosewic2010}. On the other hand, northermost populations have shown
% stable, or even increasing population trends, suggesting that, if marine
% survival were to be a factor in many of these declines, these changes are not
% uniformly distributed across all populations.

% While the purported decline of marine survival is mentioned widely in the 
% scientific literature, only relatively few studies have quantified it in detail.
% The basis for this premise is a limited number of time series that exhibit
% temporal declines in a proxy of marine survival (i.e. return rates or
% post-smolt survival).

% 3. Nonetheless, this premise has never been examined in detail
Given the logistical challenges associated with estimating at-sea survival, it
is not surprising that the number of studies that have estimated temporal
trends has been limited. An additional limitation has been the derivation of
proxies (e.g., return rates), rather than direct model-based estimates, of
marine survival.
\citet{Chaput2012a}, for example, examined the return rate of smolts to adult salmon 
as a metric of marine survival, finding that most Canadian populations 
had experienced declining return rates. 
However, examination of trends in return rates alone
can mask changes in differential survival during different years at sea, as well
as changes in the proportion of adults returning after one or two years at sea.
Recently \citet{Olmos2019} suggested that trends in post-smolt
survival, when estimated at the stock unit level, are synchronously declining
for all Atlantic salmon in Eastern North America and Canada, a conclusion ultimately 
grounded on the veracity of highly variable stock-recruitment relationships.

%4. 
In the present study, we compile data on the number of migrating smolts and number of returning adults 
for seven wild Canadian populations of Atlantic salmon to model trends in marine survival.
While some studies have previously used maturity-schedule models to estimate marine
survival for a limited number of salmon populations \citep{Chaput2003b}, none
have incorporated data extending over multiple decades, nor have they examined
trends among more than two or three populations. 
Here, we develop a hierarchical Bayesian model that uses Murphy's maturity
schedule method, in conjunction with informative priors, to estimate yearly
marine survival in salmon. In addition to accounting for observation error in
smolt and return estimates, we estimate the proportion of salmon returning
after one winter hierarchically.

\section*{Methods}

\subsection*{Data}

We obtained time series data of outmigrating smolt and returning adult
abundances for seven Atlantic salmon populations in eastern Canada, encompassing a
wide range of the species' distribution (Fig.~\ref{fig:map}). 
Populations included the LaHave River in the Southern Uplands region of Nova
Scotia (NS), Nashwaak River, New Brunswick (NB), Rivi\`{e}re de la Trinit\'{e} (Trinit\'{e}) and
Saint-Jean rivers in Qu\'{e}bec (QC), and  Western Arm Brook (also referred to as  WAB), Campbellton, and
Conne River, Newfoundland (NL). 
Data were collected in NS, NB, and Newfoundland
and Labrador (NL) by Fisheries and Oceans Canada (DFO) and in QC
by the Minist\`{e}re des For\^{e}ts, de la Faune et des Parcs, Qu\'{e}bec.

\begin{figure}[htbp] \centering
    \includegraphics[width=0.85\linewidth]{figures/rivers-map2.png}
    \caption{Locations of the seven rivers in eastern Canada with time series abundance data of outmigrating smolts and 
    returning adults.} \label{fig:map} 
\end{figure}

\subsubsection*{Smolt and adult return abundance data}

Smolt and adult return abundance estimates originate from a variety of
sources. Smolt estimates from the Trinit\'{e}, Saint-Jean, and Conne populations were obtained using a
mark-recapture approach, while estimates from the ..., WAB, and Campbellton populations
were obtained by direct counts using fish counting fences.
For further details on the data collection methodologies refer to 
\citet{Dempson1991, Venoitt2018} for NL populations, 
\citet{April2018}  for QC populations,
\citet{Jones2014} for NB populations,
and \citet{Gibson2009} for NS populations. 

Yearly return data are often recorded in terms of two size groups: small ($< 63$ cm
FL) and large ($\geq 63$ cm FL) salmon, as these closely represent different
life-history strategies (i.e. 1SW and 2SW), but can be confounded with repeat
spawners of different sizes. To correct for this in returns 
reported as small and large salmon, we estimated the abundance
of 1SW and 2SW returns using yearly scale age data of a subsample of returns:

\begin{equation}
    p_{r,t,a} = \frac{\sum_{s}{(\frac{n_{r,t,s,a}}{n_{r,t,s}} * N_{r,y,s})}}{\sum_{s}{N_{r,t,s}}}
\end{equation}

where $p_{r,t,a}$ is the proportion of annual returns in river $r$, year $t$,
and of spawning history $a$ (either 1SW or 2SW returns); $n_{r,t,s,a}$ is the
number of samples in river $r$, year $t$, of spawning history $a$, and of size
group $s$; $n_{r,t,s}$ is the total number of samples in river $r$, year $t$,
and of size group $s$; and $N_{r,t,s}$ is the returns of salmon
in river $r$, year $t$, and of size group $s$.

\subsection*{Bayesian model}

We developed a hierarchical Bayesian model that uses Murphy's maturity
schedule method, in conjunction with informative priors, to estimate yearly
marine survival in seven populations of Atlantic salmon. We account for
observation error in smolt and return estimates, as well as estimating the
proportion returning after one winter (i.e. \Pg) hierarchically.
There is an identifiability problem in the maturity schedule equations where
the parameter estimates cannot be optimally solved \citep{Chaput2003a}.
However, this issue can be mathematically overcome, at least partially, by
using informative priors for all three marine survival parameters in a
Bayesian framework.
This method requires abundance estimates of smolts as well as abundance estimates
of returning one-sea-winter (1SW) and two-sea-winter adults. With these data,
it estimates three parameters: survival in the first year at sea (\So), survival
in the second year at sea (\St), and the proportion of fish returning after one
year at sea (\Pg). 

Our model does not include repeat spawners and assumes that no fish spend
three or more winters at sea before returning to spawn for the first time.
The model also assumes that mortality in the second winter at sea (\St)
is additional to mortality in the first winter at sea in the previous year, 
and therefore does not account for differences in environmental conditions experienced
between 1SW and 2SW fish of the same smolt cohort during their overlapping first year at sea.
In other words, our model assumes that the decision of returning occurs just before
actually being counted as returns and that \St is any additional mortality in
the subsequent year. 

%\comment{Other assumptions...}

Observed smolt estimates were modelled hierarchically and included
observation error:

\begin{equation}
log(smolts_{obs,t,r}) = log(smolts_{true,t,r}) + \epsilon_{t,r}
\end{equation}

where $smolts_{true,t,r}$ are the true smolt abundances for year $t$ and river
$r$, and $\epsilon_{t,r}$ is the error term, which is empirically derived from
by calculating the yearly coefficient of variation in the empirically derived
smolt estimates (see Table S1 in the Supplementary material). 
Where available, we used population-specific measurement error estimates for smolt abundances; if not 
available, we set measurement error at 5\%. 
The log-transformed true smolt abundances are
normally distributed around a population-level mean and standard deviation:

\begin{equation}
log(smolts_{true,t,r}) \sim Normal(\mu_{smolts,r}, \sigma_{smolts,r})
\end{equation}

where $\mu_{smolts,r}$ and $\sigma_{smolts,r}$ are parameters estimated by the
model for each river.

Once we have yearly estimates of smolt, 1SW, and 2SW abundances, we estimate
marine survival parameters using Murphy's maturity schedule method
\citep{Murphy1952, Ricker1975}:

\begin{align}
    R_{1,t} &= smolts_{true,t-1} * S_{1,t} * Pr_t \label{eq:1}, \\
    R_{2,t+1} &= smolts_{true,t-1} * S_{1,t} * (1 - Pr_t) * S_{2,t+1} \label{eq:2}
\end{align}

where $R_{1,t}$ and $R_{2,t+1}$ are the estimated abundances of 1SW and 2SW
salmon returning in years $t$ and $t+1$, respectively, $smolts_{true,t-1}$ is the
estimated number of outmigrating smolts in year $t-1$, $S_{1,t}$ is the proportion of
salmon surviving in their first year ($t$) at sea, $Pr_t$ is the proportion of
salmon that return to spawn at year $t$, $S_{2,t+1}$ is the survival in their
second year at sea of the same cohort of salmon who did not return to spawn at
year $t$.

However, to allow for normally and log-normally-distributed parameters we
log-transform equations~\ref{eq:1} and~\ref{eq:2} to obtain

\begin{align}
    log(R_{1,r,t}) &= log(smolts_{r,t-1}) + log(Pr_{r,t}) - Z_{1,r,t} \label{eq:3}, \\
    log(R_{2,r,t+1}) &= log(smolts_{r,t-1}) - Z_{1,r,t} + log(1 - Pr_{r,t})  - Z_{2,r,t+1} \label{eq:4} 
\end{align}

where $Z_{1,r,t}$ and $Z_{2,r,t+1}$ are the instantaneous mortality rates.
Process error was included as the standard deviation of the log-transformed
return estimates from equation~\ref{eq:4}:

\begin{align}
log(R_{obs,1,t,r}) &\sim Normal(log(R_{1,t,r}), \epsilon_{1,r}), \\
log(R_{obs,2,t,r}) &\sim Normal(log(R_{2,t,r}), \epsilon_{2,r}) \label{eq:5} 
\end{align}

where $R_{obs,1,t,r}$ and $R_{obs,2,t,r}$ are the observed return estimates
for year $t$ and river $r$ of 1SW and 2SW fish, respectively, $\epsilon_{1,r}$
and $\epsilon_{2,r}$ are the process error terms. 
These error terms are normally distributed with a standard deviation
of 0.01, which are almost equivalent to using a coefficient of variation in
return estimates 1\%, given that return estimates in equations~\ref{eq:3}
and~\ref{eq:4} are log-transformed:

\begin{align}
\epsilon_{1,r} &\sim Normal(0, 0.01) \\
\epsilon_{2,r} &\sim Normal(0, 0.01).
\end{align}

Furthermore, we use instantaneous mortality rates in the model instead of survival probabilities
as the model is more efficient in its parameter search in log-space, and instantaneous rates
are easy to interpret. Instantaneous rates are easily converted to survival probabilities by 

\begin{align}
 S_{1} &= e^{-Z_1}, \\
 S_{2} &= e^{-Z_2}. 
\end{align}

We estimate population-level mean \Pg values around which the yearly \Pg
values are normally distributed. We specify different informative hyperpriors
for \prmu and \prsig based on whether the population is 1SW-dominated or not:

\begin{align}
    logit(P_{r,t}) &\sim Normal(\mu_r, \sigma_r) \\
    \mu_r &\sim 
    \begin{cases}
       Normal(2.3, 0.4),  &\text{for 1SW-dominated populations} \\
       Normal(0, 2.8), &\text{for non-1SW-dominated populations} \\
   \end{cases} \\
    \sigma_r &\sim halfNormal(0, 1).
\end{align}

The priors for $Z_1$ and $Z_2$ are specified as log-normal distributions:

\begin{align}
Z_1 &\sim logNormal(1, 0.22),   \\ 
Z_2 &\sim logNormal(0.2, 0.3).
\end{align}

The model was written in Stan \citep{Carpenter2017} and run in R version 3.6.1
\citep{RCoreTeam2019} using the \texttt{rstan} package version 2.19.2
\citep{StanDevelopmentTeam2019}.
The model was run with three chains and 3,000 iterations, with the first 1,500
discarded as a burn-in. The models were considered to have converged when the
$\hat R$ of all parameters were lower than 1.03 and the effective sample size 
were higher than 500.

\subsection*{Correlations among trends in survival}

%\comment{How do were quantitatively compare trends among rivers? DFA is an
%option, but so is a t-test of averages before and after a certain year}

We looked at the correlation of trends in \So by calculating the Pearson's
correlation coefficient for the Z-scores of these trends. Given that the time
series do not cover the same years, and that some rivers have missing years in
the middle of the time series, pair-wise Pearson's tests were done using only
the years where there is data for both rivers.

\section*{Results}

%\subsubsection*{Model convergence}

\subsubsection*{Trends in marine survival parameters}

Trends in estimates of \So were highly variable within and among rivers
(Fig.~\ref{fig:s1-dual}). The highest median posterior estimates of \So
were for the Nashwaak River in 2006 and 2008, with values of 0.18 and 0.21,
respectively. The lowest median \So estimate was in the Trinit\'{e} in 2001,
with an estimate of \So of 0.007, while the Conne, LaHave, and Trinit\'{e} had
years where estimates of \So varied between 0.01 and 0.02 (Fig.~\ref{fig:s1-faceted}).

Trends among populations also varied: Campbellton,
Saint-Jean, and Western Arm Brook populations showed increases in \So
over time, Trinit\'{e}, Conne, and La Have populations showed decreases,
while at Nashwaak there was an increase in median \So during the early
2000s but a decrease in the 2010s. Yearly estimates of \So had very little
variability for one sea-winter dominated populations (Conne, Campbellton, and
WAB), but were more variable (i.e. wider credible intervals) in the other
populations.

\begin{figure}[htbp] \centering
    \includegraphics[width=0.95\linewidth]{figures/s1-trends-dual.png}
    \caption{Trends in survival in the first year at sea (\So). a) Posterior
        estimates of \So for the seven rivers examined, error bars indicate
        the 90\% credible intervals, and dashed lines denote years of commercial
        fishing moratoria for each province. b) Z-scores of median posterior
        estimates.} \label{fig:s1-dual} \end{figure}

\begin{figure}[htbp] \centering
    \includegraphics[width=0.95\linewidth]{figures/s1-trends-faceted.png}
    \caption{Posterior estimates of \So for the seven rivers examined, error
        bars indicate the 90\% credible intervals.} \label{fig:s1-faceted}
\end{figure}

Estimates of \St were highly uncertain in all rivers, and trends 
were not apparent in most rivers given the large range of the credible
intervals in the yearly estimates (Fig.~\ref{fig:s2-faceted}). The estimates
of \St for the Saint-Jean and Trinit\'{e} were considerably less uncertain
than for the other populations, but showed no apparent trends through time.

\begin{figure}[htbp] \centering
    \includegraphics[width=0.95\linewidth]{figures/s2-trends-faceted.png}
    \caption{Posterior estimates of \St for the seven rivers examined, error
        bars indicate the 90\% credible intervals.} \label{fig:s2-faceted}
\end{figure}

Estimates of \Pg were mostly stable across time, except for the LaHave and
Nashwaak Rivers; The estimates of \Pg were slightly lower in the last four
years than in the previous ones, while in the Nashwaak the posterior estimates
of \Pg in 2012 were much lower than in all other years
(Fig.~\ref{fig:s2-faceted}). Uncertainty in yearly estimates was highest in
the LaHave, Nashwaak, and Trinit\'{e}, and lowest in the 1SW-dominated
populations.

\begin{figure}[htbp] \centering
    \includegraphics[width=0.95\linewidth]{figures/pr-trends-faceted.png}
    \caption{Posterior estimates of \Pg for the seven rivers examined, error
        bars indicate the 90\% credible intervals.} \label{fig:pr-faceted}
\end{figure}

Population-level estimates of \prmu and \prsig varied considerably among rivers. For all
three 1SW-dominated rivers (Campbellton, Conne, and WAB), estimates of \prmu
were very close to 1.0 and had little variability in \prsig (Fig~\ref{fig:prmu-post}).
Estimates of \prmu were the lowest for the two QC rivers, particularly the Saint-Jean (median \prmu = 0.11).
Estimates of \prmu for the Nashwaak and the LaHave Rivers were close to 0.5, with these two rivers having 
the highest estimated values of \prsig, particularly the Nashwaak (Fig~\ref{fig:prmu-post}).

\begin{figure}[htbp] \centering
    \includegraphics[width=1.0\linewidth]{figures/pr-mu-posteriors.png}
    \caption{Posterior estimates of the population-level parameters $Pr_{\mu}$
       $logit(Pr_{\mu})$, and $logit(Pr_{\sigma})$. Dots denote median estimates, while the thick and thing error bars indicate
       the 50\% and 90\% credible intervals, respectively.} 
   \label{fig:prmu-post} 
\end{figure}

\subsubsection*{Correlations}

Most correlations between z-scored trends were not statistically significant, with only 
three out of the 21 pair-wise comparisons having a p-value below 0.01 (Fig.~\ref{fig:s1-corr}a).
When looking at the direction of the correlation, regardless whether they were
significant or not, these spanned both positive and negative coefficients
(Fig.~\ref{fig:s1-corr}b).

\begin{figure}[htbp] \centering
    \includegraphics[width=0.95\linewidth]{figures/corr-s1.png} \caption{
        Correlation among Z-scores of median estimated trends in \So among
        rivers. a) Pearson's correlation coefficients are shown in each square,
        while colouring denotes significance of the correlation ($p \leq 0.01$), b)
        colours denote direction and magnitude of correlation, while asterisks denote significance.}
\label{fig:s1-corr} 
\end{figure}

\section*{Discussion} 

% 1. Main findings
% 2. How they compare to other publications
% 3. Potential reasons
% 4. Caveats
% 5. Future directions

% 1. Main findings
Our results challenge the narrative that marine survival, specifically survival in the first year at sea, is declining
uniformly throughout the range of Atlantic salmon in the northwest Atlantic.
Temporal trends are not consistent among populations. 
Over the time periods for which data were available, some rivers show positive trends in survival in the
first winter at sea (\So) while other exhibit highly variable yet stable trends, and some show
declines. We could not assess trends in the second winter at sea (\St) or
proportion returning as grilse (\Pg), as these parameter estimates were highly
uncertain and were strongly influenced by the priors.
Perhaps there is a need to rethink our understanding of Atlantic salmon
population dynamics in light of the possibilities that (1) any real and persistent decline in marine survival was
experienced by some but not necessarily all populations, (2) reductions in survival might
have occurred over a relatively brief period of time and have not persisted, and (3) marine survival has
been relatively stable, or increasing in some populations for one or more decades.

% 2. How they compare to other publications
Our results are contrary to those of \citet{Olmos2019}, who detected positive
correlations in post-smolt survival among spatially broad stock units. These differences
could be due to a number of reasons: different model specifications and
structure, different methods for estimating covariance, difference in the
spatial scales of data sources (i.e. river vs province scales), and perhaps most importantly the use stock-recruitment relationships
rather than empirical smolt count data to estimate marine survival.
As trends in marine survival during the first winter at sea are highly independent
among rivers on relatively small spatial scale, trends from broader
geographical areas (i.e. province, state, or country-wide estimates) may not
be representative.
Interestingly, our estimates of \So and \St are very similar to those produced
by \citet{Chaput2003b}, and our trends are almost identical for the
overlapping time period that marine survival was estimated for in their study
(1984-1998). 
While \citet{Chaput2003b} separated abundance data for males and females
and assumed their survival rates were the same (to be able to reach an
analytical solution), our study reached almost the same results (albeit with
slightly higher uncertainty), using a Bayesian approach with informative
priors. These overlapping trends obtained with two different methods 
suggest that our method is effective at estimating marine survival.

Trends in marine survival
among populations were compared by \citet{Chaput2012a} using adult return rates.
He found that for 4 of 6 populations examined, return rates in the 1990s 
were lower than those during the 1970s.
\citet{Friedland1993} compared return rates for a number of rivers in eastern
North America between 1973 and 1988, and suggested there are similar trends among these. 
However, the similarity in these trends was driven primarily by two years, 1977 and 1978, which
show concurrent low and high relative return rates across rivers,
respectively. Other years are much more variable relative to each other.
\citeauthor{Friedland1993}'s \citeyear{Friedland1993} time series ends in  
1988; thus there are only a few years for which to assess overlap with the
time series in our study.
In any event, we caution that the pooling of adult return rates \citep{Chaput2012a, Friedland1993} 
can mask inter-annual variability in marine survival,
and hence might not produce an accurate depiction of marine survival trends.
\citet{Dempson2003} described a general declining trend in marine survival for
Newfoundland rivers (except WAB); we drew the same conclusion for 
Conne River but not Campbellton River or WAB. It is not possible to draw broader conclusions
with data from only three Newfoundland rivers, but it seems that among index rivers,
those in Newfoundland are among those with the highest marine survival rates.

% 3. Potential reasons
There are a variety of potential explanations for the lack of synchronous
trends in estimates of \So. 
Marine survival in the first winter at sea could be highly variable between
populations because of the predominance of spatially local environmental drivers of survival (e.g., temperature, predation) 
relative to broader-scale, even ocean-wide, drivers.
The synchrony reported for marine survival trends at broader spatial scales \citep{Olmos2019}
might be attributable to the use of stock-recruitment relationships to estimate survival,
relationships that may have been confounded by changes in recruitment dynamics.
There is some evidence of a correlation between return rate and growth (as
indicated by inter-circuli spacing on scales), where years of poor growth
tended to also be years of poor survival \citep{Friedland1993}, supporting the
idea that environmental variability can affect marine survival.
Furthermore, among European salmon, there is evidence of a positive correlation
between spring temperature in the Norwegian and North Seas and population abundance, suggesting warmer
conditions favour post-smolts \citep{Friedland1998}, based on mapping the
extent of area of suitable temperature (7-13 \textdegree C).

Nonetheless, the causal mechanisms for why warming should affect post-smolt
survival almost certainly differs depending on the difference between
temperature experienced by the post-smolts and their respective
population-specific thermal optima. 
This difference could explain why populations in eastern North America are
declining in the southern part of their range but potentially increasing
further north, and also why some studies find positive correlations between
temperature and abundance \citep{Friedland1998, Friedland1998b, Jonsson2004}
while others find negative ones \citep{Friedland1993, Todd2008}.
Putative associations between temperature and direct estimates of marine
survival warrants further study at the population level.

While there is little evidence that marine survival is density-dependent in
Atlantic salmon \citep{Jonsson1998,Gibson2006}, there could potentially be
some density-dependent processes during parts of the post-smolt migration
period, particularly for populations that are likely to be subjected to
declining per capita population growth rates ($r$) generated by Allee effects.
Exploring relationships between survival and population size could potentially
shed light about the processes that have caused many of the population
declines that have been documented.

Oceanic conditions have been correlated with abundance trends and growth
\citep{Todd2008}, however, the mechanism by which such bottom-up effects, \
mediated by changes in food availability,
affect population dynamics beyond marine survival needs to
be thoroughly reassessed. If marine survival on its own cannot fully explain
trends in abundance, then there are potential carry-on effects of oceanic
conditions that manifest with regards to fresh production. 
For example, adults
that return to spawn after spending suboptimal conditions at sea might be less
likely to make it to their spawning grounds, successfully secure a mate,
produce fewer eggs, or produce eggs with lower per capita fitness than those
produced by adults which grew in optimal oceanic conditions.
As larger females tend to be more productive, in terms of fecundity and total
reproductive energy, than the same weight's worth of smaller females
\citep{Barneche2018}, a small decrease in body condition resulting from bottom-up
impacts on food availability could potentially have disproportionate effects on fecundity
and fitness of the offspring.

Egg-to-smolt survival in Atlantic salmon is highly variable \citep{Klemetsen2003,Chaput2015}
and changes in the oceanic conditions that spawners experience could be
contributing to this variability.
Obviously, there would be a time lag (perhaps as much as a generation) in how such effects
might be manifest at the adult stage.
However, given that most correlations are between relatively monotonic declines
in abundance coupled also monotonic increases in climatic indices
over decadal time scales \citep[e.g.,][]{Friedland1998, Todd2008,
    Beaugrand2012}, it would be expected that this correlations would be
maintained even if salmon abundances were lagged by a generation length.


% 4. Caveats
As with all novel modelling approaches, there are caveats to acknowledge.
The seven populations explored in the present study might not be representative
of regional trends in marine survival. However, there are no other
long-term time series of smolts and adult returns to draw inferences from.
While there are analytical issues associated with the estimation of \So, \St, and \Pg,
the assumption that \St is additive to \So could produce unrealistic results.
We know there is a period of a few months where 1SW
returns are subject to a different environment than those salmon that will
return as 2SW the next year. 
While this is not ideal,
overcoming this assumption would require an additional parameter to be
estimated, or an additional assumption as to what proportion of \So is not
additive to \St (as the returning 1SW adults do not experience the same
environment when they return to their natal streams as those fish who stayed
at sea for an additional winter before returning to spawn).

Secondly, the assumed hierarchical structure might not be the most appropriate
for modelling smolt abundances. This approach results in shrinkage to the
mean, which means that the variability of yearly smolt estimates is less than
it would be if not modelled hierarchically. 
That said, this assumption is likely to result in more conservative trends in marine survival, as the
variation is smolt estimates is reduced.
It is important to note that the hierarchical structure in the estimation of
\Pg seems like a reasonable assumption given that the probability of returning
as grilse has a genetic component associated to it \citep{Aykanat2019} and is
not expected to vary much, within a population, among years.

% 5. Future directions
Perhaps a reframing of the issue of marine survival is key to furthering our
understanding of Atlantic salmon population dynamics. Marine survival may not
have declined consistently, and over the same time periods, across all
populations. 
But the fact that it has remained at roughly similar levels as it was
previously \emph{despite} reduced commercial fishing mortalities, suggests
that there may well be an interaction between small population size (small
relative to unfished population size or carrying capacity), recovery
potential, and environmental stochasticity that has not been fully explored in
Atlantic salmon. 
All else being equal, relatively small populations are more vulnerable to
demographic, environmental, and genetic stochasticity than large populations
\citep{Lande1993, Hutchings2015}. Interactions between population size and the
demographic consequences of environmental stochasticity appear to have
affected recovery in many marine fishes that have exhibited impaired recovery
since mitigation of the threat posed by fishing mortality
\citep{Hutchings2017, Hutchings2020}. The possibility that similar
interactions may be impairing the recovery of wild Atlantic salmon merits
study.

\section*{Acknowledgements}

% We would like to thank Geir Bolstad, G\'{e}rald Chaput, Brian Dempson, and
% Martha Robertson for their useful discussions on estimating marine survival
% in Atlantic salmon, and Amanda Kissel for her helpful comments on the
% manuscript. Brian Dempson and Geoff Venoit for providing the data Conne
% River data.
We would like to thank G\'{e}rald Chaput for his useful discussions on
estimating marine survival in Atlantic salmon, Carmen David for her comments
on the manuscript, and Sean Anderson for his help with implementing the
non-centered parameterization of the model. This research was supported by the
Atlantic Salmon Conservation Foundation and the Atlantic Salmon Research Joint
Venture.

\section*{Conflicts of Interest}

The authors declare no conflicts of interest.
 
\bibliography{subset}

%\documentclass[12pt]{article}
\usepackage[top=0.85in,left=1.0in,right=1.0in,footskip=0.75in]{geometry}
%\usepackage[parfill]{parskip}
\usepackage{setspace}
\usepackage{lineno}
\usepackage[hidelinks]{hyperref}
%\onehalfspacing
\doublespacing

\usepackage[round,sectionbib]{natbib}
\setcitestyle{authoryear}
\bibpunct{(}{)}{;}{a}{}{;}
\bibliographystyle{fishfishnourl}

%\usepackage{textcomp}
%\usepackage{libertine}
%\usepackage{inconsolata} % sans serif typewriter

\usepackage{mathtools} % for dcases
\usepackage{xcolor} % for textcolor
\usepackage{makecell} % for \makecell and within cell line breaks (\thead)

\usepackage[T1]{fontenc}

%\makeatletter\let\expandableinput\@@input\makeatother % expandable input for \input inside tables

% Linux Libertine:
\usepackage{textcomp}
\usepackage[sb]{libertine}
\usepackage[varqu,varl]{inconsolata}% sans serif typewriter
\usepackage[libertine,bigdelims,vvarbb]{newtxmath} % bb from STIX
\usepackage[vvarbb]{newtxmath} % bb from STIX, removed bigdelims for ScholarOne rendering
\usepackage[cal=boondoxo]{mathalfa} % mathcal
%\useosf % osf for text, not math
%\usepackage[supstfm=libertinesups,%
%  supscaled=1.2,%
%  raised=-.13em]{superiors}

\usepackage{xspace}
\usepackage{xfrac} % for diagonal inline fractions in text
%\usepackage{array} % for making whole row bold in table
\usepackage{colortbl} % for background colours in table rows
\usepackage{longtable}
\usepackage{amssymb} % for \checkmark 
\usepackage{amsmath} % for \checkmark 
\usepackage{rotating}
\usepackage[nolists,tablesfirst]{endfloat} % for putting figs and tables at end of document
\DeclareDelayedFloatFlavor{sidewaystable}{table}
\usepackage{makecell} % for \makecell in tables
\usepackage{doi}

%\usepackage{xr} % to obtain label references from supp materials file
%\externaldocument[S-]{suppmat}

\usepackage{tabularx}
\usepackage{booktabs}
\usepackage{array} % for table wrapping of columns
\newcolumntype{L}[1]{>{\raggedright\let\newline\\\arraybackslash\hspace{0pt}}m{#1}}
\newcolumntype{C}[1]{>{\centering\let\newline\\\arraybackslash\hspace{0pt}}m{#1}}
\newcolumntype{R}[1]{>{\raggedleft\let\newline\\\arraybackslash\hspace{0pt}}m{#1}}
%\usepackage[detect-all]{siunitx} % for SI units

% custom hyphenation:
\hyphenation{inverse}
\hyphenation{At-lantic}
\hyphenation{elasmo-branchs}


% Macros
\newcommand{\So}{$S_{1}$\xspace}
\newcommand{\St}{$S_{2}$\xspace}
\newcommand{\Pg}{$P_r$\xspace}
\newcommand{\prmu}{$\mu_r$\xspace}
\newcommand{\prsig}{$\sigma_r$\xspace}
\newcommand{\Linf}{$L_{\infty}$}
\newcommand{\DWinf}{$DW_{\infty}$}
\newcommand{\alphat}{$\tilde{\alpha}$}
\newcommand{\lamat}{$l_{\alpha_{mat}}$}
\newcommand{\lamatb}{$l_{\alpha_{mat}}b$}
\newcommand{\rmax}{$r_{max}$\xspace}
\newcommand{\ageratio}{$\alpha_{mat}/\alpha_{max}$}
\newcommand{\yr}{year\textsuperscript{-1}}
\newcommand{\rsq}{$R^2$\xspace}
\newcommand{\mytilde}{\raise.17ex\hbox{$\scriptstyle\mathtt{\sim}$}}
% Select what to do with command \comment:  
%\newcommand{\comment}[1]{}  % comment not shown
\newcommand{\comment}[1]{\par {\bfseries \color{blue} #1 \par}} % comment shown
%% END MACROS SECTION

\begin{document}
\linenumbers


\section*{Trends in marine survival of Atlantic salmon in eastern Canada}

\textbf{Sebasti\'{a}n A. Pardo\textsuperscript{1*}, 
        Geir H. Bolstad\textsuperscript{2}, 
        J. Brian Dempson\textsuperscript{3}, 
        Julien April\textsuperscript{4}, 
        Ross A. Jones\textsuperscript{5}, 
        Martha J. Robertson\textsuperscript{3}, 
        Dustin Raab\textsuperscript{6}, 
Jeffrey A. Hutchings\textsuperscript{1}} 

\noindent\small{\textsuperscript{1} Department of Biology, Dalhousie University, Halifax, NS, Canada\\}
\small{\textsuperscript{2} Norwegian Institute for Nature Research (NINA), Trondheim, Norway\\}
\small{\textsuperscript{3} Fisheries and Oceans Canada, St. John's, NL, Canada\\}
\small{\textsuperscript{4} Minist\`{e}re des For\^{e}ts, de la Faune et des Parcs, Qu\'{e}bec, QC, Canada\\}
\small{\textsuperscript{5} Fisheries and Oceans Canada, Moncton, NB, Canada\\}
\small{\textsuperscript{6} Fisheries and Oceans Canada, Dartmouth, NS, Canada\\}
\small{\textsuperscript{*} Corresponding author: spardo@dal.ca}

\section*{Abstract}

Declines in wild Atlantic salmon (\emph{Salmo salar}) throughout the north
Atlantic are primarily attributed to declining survival at sea. This
hypothesis has proven challenging to test on a river-by-river basis because of
the need to model data on both migrating smolts and returning adults and to
simultaneously estimate multiple parameters, especially for salmon spending
more than one winter at sea (1SW) before spawning. We fit a hierarchical Bayesian
maturity schedule model to data (19-42 years) for seven populations in  
Newfoundland and Labrador (NL), Qu\'{e}bec (QC), New Brunswick (NB), and Nova
Scotia (NS), Canada. We estimate survival in the first (\So) and second year at sea (\St),
and the proportion returning as 1SW adults (\Pg). Trends in  were not consistent
among rivers. Since 1990, \So increased at Western Arm Brook (NL)
and Rivi\`{e}re Saint-Jean (QC), but declined at Conne River (NL) and Rivière
de la Trinité (QC). Since the mid-1990s, \So increased at Campbellton River (NL),
declined at LaHave River (NS), and fluctuated at Nashwaak
River (NB). Estimates of \St were highly uncertain, particularly for
1SW-dominated populations; \Pg was generally stable. These results challenge the
narrative that marine survival has changed in a temporally consistent manner
among spatially disparate populations. Our findings suggest that, at the
population level, changes in abundance are attributable to temporal shifts in
multiple components of individual fitness, including at-sea survival. If
salmon populations do not respond in a consistent, uniform manner to changing
environmental conditions throughout their range, future research initiatives
should explore why.

% The marine phase of anadromous Atlantic salmon (\emph{Salmo salar}) is the
% least known yet one of the most crucial with regards to population
% persistence. Declines in many Atlantic salmon populations in eastern
% Canada have often been attributed to changes in conditions at sea, negatively
% affecting their survival. However, marine survival estimates are difficult to
% obtain given that many individuals spend multiple winters in the ocean before
% returning to freshwater to spawn, necessitating the estimation of multiple
% parameters. To do so, we fit a hierarchical Bayesian maturity schedule model
% to smolt and adult abundance time series for seven populations located in
% Newfoundland and Labrador (NL), Qu\'{e}bec (QC), New Brunswick (NB), and Nova
% Scotia (NS). The datasets ranged between 19 and 42 years in length. We
% estimated three components of marine survival: survival in the first year at
% sea (\So), survival in the second year at sea (\St), and proportion returning
% as one sea-winter adults (\Pg). Controlling for time frame, trends in
% estimates of \So were not consistent among rivers.
% In the four populations for which data extended to 1990, marine survival
% during the first year at sea predominantly increased over time in Western Arm
% Brook (NL) and Rivi\`{e}re Saint-Jean (QC) but largely declined in Conne River
% (NL) and Rivi\`{e}re Trinit\'{e} (QC). In the three other populations, \So
% exhibited an increase in Campbellton River (NL), a decline in LaHave River
% (NS), and fluctuating stability in Nashwaak River (NB) since approximately the
% mid-1990s. 
% Estimates of \St were highly uncertain, particularly for 1SW-dominated rivers
% (Conne, Campbellton and WAB, where the abundance of 2SW returns is very low)
% and thus the posterior distributions matched our choice of prior and trends
% could not be assessed. 
% \Pg was temporally stable within rivers, except for LaHave and Nashwaak Rivers. 
% Our findings challenge the assumption that marine survival has changed in a consistent
% manner among populations across a broad geographic scale, and suggest that
% trends can be river-specific. 
% The well established correlations between climate variables and abundance are
% not being mediated solely by marine survival, and there can potentially be
% some important indirect effects on fecundity.
% Further work exploring potential correlates of marine survival and the
% potential non-lethal effects of climate effects is warranted.


Keywords: salmonid, survival at sea, natural mortality, marine mortality

%Running Head: 

\section*{Introduction} % (4-5 paragraphs)

%1. Declining salmon numbers

Reductions in fishing mortality, albeit necessary, are not always sufficient
to facilitate population recovery. Experience with numerous commercially
exploited marine fisheries since the early 1990s has shown that not all
populations respond as favourably as anticipated to major reductions in
exploitation \citep{Hutchings2017}. Gradual efforts to close commercial
Atlantic salmon (\emph{Salmo salar}) fisheries in eastern Canada culminated in
full moratoria in all regions, beginning in the Maritime provinces (1984) and
following in Newfoundland (1992), Labrador (1998), and Qu\'{e}bec (2000). Since
these closures, many populations have not increased as 
expected \citep{Dempson2004, ICES2019}; some 
have been assessed as considered threatened or endangered by the 
Committee on the Status of Endangered Wildlife in Canada \citep[][]{Cosewic2010}, 
Canada's national science advisory body (to the national government) on
species risk of extinction.
While it is not fully understood what is driving population declines, the potential
drivers of these are many \citep[see ][for a detailed discussion of possible
causes]{Cairns2001}, including but not limited to: fishing mortality \citep{Dempson2004}, 
damming of waterways and changes in the freshwater habitat \citep{Dunfield1985}, acidification
\citep[particularly in the Southern Uplands region of
NS, see][]{Gibson2010}, predation by seals and birds \citep{Cairns2000}, negative
effects of interbreeding or interactions with escaped farmed salmon
\citep{Keyser2018}, and climate-driven changes in survival and productivity \citep{Mills2013}.

% 2.
Over the past three decades, a narrative has emerged that marine survival of
Atlantic salmon has declined throughout the North Atlantic \citep{ICES2019}.
%\citep{Hansen1998,OMaoileidigh2003,Chaput2012a}.
Based on multiple lines of evidence that climate conditions can directly and
indirectly influence the abundance and productivity of Atlantic salmon
populations \citep{Mills2013,Almodovar2019}, it has been presumed that oceanic climate effects are
driving population dynamics primarily through changes in marine survival.
An implicit assumption is that any trend in
survival in ocean habitat that is shared by multiple populations during their
seaward migration period will be experienced similarly. 
Put another way, given that salmon from different rivers 
are hypothesized to share marine habitat during some of their time at sea, it
has been presumed that populations share similar temporal trends in
at-sea mortality \citep{Friedland1993, Friedland1998, Russell2012}.  
A recent study by \citet{Olmos2019} suggested that trends in post-smolt
survival, when estimated at the stock level, are synchronously declining
for all Atlantic salmon in eastern North America.

% This perception is widespread both in the scientific
% literature as well as federal reports. The latest status report on Atlantic
% salmon by COSEWIC states that ``While the mechanism(s) of marine mortality is
% uncertain, what is clear is that the recent period of poor sea survival is
% occurring in parallel with many widespread changes in the North Atlantic
% ecosystem.'' \citep{Cosewic2010}. Consequently, there have several lines of
% inquiry as to what the causes behind these declines might be attributable to
% \citep{Friedland1993, Friedland1998}.


In contrast to the narrative of widespread, demographically similar increases
in at-sea mortality, the conservation status of Canadian salmon populations differs
considerably. Populations in the southern part of their range are more
likely to be assessed as being of conservation concern than those in more
northerly regions \citep{Cosewic2010}. This geographical disparity in status
suggests that if marine survival has been, or is, a key factor responsible for
most population declines, these changes are not uniformly distributed across
all populations. 

% Nonetheless, most of the populations declines assessed by COSEWIC have
% occurred in populations in the southern extent of its distribution:
% populations in the Bay of Fundy, Anticosti Island, and the Atlantic coast of
% Nova Scotia being assessed as Endangered, populations in the south coast of
% Newfoundland assessed as Threatened, And the populations in the New Brunswick
% and Qu\`{e}bec coasts of the Gulf of St. Lawrence assessed as Special Concern
% \citep{Cosewic2010}. On the other hand, northermost populations have shown
% stable, or even increasing population trends, suggesting that, if marine
% survival were to be a factor in many of these declines, these changes are not
% uniformly distributed across all populations.

% While the purported decline of marine survival is mentioned widely in the 
% scientific literature, only relatively few studies have quantified it in detail.
% The basis for this premise is a limited number of time series that exhibit
% temporal declines in a proxy of marine survival (i.e. return rates or
% post-smolt survival).

% 3. Nonetheless, this premise has never been examined in detail
Given the logistical challenges associated with estimating at-sea survival, it
is not surprising that the number of studies that have estimated temporal
trends has been limited. An additional limitation has been the derivation of
proxies (e.g., return rates), rather than direct model-based estimates, of
marine survival.
\citet{Chaput2012a}, for example, examined the return rate of smolts to adult salmon 
as a metric of marine survival, finding that most Canadian populations 
had experienced declining return rates. 
However, examination of trends in return rates alone
can mask changes in differential survival during different years at sea, as well
as changes in the proportion of adults returning after one or two years at sea.
Recently \citet{Olmos2019} suggested that trends in post-smolt
survival, when estimated at the stock unit level, are synchronously declining
for all Atlantic salmon in Eastern North America and Canada, a conclusion ultimately 
grounded on the veracity of highly variable stock-recruitment relationships.

%4. 
In the present study, we compile data on the number of migrating smolts and number of returning adults 
for seven wild Canadian populations of Atlantic salmon to model trends in marine survival.
While some studies have previously used maturity-schedule models to estimate marine
survival for a limited number of salmon populations \citep{Chaput2003b}, none
have incorporated data extending over multiple decades, nor have they examined
trends among more than two or three populations. 
Here, we develop a hierarchical Bayesian model that uses Murphy's maturity
schedule method, in conjunction with informative priors, to estimate yearly
marine survival in salmon. In addition to accounting for observation error in
smolt and return estimates, we estimate the proportion of salmon returning
after one winter hierarchically.

\section*{Methods}

\subsection*{Data}

We obtained time series data of outmigrating smolt and returning adult
abundances for seven Atlantic salmon populations in eastern Canada, encompassing a
wide range of the species' distribution (Fig.~\ref{fig:map}). 
Populations included the LaHave River in the Southern Uplands region of Nova
Scotia (NS), Nashwaak River, New Brunswick (NB), Rivi\`{e}re de la Trinit\'{e} (Trinit\'{e}) and
Saint-Jean rivers in Qu\'{e}bec (QC), and  Western Arm Brook (also referred to as  WAB), Campbellton, and
Conne River, Newfoundland (NL). 
Data were collected in NS, NB, and Newfoundland
and Labrador (NL) by Fisheries and Oceans Canada (DFO) and in QC
by the Minist\`{e}re des For\^{e}ts, de la Faune et des Parcs, Qu\'{e}bec.

\begin{figure}[htbp] \centering
    \includegraphics[width=0.85\linewidth]{figures/rivers-map2.png}
    \caption{Locations of the seven rivers in eastern Canada with time series abundance data of outmigrating smolts and 
    returning adults.} \label{fig:map} 
\end{figure}

\subsubsection*{Smolt and adult return abundance data}

Smolt and adult return abundance estimates originate from a variety of
sources. Smolt estimates from the Trinit\'{e}, Saint-Jean, and Conne populations were obtained using a
mark-recapture approach, while estimates from the ..., WAB, and Campbellton populations
were obtained by direct counts using fish counting fences.
For further details on the data collection methodologies refer to 
\citet{Dempson1991, Venoitt2018} for NL populations, 
\citet{April2018}  for QC populations,
\citet{Jones2014} for NB populations,
and \citet{Gibson2009} for NS populations. 

Yearly return data are often recorded in terms of two size groups: small ($< 63$ cm
FL) and large ($\geq 63$ cm FL) salmon, as these closely represent different
life-history strategies (i.e. 1SW and 2SW), but can be confounded with repeat
spawners of different sizes. To correct for this in returns 
reported as small and large salmon, we estimated the abundance
of 1SW and 2SW returns using yearly scale age data of a subsample of returns:

\begin{equation}
    p_{r,t,a} = \frac{\sum_{s}{(\frac{n_{r,t,s,a}}{n_{r,t,s}} * N_{r,y,s})}}{\sum_{s}{N_{r,t,s}}}
\end{equation}

where $p_{r,t,a}$ is the proportion of annual returns in river $r$, year $t$,
and of spawning history $a$ (either 1SW or 2SW returns); $n_{r,t,s,a}$ is the
number of samples in river $r$, year $t$, of spawning history $a$, and of size
group $s$; $n_{r,t,s}$ is the total number of samples in river $r$, year $t$,
and of size group $s$; and $N_{r,t,s}$ is the returns of salmon
in river $r$, year $t$, and of size group $s$.

\subsection*{Bayesian model}

We developed a hierarchical Bayesian model that uses Murphy's maturity
schedule method, in conjunction with informative priors, to estimate yearly
marine survival in seven populations of Atlantic salmon. We account for
observation error in smolt and return estimates, as well as estimating the
proportion returning after one winter (i.e. \Pg) hierarchically.
There is an identifiability problem in the maturity schedule equations where
the parameter estimates cannot be optimally solved \citep{Chaput2003a}.
However, this issue can be mathematically overcome, at least partially, by
using informative priors for all three marine survival parameters in a
Bayesian framework.
This method requires abundance estimates of smolts as well as abundance estimates
of returning one-sea-winter (1SW) and two-sea-winter adults. With these data,
it estimates three parameters: survival in the first year at sea (\So), survival
in the second year at sea (\St), and the proportion of fish returning after one
year at sea (\Pg). 

Our model does not include repeat spawners and assumes that no fish spend
three or more winters at sea before returning to spawn for the first time.
The model also assumes that mortality in the second winter at sea (\St)
is additional to mortality in the first winter at sea in the previous year, 
and therefore does not account for differences in environmental conditions experienced
between 1SW and 2SW fish of the same smolt cohort during their overlapping first year at sea.
In other words, our model assumes that the decision of returning occurs just before
actually being counted as returns and that \St is any additional mortality in
the subsequent year. 

%\comment{Other assumptions...}

Observed smolt estimates were modelled hierarchically and included
observation error:

\begin{equation}
log(smolts_{obs,t,r}) = log(smolts_{true,t,r}) + \epsilon_{t,r}
\end{equation}

where $smolts_{true,t,r}$ are the true smolt abundances for year $t$ and river
$r$, and $\epsilon_{t,r}$ is the error term, which is empirically derived from
by calculating the yearly coefficient of variation in the empirically derived
smolt estimates (see Table S1 in the Supplementary material). 
Where available, we used population-specific measurement error estimates for smolt abundances; if not 
available, we set measurement error at 5\%. 
The log-transformed true smolt abundances are
normally distributed around a population-level mean and standard deviation:

\begin{equation}
log(smolts_{true,t,r}) \sim Normal(\mu_{smolts,r}, \sigma_{smolts,r})
\end{equation}

where $\mu_{smolts,r}$ and $\sigma_{smolts,r}$ are parameters estimated by the
model for each river.

Once we have yearly estimates of smolt, 1SW, and 2SW abundances, we estimate
marine survival parameters using Murphy's maturity schedule method
\citep{Murphy1952, Ricker1975}:

\begin{align}
    R_{1,t} &= smolts_{true,t-1} * S_{1,t} * Pr_t \label{eq:1}, \\
    R_{2,t+1} &= smolts_{true,t-1} * S_{1,t} * (1 - Pr_t) * S_{2,t+1} \label{eq:2}
\end{align}

where $R_{1,t}$ and $R_{2,t+1}$ are the estimated abundances of 1SW and 2SW
salmon returning in years $t$ and $t+1$, respectively, $smolts_{true,t-1}$ is the
estimated number of outmigrating smolts in year $t-1$, $S_{1,t}$ is the proportion of
salmon surviving in their first year ($t$) at sea, $Pr_t$ is the proportion of
salmon that return to spawn at year $t$, $S_{2,t+1}$ is the survival in their
second year at sea of the same cohort of salmon who did not return to spawn at
year $t$.

However, to allow for normally and log-normally-distributed parameters we
log-transform equations~\ref{eq:1} and~\ref{eq:2} to obtain

\begin{align}
    log(R_{1,r,t}) &= log(smolts_{r,t-1}) + log(Pr_{r,t}) - Z_{1,r,t} \label{eq:3}, \\
    log(R_{2,r,t+1}) &= log(smolts_{r,t-1}) - Z_{1,r,t} + log(1 - Pr_{r,t})  - Z_{2,r,t+1} \label{eq:4} 
\end{align}

where $Z_{1,r,t}$ and $Z_{2,r,t+1}$ are the instantaneous mortality rates.
Process error was included as the standard deviation of the log-transformed
return estimates from equation~\ref{eq:4}:

\begin{align}
log(R_{obs,1,t,r}) &\sim Normal(log(R_{1,t,r}), \epsilon_{1,r}), \\
log(R_{obs,2,t,r}) &\sim Normal(log(R_{2,t,r}), \epsilon_{2,r}) \label{eq:5} 
\end{align}

where $R_{obs,1,t,r}$ and $R_{obs,2,t,r}$ are the observed return estimates
for year $t$ and river $r$ of 1SW and 2SW fish, respectively, $\epsilon_{1,r}$
and $\epsilon_{2,r}$ are the process error terms. 
These error terms are normally distributed with a standard deviation
of 0.01, which are almost equivalent to using a coefficient of variation in
return estimates 1\%, given that return estimates in equations~\ref{eq:3}
and~\ref{eq:4} are log-transformed:

\begin{align}
\epsilon_{1,r} &\sim Normal(0, 0.01) \\
\epsilon_{2,r} &\sim Normal(0, 0.01).
\end{align}

Furthermore, we use instantaneous mortality rates in the model instead of survival probabilities
as the model is more efficient in its parameter search in log-space, and instantaneous rates
are easy to interpret. Instantaneous rates are easily converted to survival probabilities by 

\begin{align}
 S_{1} &= e^{-Z_1}, \\
 S_{2} &= e^{-Z_2}. 
\end{align}

We estimate population-level mean \Pg values around which the yearly \Pg
values are normally distributed. We specify different informative hyperpriors
for \prmu and \prsig based on whether the population is 1SW-dominated or not:

\begin{align}
    logit(P_{r,t}) &\sim Normal(\mu_r, \sigma_r) \\
    \mu_r &\sim 
    \begin{cases}
       Normal(2.3, 0.4),  &\text{for 1SW-dominated populations} \\
       Normal(0, 2.8), &\text{for non-1SW-dominated populations} \\
   \end{cases} \\
    \sigma_r &\sim halfNormal(0, 1).
\end{align}

The priors for $Z_1$ and $Z_2$ are specified as log-normal distributions:

\begin{align}
Z_1 &\sim logNormal(1, 0.22),   \\ 
Z_2 &\sim logNormal(0.2, 0.3).
\end{align}

The model was written in Stan \citep{Carpenter2017} and run in R version 3.6.1
\citep{RCoreTeam2019} using the \texttt{rstan} package version 2.19.2
\citep{StanDevelopmentTeam2019}.
The model was run with three chains and 3,000 iterations, with the first 1,500
discarded as a burn-in. The models were considered to have converged when the
$\hat R$ of all parameters were lower than 1.03 and the effective sample size 
were higher than 500.

\subsection*{Correlations among trends in survival}

%\comment{How do were quantitatively compare trends among rivers? DFA is an
%option, but so is a t-test of averages before and after a certain year}

We looked at the correlation of trends in \So by calculating the Pearson's
correlation coefficient for the Z-scores of these trends. Given that the time
series do not cover the same years, and that some rivers have missing years in
the middle of the time series, pair-wise Pearson's tests were done using only
the years where there is data for both rivers.

\section*{Results}

%\subsubsection*{Model convergence}

\subsubsection*{Trends in marine survival parameters}

Trends in estimates of \So were highly variable within and among rivers
(Fig.~\ref{fig:s1-dual}). The highest median posterior estimates of \So
were for the Nashwaak River in 2006 and 2008, with values of 0.18 and 0.21,
respectively. The lowest median \So estimate was in the Trinit\'{e} in 2001,
with an estimate of \So of 0.007, while the Conne, LaHave, and Trinit\'{e} had
years where estimates of \So varied between 0.01 and 0.02 (Fig.~\ref{fig:s1-faceted}).

Trends among populations also varied: Campbellton,
Saint-Jean, and Western Arm Brook populations showed increases in \So
over time, Trinit\'{e}, Conne, and La Have populations showed decreases,
while at Nashwaak there was an increase in median \So during the early
2000s but a decrease in the 2010s. Yearly estimates of \So had very little
variability for one sea-winter dominated populations (Conne, Campbellton, and
WAB), but were more variable (i.e. wider credible intervals) in the other
populations.

\begin{figure}[htbp] \centering
    \includegraphics[width=0.95\linewidth]{figures/s1-trends-dual.png}
    \caption{Trends in survival in the first year at sea (\So). a) Posterior
        estimates of \So for the seven rivers examined, error bars indicate
        the 90\% credible intervals, and dashed lines denote years of commercial
        fishing moratoria for each province. b) Z-scores of median posterior
        estimates.} \label{fig:s1-dual} \end{figure}

\begin{figure}[htbp] \centering
    \includegraphics[width=0.95\linewidth]{figures/s1-trends-faceted.png}
    \caption{Posterior estimates of \So for the seven rivers examined, error
        bars indicate the 90\% credible intervals.} \label{fig:s1-faceted}
\end{figure}

Estimates of \St were highly uncertain in all rivers, and trends 
were not apparent in most rivers given the large range of the credible
intervals in the yearly estimates (Fig.~\ref{fig:s2-faceted}). The estimates
of \St for the Saint-Jean and Trinit\'{e} were considerably less uncertain
than for the other populations, but showed no apparent trends through time.

\begin{figure}[htbp] \centering
    \includegraphics[width=0.95\linewidth]{figures/s2-trends-faceted.png}
    \caption{Posterior estimates of \St for the seven rivers examined, error
        bars indicate the 90\% credible intervals.} \label{fig:s2-faceted}
\end{figure}

Estimates of \Pg were mostly stable across time, except for the LaHave and
Nashwaak Rivers; The estimates of \Pg were slightly lower in the last four
years than in the previous ones, while in the Nashwaak the posterior estimates
of \Pg in 2012 were much lower than in all other years
(Fig.~\ref{fig:s2-faceted}). Uncertainty in yearly estimates was highest in
the LaHave, Nashwaak, and Trinit\'{e}, and lowest in the 1SW-dominated
populations.

\begin{figure}[htbp] \centering
    \includegraphics[width=0.95\linewidth]{figures/pr-trends-faceted.png}
    \caption{Posterior estimates of \Pg for the seven rivers examined, error
        bars indicate the 90\% credible intervals.} \label{fig:pr-faceted}
\end{figure}

Population-level estimates of \prmu and \prsig varied considerably among rivers. For all
three 1SW-dominated rivers (Campbellton, Conne, and WAB), estimates of \prmu
were very close to 1.0 and had little variability in \prsig (Fig~\ref{fig:prmu-post}).
Estimates of \prmu were the lowest for the two QC rivers, particularly the Saint-Jean (median \prmu = 0.11).
Estimates of \prmu for the Nashwaak and the LaHave Rivers were close to 0.5, with these two rivers having 
the highest estimated values of \prsig, particularly the Nashwaak (Fig~\ref{fig:prmu-post}).

\begin{figure}[htbp] \centering
    \includegraphics[width=1.0\linewidth]{figures/pr-mu-posteriors.png}
    \caption{Posterior estimates of the population-level parameters $Pr_{\mu}$
       $logit(Pr_{\mu})$, and $logit(Pr_{\sigma})$. Dots denote median estimates, while the thick and thing error bars indicate
       the 50\% and 90\% credible intervals, respectively.} 
   \label{fig:prmu-post} 
\end{figure}

\subsubsection*{Correlations}

Most correlations between z-scored trends were not statistically significant, with only 
three out of the 21 pair-wise comparisons having a p-value below 0.01 (Fig.~\ref{fig:s1-corr}a).
When looking at the direction of the correlation, regardless whether they were
significant or not, these spanned both positive and negative coefficients
(Fig.~\ref{fig:s1-corr}b).

\begin{figure}[htbp] \centering
    \includegraphics[width=0.95\linewidth]{figures/corr-s1.png} \caption{
        Correlation among Z-scores of median estimated trends in \So among
        rivers. a) Pearson's correlation coefficients are shown in each square,
        while colouring denotes significance of the correlation ($p \leq 0.01$), b)
        colours denote direction and magnitude of correlation, while asterisks denote significance.}
\label{fig:s1-corr} 
\end{figure}

\section*{Discussion} 

% 1. Main findings
% 2. How they compare to other publications
% 3. Potential reasons
% 4. Caveats
% 5. Future directions

% 1. Main findings
Our results challenge the narrative that marine survival, specifically survival in the first year at sea, is declining
uniformly throughout the range of Atlantic salmon in the northwest Atlantic.
Temporal trends are not consistent among populations. 
Over the time periods for which data were available, some rivers show positive trends in survival in the
first winter at sea (\So) while other exhibit highly variable yet stable trends, and some show
declines. We could not assess trends in the second winter at sea (\St) or
proportion returning as grilse (\Pg), as these parameter estimates were highly
uncertain and were strongly influenced by the priors.
Perhaps there is a need to rethink our understanding of Atlantic salmon
population dynamics in light of the possibilities that (1) any real and persistent decline in marine survival was
experienced by some but not necessarily all populations, (2) reductions in survival might
have occurred over a relatively brief period of time and have not persisted, and (3) marine survival has
been relatively stable, or increasing in some populations for one or more decades.

% 2. How they compare to other publications
Our results are contrary to those of \citet{Olmos2019}, who detected positive
correlations in post-smolt survival among spatially broad stock units. These differences
could be due to a number of reasons: different model specifications and
structure, different methods for estimating covariance, difference in the
spatial scales of data sources (i.e. river vs province scales), and perhaps most importantly the use stock-recruitment relationships
rather than empirical smolt count data to estimate marine survival.
As trends in marine survival during the first winter at sea are highly independent
among rivers on relatively small spatial scale, trends from broader
geographical areas (i.e. province, state, or country-wide estimates) may not
be representative.
Interestingly, our estimates of \So and \St are very similar to those produced
by \citet{Chaput2003b}, and our trends are almost identical for the
overlapping time period that marine survival was estimated for in their study
(1984-1998). 
While \citet{Chaput2003b} separated abundance data for males and females
and assumed their survival rates were the same (to be able to reach an
analytical solution), our study reached almost the same results (albeit with
slightly higher uncertainty), using a Bayesian approach with informative
priors. These overlapping trends obtained with two different methods 
suggest that our method is effective at estimating marine survival.

Trends in marine survival
among populations were compared by \citet{Chaput2012a} using adult return rates.
He found that for 4 of 6 populations examined, return rates in the 1990s 
were lower than those during the 1970s.
\citet{Friedland1993} compared return rates for a number of rivers in eastern
North America between 1973 and 1988, and suggested there are similar trends among these. 
However, the similarity in these trends was driven primarily by two years, 1977 and 1978, which
show concurrent low and high relative return rates across rivers,
respectively. Other years are much more variable relative to each other.
\citeauthor{Friedland1993}'s \citeyear{Friedland1993} time series ends in  
1988; thus there are only a few years for which to assess overlap with the
time series in our study.
In any event, we caution that the pooling of adult return rates \citep{Chaput2012a, Friedland1993} 
can mask inter-annual variability in marine survival,
and hence might not produce an accurate depiction of marine survival trends.
\citet{Dempson2003} described a general declining trend in marine survival for
Newfoundland rivers (except WAB); we drew the same conclusion for 
Conne River but not Campbellton River or WAB. It is not possible to draw broader conclusions
with data from only three Newfoundland rivers, but it seems that among index rivers,
those in Newfoundland are among those with the highest marine survival rates.

% 3. Potential reasons
There are a variety of potential explanations for the lack of synchronous
trends in estimates of \So. 
Marine survival in the first winter at sea could be highly variable between
populations because of the predominance of spatially local environmental drivers of survival (e.g., temperature, predation) 
relative to broader-scale, even ocean-wide, drivers.
The synchrony reported for marine survival trends at broader spatial scales \citep{Olmos2019}
might be attributable to the use of stock-recruitment relationships to estimate survival,
relationships that may have been confounded by changes in recruitment dynamics.
There is some evidence of a correlation between return rate and growth (as
indicated by inter-circuli spacing on scales), where years of poor growth
tended to also be years of poor survival \citep{Friedland1993}, supporting the
idea that environmental variability can affect marine survival.
Furthermore, among European salmon, there is evidence of a positive correlation
between spring temperature in the Norwegian and North Seas and population abundance, suggesting warmer
conditions favour post-smolts \citep{Friedland1998}, based on mapping the
extent of area of suitable temperature (7-13 \textdegree C).

Nonetheless, the causal mechanisms for why warming should affect post-smolt
survival almost certainly differs depending on the difference between
temperature experienced by the post-smolts and their respective
population-specific thermal optima. 
This difference could explain why populations in eastern North America are
declining in the southern part of their range but potentially increasing
further north, and also why some studies find positive correlations between
temperature and abundance \citep{Friedland1998, Friedland1998b, Jonsson2004}
while others find negative ones \citep{Friedland1993, Todd2008}.
Putative associations between temperature and direct estimates of marine
survival warrants further study at the population level.

While there is little evidence that marine survival is density-dependent in
Atlantic salmon \citep{Jonsson1998,Gibson2006}, there could potentially be
some density-dependent processes during parts of the post-smolt migration
period, particularly for populations that are likely to be subjected to
declining per capita population growth rates ($r$) generated by Allee effects.
Exploring relationships between survival and population size could potentially
shed light about the processes that have caused many of the population
declines that have been documented.

Oceanic conditions have been correlated with abundance trends and growth
\citep{Todd2008}, however, the mechanism by which such bottom-up effects, \
mediated by changes in food availability,
affect population dynamics beyond marine survival needs to
be thoroughly reassessed. If marine survival on its own cannot fully explain
trends in abundance, then there are potential carry-on effects of oceanic
conditions that manifest with regards to fresh production. 
For example, adults
that return to spawn after spending suboptimal conditions at sea might be less
likely to make it to their spawning grounds, successfully secure a mate,
produce fewer eggs, or produce eggs with lower per capita fitness than those
produced by adults which grew in optimal oceanic conditions.
As larger females tend to be more productive, in terms of fecundity and total
reproductive energy, than the same weight's worth of smaller females
\citep{Barneche2018}, a small decrease in body condition resulting from bottom-up
impacts on food availability could potentially have disproportionate effects on fecundity
and fitness of the offspring.

Egg-to-smolt survival in Atlantic salmon is highly variable \citep{Klemetsen2003,Chaput2015}
and changes in the oceanic conditions that spawners experience could be
contributing to this variability.
Obviously, there would be a time lag (perhaps as much as a generation) in how such effects
might be manifest at the adult stage.
However, given that most correlations are between relatively monotonic declines
in abundance coupled also monotonic increases in climatic indices
over decadal time scales \citep[e.g.,][]{Friedland1998, Todd2008,
    Beaugrand2012}, it would be expected that this correlations would be
maintained even if salmon abundances were lagged by a generation length.


% 4. Caveats
As with all novel modelling approaches, there are caveats to acknowledge.
The seven populations explored in the present study might not be representative
of regional trends in marine survival. However, there are no other
long-term time series of smolts and adult returns to draw inferences from.
While there are analytical issues associated with the estimation of \So, \St, and \Pg,
the assumption that \St is additive to \So could produce unrealistic results.
We know there is a period of a few months where 1SW
returns are subject to a different environment than those salmon that will
return as 2SW the next year. 
While this is not ideal,
overcoming this assumption would require an additional parameter to be
estimated, or an additional assumption as to what proportion of \So is not
additive to \St (as the returning 1SW adults do not experience the same
environment when they return to their natal streams as those fish who stayed
at sea for an additional winter before returning to spawn).

Secondly, the assumed hierarchical structure might not be the most appropriate
for modelling smolt abundances. This approach results in shrinkage to the
mean, which means that the variability of yearly smolt estimates is less than
it would be if not modelled hierarchically. 
That said, this assumption is likely to result in more conservative trends in marine survival, as the
variation is smolt estimates is reduced.
It is important to note that the hierarchical structure in the estimation of
\Pg seems like a reasonable assumption given that the probability of returning
as grilse has a genetic component associated to it \citep{Aykanat2019} and is
not expected to vary much, within a population, among years.

% 5. Future directions
Perhaps a reframing of the issue of marine survival is key to furthering our
understanding of Atlantic salmon population dynamics. Marine survival may not
have declined consistently, and over the same time periods, across all
populations. 
But the fact that it has remained at roughly similar levels as it was
previously \emph{despite} reduced commercial fishing mortalities, suggests
that there may well be an interaction between small population size (small
relative to unfished population size or carrying capacity), recovery
potential, and environmental stochasticity that has not been fully explored in
Atlantic salmon. 
All else being equal, relatively small populations are more vulnerable to
demographic, environmental, and genetic stochasticity than large populations
\citep{Lande1993, Hutchings2015}. Interactions between population size and the
demographic consequences of environmental stochasticity appear to have
affected recovery in many marine fishes that have exhibited impaired recovery
since mitigation of the threat posed by fishing mortality
\citep{Hutchings2017, Hutchings2020}. The possibility that similar
interactions may be impairing the recovery of wild Atlantic salmon merits
study.

\section*{Acknowledgements}

% We would like to thank Geir Bolstad, G\'{e}rald Chaput, Brian Dempson, and
% Martha Robertson for their useful discussions on estimating marine survival
% in Atlantic salmon, and Amanda Kissel for her helpful comments on the
% manuscript. Brian Dempson and Geoff Venoit for providing the data Conne
% River data.
We would like to thank G\'{e}rald Chaput for his useful discussions on
estimating marine survival in Atlantic salmon, Carmen David for her comments
on the manuscript, and Sean Anderson for his help with implementing the
non-centered parameterization of the model. This research was supported by the
Atlantic Salmon Conservation Foundation and the Atlantic Salmon Research Joint
Venture.

\section*{Conflicts of Interest}

The authors declare no conflicts of interest.
 
\bibliography{subset}

%\input{ms.bbl}

\end{document}




\end{document}




\end{document}




\end{document}


